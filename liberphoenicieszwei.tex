\documentclass[a5paper,8pt]{book}
\usepackage[ngerman]{babel} % Deutsch
\usepackage[utf8]{inputenc} % Dateiencodeing - wichtig
\usepackage{graphicx} % Bilder
\usepackage{yfonts,color} %Initialien
\usepackage[T1]{fontenc}
\usepackage{calligra}
\usepackage{amssymb}
%\newcommand*\initfamily{\usefont{U}{RoyalIn}{xl}{n}}
\pagestyle{headings}

\begin{document}

%

\newfont{\hge}{hge scaled 1500}


\begin{center}
Liber Phoenicies II\\
erste überarbeitete Version\\

\vspace{30mm}

Florian Phelleas Ph"onixflug\\
Magus Galad, Akademia Clavis Mundi Grenzbrueckensis\\

\vspace{10mm}

F"ur meine Eltern, meinen Bruder, meinen Neffen und meine Nichte\\
F"ur Lix, Ayla und Kaya\\

\vspace{10mm}

Im Gedenken an: \\
Magister Dereon Garad von Ayd Owl \\
Varn Chamounde Adeptus Cantus Harmonae \\
Scolarius Via Pugna Frederic von Ayd Owl \\
Miasaris Toreville Adepta der Academia zu Nektor \\
262 nach Jeldrik\\
\end{center}

\newpage

Vorwort: \\
\yinipar{\color{red}D}ies ist das erste vollst"andige Buch, da"s ich publiziere. Ich hoffe, da"s noch viele folgen werden, doch werde ich hiermit den Anfang machen. Es ist keine zwei Jahre her, seit dem ich an der Akademia Clavis Mundi Grenzbrueckensis aufgenommen wurde und kein ganzes Jahr, seit mir die Ehren und W"urden des Magus Minor verliehen wurden. Dennoch habe ich in den Jahren am Hofe des Markgrafen Jerevan von Arkenwald und in meiner Zeit an der Akademia Ayd Owl aus Engonien einiges Wi"sen gesammelt, da"s sich lohnt niederzuschreiben.


Dieses Buch ist mehr eine Sammlung von fortgeschrittenem und manchmal recht spezialisiertem Wi"sen. Es ist ausdr"ucklich nicht zur Lekt"ure f"ur Anf"anger geeignet und viele beschriebene Prozeduren k"onnen sich als "au"serst gef"ahrlich herau"stellen, sollten sie nicht von fachkundigem Personal durchgef"uhrt werden.
Da es kein "ubergeordnetes Thema und keinen roten Faden gibt wird sich die Lekt"ure unter Umst"anden als nicht ganz einfach herau"stellen, daf"ur wird der geneigte Leser aber unter anderem auch mit Wi"sen belohnt werden, da"s sich in keinem anderen arkanen Werk finden la"sen wird.
Meine Freunde und meine Familie werden in wohlgemeintem Zweifel den Kopf sch"utteln, aber in meiner kurzen Zeit an der Clavis Mundi ist mir der Ruf angetragen worden Praksisnah zu sein. Auch wenn mir dies in meiner Zeit der Wanderschaft nie jemand vorgeworfen h"atte und ich mich selber nicht als solchen sehe, so mu"s ich doch gestehen, bei objektiver Lekt"ure meiner Texte, die ein oder andere Phrase zu entdecken, die dem wohl gerecht wird. Drum will ich an dieser Stelle eingestehen, da"s ich tats"achlich den holden Pfad des Sammelns von Weisheit um der Weisheit willen verla"sen habe und meine Suche nach Wi"sen zur Zeit von nur all zu  mundanen Zielsetzungen bestimmt wird.
Solcherart gest"arkt kann ich Ihnen nun, nicht ohne einen gewi"sen Stolz, die akademischen Fr"uchte meines bisherigen Lebens pr"asentieren.

Mit akademischen Gr"u"sen \\

Florian Phelleas Ph"onixflug, \\
Magus Minor Akademia Clavis Mundi Grenzbrueckensis im Jahre 259 nach Jeldrik \\


\tableofcontents



\chapter{ Magietheoretische Publikationen}

\section{Pyrodynamik}


In der Pyrodynamik beschäftigen wir uns vornehmlich mit Feuer oder stark feuerhaltigen Substanzen. Die komplette Pyrodynamik ist sowohl Elementarmagisch, als auch hermetisch konsistent, in dem man die Phlogiston/Caloricum-Theorie zur Übersetzung benutzt. Ich werde hier einen Elementarmagischen Ansatz benutzen. Wenn sie eine rein thaumaturgische Ansicht bevorzugen, ersetzen sie einfach nur das Wort Feuer durch Caloricum.

Feuer ist das leichteste der Elemente. Leichter noch als Luft, welches wir eigentlich in unserem alltäglichen Sprachgebrauch und normalen Weltverständnis als leicht wahrnehmen. Auf der anderen Seite ist Luft flüchtiger, als Feuer, wodurch, vorallem in Kombination mit dem Leichtigkeitsunterschied, eine charakteristische und wichtige Dynamik entsteht. Feuer ist nach der Luft das zweitflüchtigste Element.
Ohne jetzt mit Elementaristen in allzugroße Glaubensdiskussionen verstrickt zu werden möchte ich diese Behauptung nicht 
beweisen, sondern sie durch die, in der Pyrodynamik verwendete Definition von leicht und flüchtig, anxiomatisch begründen.\\

Leicht := Je leichter ein Objekt ist, desto mehr strebt es nach Oben \footnotemark[1].
Können sich Objekte ungehindert bewegen, dann ordnen sie sich ihrer Leichtigkeit (bzw. der Schwere genannten reversiblen 
Leichtigkeit) entsprechend an.\\

Flüchtigkeit := Flüchtigkeit ist ein Maß für den inneren Zusammehalt eines Stoffes. Je mehr ein Stoff dazu tendiert, bei 
gleichbleibendem äußeren Einfluss zu dissoziieren, desto flüchtiger ist er.\\

\textbf{Beispiel:}
An dieser Stelle möchte ich auch schon direkt die ersten Beispiele geben um das Prinzip zu verdeutlichen. Bekannterweise 
steigt warme Luft nach Oben. Sie ist also leichter, als kalte Luft. Das liegt laut den fundamentalen Prinzipien der 
Pyrodynamik an dem Gehalt an Feuer in der Luft. Warme Luft enthält mehr Feuer, als kalte Luft. Daher ist warme Luft 
leichter und steigt nach Oben.

Auf die gleiche Weise tendiert kalte Luft dazu länger an einem Flecken zu verweilen. Während sich von einem Feuer erwärmte 
Luft schnell in einem Raum ausbreitet und auch schnell wieder verfliegt, sollte man vergessen die Fenster geschlossen 
zu halten, tendiert kalte Luft, wie zum Beispiel Nebel\footnotemark[2] dazu so lange zu verweilen, bis sie von den Sonnenstrahlen 
erhitzt wird\footnotemark[3].

In der Phlogiston-Theorie versehen wir das Caloricum mit einer negativen schweren Masse. Stoffe werden beim Verbrennen 
bekannterweise schwerer, da beim Verbrennungsprozess Phlogiston frei wird und so die Gesamtmasse zunimmt. Warme Substanzen 
sind dementsprechend leichter, da sie Caloricum an sich, in Form von Wärme, binden.

\subsection{Pyrostatik:}
Feuermagie oder feuerhaltige Magie wird gerne und oft benutzt, vorallem zur Energieübertragung. Energie kann leicht und mit 
relativ geringem Verlust in Feuer, genauer gesagt Feuerbewegung (aber das ist eine eigenständige Vorlesung), umgewandelt 
werden und umgekehrt. Dies ist eine der fundamentalen Prinzipien der Pyrostatik, welcher wir uns widmen wollen, bevor wir 
zur Pyrodynamik übergehen.
Wir können jeder Art von Materie einen Pyrostatischen Koeffizienten zuordnen, welcher linear beschreibt, wieviel Energie 
und Zeit wir aufwenden müssen um ein durchschnittliches Element durch ein Feuerelement zu ersetzen. Der Pyrostatische 
Koeffizient ist eine reelle Zahl zwischen Null und Eins und besonders hoch bei leicht verfeuerbarer Materie und sehr 
niedrig bei entsprechend resistenten Substanzen. Das Extremum Eins nimmt er ausschließlich bei elementarem Feuer und das 
Extremum Null niemals an.
Für die Anwendung ist der Pyrostatische Koeffizient nur auf bestimmten Intervallen nützlich, denn er hat leider die 
Angewohnheit solcherart Materienspezifisch zu sein, dass er sich verändert, sobald sich die elementare Zusammensetzung des 
untersuchten Stoffes ändert.

\textbf{Beispiel:}
Zum Beispiel unterscheiden sich der Koeffizient eines kalten Stückes Holz und der eines handwarmen Stückes nur 
unwesentlich, doch sobald beim Holz eine 
gewisse Sättigung an Feuer erreicht ist ändert sich die Pyrostatik sprunghaft und der Koeffizient wird in diesem 
Spezialfall sprunghaft größer. Im Volksmund sagt man, das Holz brennt.

\footnotetext[1]{Oben ist natürlich eine Frage des Bezugssystems und der Topologie des Raumes. Ich gehe hier der 
Einfachheit halber von einem einfach orientierten Bezugssystem aus und bitte den geneigten Lesen dieses Attribut selber 
bei Bedarf zu verallgemeinern.}

\footnotetext[2]{Der natürlich erschwerenderweise auch noch recht viel Wasser enthält.}

\footnotetext[3]{Vorausgesetzt wir haben keine eigenständige Luftbewegung.}

Dieses Beispiel führt uns direkt zur nächsten fundamentalen Eigenschaft der Pyrostatik, der pyrostatischen Äquivalenz. Da 
Feuer ein sehr flüchtiges Element ist neigt es dazu eben keine Accumulationen zu bilden, sondern sich gleichmäßig zu 
verteilen. Ein warmes Objekt wird so lange Feuer an die Umgebung abgeben, bis es selber die gleiche Temperatur erreicht 
hat, wie seine Umgebung. Dabei wird die Menge an Feuer einer bestimmten Substanz immer mit dem pyrostatischen Koeffizienten 
gewichtet.

\textbf{Beispiel:}
Um ein Bett im Winter zu wärmen legt man einen Ziegelstein ins Herdfeuer, oder auf den Ofen. Der Ziegelstein nimmt Feuer 
aus dem Ofen auf, bis er die gleiche Temperatur erhalten hat, wie der Ofen. Dann, unter den Decken platziert, gibt er das 
Feuer an das Bett und die darin befindliche Person ab, bis er seinerseits die Temperatur des Bettes angenommen hat.
Würde man gleiches mit einem gleichgroßen Block Eisen (oder Kupfer) machen, die einen höheren pyrostatischen Koeffizienten 
haben,  dann würde sich das Metall zwar sehr viel schneller auf die Temperatur des Ofens erhöhen, aber auch genausoschnell 
wieder abkühlen und nicht so viel Feuer speichern können, wie der Stein.

Wie wir in diesem Beispiel auch schon angedeutet haben hängt die Geschwindigkeit auch mit dem pyrostatischen Koeffizienten 
zusammen, oder besser gesagt mit dem Gradienten der Pyrostatischen Koeffizienten. Je größer der Gradient, desto schneller 
der Austausch von Feuer. Außerdem gibt das allgemeine Gradientenfeld darüber hinaus die Richtung des Feueraustausches an.

Beispiel:
Es gibt Kochpfannen mit Metall und andere mit Holzgriff. Der Holzgriff ist dazu da die Hände vor der heißen Pfanne zu 
schützen. Da der Metallgriff einen hohen pyrostatischen Koeffizienten hat, nimmt er schnell von der heißen Pfanne Feuer 
auf, während der Holzgriff mit dem niedrigen Pyrostatischen Koeffizienten dafür länger braucht.
Da außerdem der pyrostatische Faktor des Holzgriffes näher an dem der umgebenden Luft liegt, als der des Metalls wird 
bei dem Holzgriff außerdem eine signifikante Menge des Feuer eher an die umgebende Luft abgeführt, als an den Griff 
selber, was die Zeit außerdem noch verlängert.\footnotemark[4]


\footnotetext[4] eine interessante Anmerkung an dieser Stelle ist die Anwendung des (Kampf)Zaubers um eine Waffe zu erhitzen. Im 
Allgemeinen ist es mit diesem Zauber auch möglich hölzerne Waffen und dergleichen zu erhitzen, doch in der Anwendung 
muss man leider oft beobachten, dass dies in vielen Gegenden oder bei manchen Waffen einfach nicht zu funktionieren 
scheint.
Dies liegt daran, dass der Gradient sehr leicht durch elementarmagische Felder beeinflusst werden kann und mitunter 
in manchen Gebieten der elementarmagische Einfluss so hoch ist, dass das projezierte Feuer schon abgeführt wird, bevor 
es sein Ziel erreicht.



Ein hohes lokales Gefälle und ein hoher und schneller Austausch von Feuer kann selbst von mundanen Personen als Flammen, 
Licht, oder eine Rötung des entsprechenden Werkstoffes wahrgenommen werden, wobei dort die Quantifizierung nicht immer 
eindeutig ist. Auch recht heißes Metall kann mundanen Personen als unverfärbt erscheinen und die Erkentniss für ihre Augen 
setzt erst bei einem noch höheren Gradienten des Feueraustausch an.


\subsubsection{Verbrennen}

Selbst der Austausch von Feuer kann mit einer quasistatischen Näherung eben noch als Pyrostatik interpretiert werden, aber 
spätestens beim Verbrennen werden wir nun an die Grenze zur Pyrodynamik stoßen.
Als Verbrennen beschreiben wir den Übergang von einem Stoff zu einem anderen durch die Abgabe von Feuer. Trivialstes 
Beispiel dafür ist das Verbrennen eines Holzscheites zu Asche.
Beim Verbrennen werden Stoffe im Allgemeinen schwerer, was natürlich darin begründet ist, dass Feuer-Elemente des 
Ausgangsstoffes durch andere Elemente ersetzt werden, die zwangsweise natürlich alle schwerer sind, als Feuer.
Im Allgemeinen ist der Pyrostatische Gradient zwischen Ausgangs und Endstoff so hoch, dass er einerseits eine 
kontinuierliche weitere Verbrennung aufrechterhalten kann und andererseits sogar für Mundane sichtbar ist.
Um die pyrostatische Verbrennung zu initialisieren muss dem Ausgangsstoff eine charakteristische Menge an Feuer zugefügt 
werden. Wie viel dies genau ist richtet sich sowohl nach dem Pyrostatischen Koeffizienten, als auch nach der Elementaren 
Konfiguration des Ausgangs, des Endstoffes und der Umgebungsstoffe. Da bei dem Austausch von Elementen und somit der 
Elementaren Transmutation von einem Stoff zu einem anderen, der Gradient ebenfalls durch die umgebenden Elemente verändert
wird beeinflussen diese die Transmutation, auch wenn sie selber keinen Veränderungen eingehen.
Zusätzlich zu der quasistatischen Näherung müssen wir uns hier auch noch einer quasilokalen Näherung bedienen. Denn die 
initialisierende Menge Feuer ist proportional zu dem Teil der Fläche, in dem sie angewendet wird. In der Regel wird man um 
ein Stück Holz zu verbrennen nicht dem gesamten Stück so viel Feuer zuführen, dass es verbrennt, sondern man wird einem 
lokal begrenzten Stück so viel zuführen, dass das lokal begrenzte Stück anfängt zu brennen. Wenn nun bei dem Initialstück, 
die Verbrennung eingesetzt hat, wird die dabei frei werdende Menge Feuer wiederum umgebende Flächen initialisieren und so 
weiter.
Die Verbrennung wird nach der elementarmagischen Theorie dadurch initialisiert, dass man sich des anxiomatischen Begriffes 
der Flüchtigkeit bedient. Erhöht man den Anteil des Feuer in einer gewissen Substanz, so neigen die Elemente des Feuer 
innerhalb des Substanz dazu sich gegenseitig abzustoßen. Ist nun der Gradient dieser Feuer zu Feuer Abstoßung höher, als 
der Feuer zu Umgebung Abstoßungsgradient, so werden wieder Feuerelemente frei. Daher funktionieren die meisten 
Verbrennungen auch besser, wenn der zu verbrennende Stoff von viel Luft umgeben ist. Da die Luft das einzige Element 
darstellt, dass noch flüchtiger als Feuer ist, ziehen die Luftelemente noch zusätzlich Feuerlementente aus dem Stoff,
während alle anderen Elemente, vorallem Erz (und/oder Wasser), sie zurückstoßen würden.



\subsubsection{Pyrostatische Formen}

Die topologische Form von Feuer wird im Allgemeinen durch das Wechselspiel mit ihrer Umgebung, sprich den umgebenden anderen 
Elementen, bestimmt.
Das feurige Firmament ist hierbei sicher das einfachste und vertrauteste Beispiel, dass jedem Leser bestens bekannt ist. 
Sowohl Sonne, Sterne, als auch Kometen nehmen im Allgemeinen eine Kugelform an, da sie selber, fast ausschließlich aus 
Feuer bestehend, vollständig von Luft umgeben sind. Da sie weniger Flüchtig sind, als die umgebende Luft streben sie 
danach, relativ zur Luft, nicht zu dissoziieren und eine möglichst kleine Oberfläche zu bilden, was im euklidisch 
dreidimensionalen Fall natürlich eine Kugel ist.
Gleichzeitig sind sie leichter, als die umgebende Luft, was ihre Position am Himmel in approximativ erster Näherung 
hinreichend beschreibt.

Erdgebundenen pyrostatische Formen sind im Allgemeinen komplexerer Natur, da sie, im Gegensatz zum feurigen Firmament, 
dem Wechselspiel mehrerer Elemente unterliegen. Ein statisches, erdgebundenes Feuer kann man topologisch am ehesten mit 
einer Trapezartigen Grundstruktur mit, nach oben gerichteten, Fortsätzen beschreiben. Die nach oben gerichteten Fortsätze 
werden im Allgemeinen als Flammen bezeichnet.
Der trapezartige Körper entsteht hauptsächlich durch das Wechselspiel mit Humus und Erz bzw. erdehaltigen Sustanzen (je 
nachdem an welche Elementartheorie sie glauben). Durch die extrem hohen Leichtigkeitsgradienten wird das Feuer zur 
kompakten Formgebung gezwunden, wodurch der Rumpf der Flamme entsteht. Die Flammen als solches werden durch das 
Wechselspiel des Flüchtigkeits- und Leichtigkeitsgradienten der Feuer und Luft Grenzschicht bestimmt. Der 
Flüchtigkeitsgradient zeigt nach unten und versucht die Form des Feuers auf eine kompakte erdgebundene Form zu bringen, 
während der Leichtigkeitsgradient nach oben wirkt und das Feuer über die Luft erheben will. In einer homogenen Flamme 
würde sich ein Gleichgewicht und eine einzelne kompakte Form entstehen, die je nachdem, ob der Leichtigkeits- oder 
Flüchtigkeitsgradient höher ist erd- oder himmelsgebunden wäre.
Da nun aber fern der Theorie kein Feuer homogen ist bilden sich Gebiete mit verschiedenen Verhältnissen zwischen dem 
Leichtigkeits- und dem Flüchtigkeitsgradienten heraus. Jene lokalen Gebiete haben verschiedene Dynamiken und 
Anordnungsparameter, ihren superpositionierten Gradientenfeldern entsprechend.
Das führt in der Regel dazu, dass ein Erdgebundenes Feuer sich nicht wie ein kompakter Makroskopischer Körper verhält, 
sondern kontinuierlich Teile als Funken und/oder Flammenzungen abspaltet.


\subsection{Übergang zur Pyrodynamik}

Nun müssen wir sogar den quasistatischen Fall verlassen und vollends zur Pyrodynamik übergehen. In der klassischen 
Lehrmeinung wird dies anhand der pyrodynamischen Formen gemacht, woran auch ich mich hier halten will.
Wie der geneigte Leser bei den Anmerkungen zum feurigen Firmament bereits bemerkt haben wird ist ein Komet natürlich nicht 
rund (bzw. wenn wir dreidimensional euklidisch rund meinen, dann sagen wir im Alltag meist “kugelförmig”) sondern hat einen 
(oder ggf. mehrere) Schweif. Diesen Schweif können wir nicht mehr, wie im quasistatischen Fall des erdgebundenen Feuers mit 
der Interaktion verschiedener Elemente oder der inhomogenität erklähren (da hier die Elemente Feuer und Luft fast in 
Reinform vorliegen), sondern müssen die Geschwindigkeit mit einbeziehen.
Konkret herrscht bei einem Kometen das gleiche Gleichgewicht, wie beim oberen Teil eines erdgebundenen Feuers, mit dem 
Unterschied, dass sich die Gradienten in Bezug auf die Oben/Unten Richtung auf Geodäten, also im Äquilibrium befinden. Im 
Allgemeinen sagen wir dazu “fest himmelsgebunden”.
Daher bleibt allein die Geschwindigkeit der Bewegung des Kometen durch die Luft als Störfaktor für die ansonsten statischen
Gradienten. Die Geschwindigkeit wirkt nicht überall gleich, da sie keine Rotationssymetrie bezüglich des Kugelmittelpunktes 
aufweist. Betrachten wir den zweidimensionalen Schnitt durch die Kugel, so wirkt sich die Geschwindigkeit oben und unten 
stärker auf den Gradienten aus, als in der Mitte. Die Wechselwirkungen zwischen Luft und Feuer sind ja im Allgemeinen 
langreichweitig und nicht auf die Oberflächenelemente beschränkt. Während jetzt den Luftelementen in der Mitte eine Anzahl 
an Feuerelementen multipliziert mit dem Radius des Kometen entgegensteht, so nimmt die Anzahl der Elemente nach oben und 
unten hin entsprechend ab, was sich natürlich auf den Gradienten und damit die Formgebung auswirkt. Ferner ist natürlich 
die Wechselwirkung zwischen den Elementen nicht instantan, sondern durch die elementare Austauschgeschwindigkeit begrenzt, 
wodurch die Gradientenmodifikation entsprechend 
gewichtet wird.
Approximativ nimmt die Geschwindikeitsstöhrung durch die Bewegung in erster Näherung eine paraboloide Form, die sogenannte 
Laminarströhmung an, die mit der absoluten Geschwindigkeit als Faktor gewichtet wird.

Bevor wir diese Erkenntnis auf den Kometen anwenden möchte ich diese Erkenntniss am Beispiel eines Feuerstrahles näher 
verdeutlichen. Ein Feuerstrahl ist die kontinuierliche Ausstoßung von Feuers in eine Richtung und bewegt sich mit einer 
anfänglichen gleichen Geschwindigkeit durch die Luft. Sobald die Feuerelemente nun aber in Wechselwirkung mit der umgebenden
Luft kommen ändern sich ihre jeweiligen Gradienten und dementsprechend auch ihre Geschwindigkeiten. Die äußeren Elemente 
werden langsamer, während die inneren nahezu gleich schnell bleiben. Dadurch entsteht eine nahezu perfekte Paraboloide 
Form, deren Grad und Exzentrizität von den Anfangsbedingungen gegeben werden, so dass der Feuerstrahl eben keine Röhre 
(oder Lanze) ist, wie wir es eigentlich gerne hätten um seine Energie möglichst effizient zu fokussieren.

Wenn wir also nun konkret eine Feuerkugel mit einer paraboloiden Laminarströmung überlagern, dann bekommen wir vorne eine 
elliptische Form und der vordere Teil der Kugel wirkt leicht deformiert, während wir hinten eine konkav, elliptische Form 
bekommen sollten. Dies wäre auch der Fall, wenn wir die Wechselwirkung der Feuerelemente oben und unten vernachlässigen 
könnten, die durch Luftelemente getrennt sind. Ein Komet würde so stehts zwei Schweife ausbilden, die homogen aus Feuer 
bestehen und die Form der Spitzen zwei inneinander verschränkter Parabeln hätten.
In der Realität können wir allerdings die Wechselwirkung der Schweife nicht vernachlässigen und ihre Wechselwirkung und die 
Wechselwikrung mit der Luft im Zwischenraum, gewichtet mit der Geschwindigkeit ergibt ein hochgradig inhomogenes Ensemble,
wie bei den Flammen eines erdgebundenen Feuers. Daher werden sich bei einem reellen Kometen in der Regel die beiden 
idealisierten Schweife zu einem einzigen vereinen, der aus einer Ansammlung von Gebieten aus Luft und Feuer besteht.

\subsection{Corporeale Fuerformen}

Zum Ende dieser Vorlesung möchte ich noch einen kleinen Ausblick auf die Wechselwirkung von Feuerformen mit den 
metaphysischen Kräften geben und den Leser zum eigenständigen Nachforschen und Reflektieren anregen. Dieser Abschnitt 
ist daher nur zur Einführung gedacht und in keinster Weise vollständig oder abgeschlossen.

Als corporeale Feuerformen bezeichnen wir im Allgemeinen Wesen und Manifestationen metaphysischen Ursprungs, die einen 
Körper aus elementarem Feuer besitzen oder einen, dessen dominierende Hauptkomponente Feuer ist. Zu Ersterem zählen z.B. 
Feuerelementare, -geister, -herren, -dschinne, noncorporeale Phönixe, Drachen und Harpyien\footnotemark[5] etc. pp. Zu letzterem reguläre 
Drachen, Phönixe, Feuersalamander usw..

\footnote[5]{Ja, eine noncorporeale Feuerhapyie, ein noncorporealer Feuerdrache oder ein noncorporealer Phönix sind corporeale 
Feuerformen. Das Adjektiv bezieht sich jeweils auf das Attribut Feuerform oder Erd/Humuskörper welche vollständig disjunkt 
sind. Achten sie in der akademischen Ausdrucksweise stets auf ihre Definitionen und Anhängigkeiten.}


Allen ist gemein, dass neben der physischen Komponente ihrer Körper, also dem Zusammenspiel, der aufbauenden Elemente, 
ihre Form darüber hinaus durch ihre Metaphysischen Qualitäten, insbesondere des Lebens und der Magie, bestimmt wird.
Wie bei allen Metaphysischen Fragestellungen gerät hier die Elementartheorie an ihre Grenzen und da wir uns Anfangs der 
Vorlesung nicht auf ein vier-, fünf- oder sechskomponentiges Elementarsystem (oder irgendein anderes) festgelegt haben, 
können wir nun leider nicht mehr consistent elementartheoretisch argumentieren.
Um trotzdem nicht auf den letzten paar Schritten den Grundtenor dieser Vorlesung zu verändern will ich mir mit folgendem 
Postulat behelfen, welches der Leser bitte innerhalb dieser Vorlesung (und nur innerhalb dieser Vorlesung) als gegeben 
annehmen soll.\\

\textbf{Postulat I (über die Metaphysik der Elementartheorie im Rahmen der Pyrodynamik):}
Metaphysische Kräfte wirken weder auf die Zusammensetzung, noch die Transmutation von Elementen, sondern ausschließlich 
auf deren eigensinnige und globale Dynamik.\\

Bislang funktionierte die gesamte Pyrodynamik ohne zuhilfenahme von Magie und andere Metaphysischen Komponenten und die 
Dynamik eines mundanen Feuerballs ist in erster Näherung dieselbe, wie die eines magisch erzeugten, doch nun wollen wir 
mit diesem Postulat ein weiteres Essemble von Möglichkeiten aufzeigen.

Die pyrostatischen und -dynamischen Formen werden durch den Flüchtigkeits- und Leichtigkeitsgradienten bestimmt, welche 
durch die Wechselwirkung mit den anderen Elementen ausreichend beschrieben werden. Nun sind aber natürlich beides 
dynamische Eigenschaften und unterliegen damit obigem Postulat und der Beeinflussung durch metaphysische Quellen.

Wenn wir schon ohne metaphysische Einflüsse so viele Komponenten hatten, dass es sich kaum asymptotisch beschreiben ließe, 
so wachsen uns spätenstens jetzt jegliche Freiheitsgrade über den Kopf hinaus. Daher möchte ich nicht versuchen dieses 
Phänomen gänzlich zu untersuchen, sondern lediglich mit einem Beispiel eine Anschauung vorgeben.

\textbf{Beispiel:}
Der Corpus eines Feuerelementares besteht fast ausschließlich aus reinem Feuer, dass nach dem bisher gelernten bei 
umgebender Luft Kugelform annehmen und bei jedweder anderen Umgebung dissoziieren sollte. Nun nehmen allerdings 
Feuerelementare meist eine gefällige Form und nicht ihre pyrostatische Idealform an. Die erreichen sie, indem sie durch 
ihre metaphysischen Eigenschaften (in diesem Fall wiederum Lebenskraft und Magie) ihre entsprechenden lokalen Gradienten 
so ändern, dass eine gefällige Form entsteht und sie z.B. statt als Feuerball, als annähernd humanoides Wesen erscheinen.

Diese Möglichkeit steht natürlich nicht nur corporealen Wesen, sondern auch jedweder corporealen Form zur Verfügung, seien 
es Zauber, Seelenfeuer oder gar clerikalen Feuermanifestationen.

\subsection{Finale Worte:}
Ich hoffe Sie durch diesen kleinen Ausblick in die Pyrodymanik zu eigenen Nachforschungen inspiriert und Ihnen gleichzeitig 
das Handwerkszeug für weiterführende Literatur an die Hand gegeben zu haben.
Viele Feuerelementaristen werden auf intuitivem Wege bereits alles hier gesagte verinnerlicht haben, aber ich hoffe, dass
auch sie erfreut sein werden nun ein kleines theoretisches Konstrukt für ihre alltäglichen Umgangsformen an der Hand zu 
haben.
Alle anderen Elementaristen und Hermetiker werden in Zukunft durch diese Einführung vielleicht ein besseres Verständniss 
für die Arbeit und Werke der Pyromantie erhalten.
Zumindest hoffe ich das.
Wie immer ist akademischer Austausch und Reflektion über dieses Thema sehr willkommen und ich bin über die üblichen Wege 
erreichbar.

Mit akademischen Grüßen
Magus Galad Florian Phelleas Phönixflug

\newpage


\section{\fraklines Di"sertation Florian Phelleas Ph"onixflug}

Vorwort: 

\yinipar{D}ies ist eine Mitschrift meiner Di"sertation und Antritsrede an der Akademia Clavis Mundi Grenzbrueckensis. Sie wurde verfa"st von meinem Schreiber Answin aus Tibur, der sich wieder einmal die k"unstlerische Freiheit genommen hat zu editieren und literarische Erg"anzungen vorzunehmen. Ich habe mich entschieden diesen Bericht solcherart subjektiv gef"arbt einzubinden um dem geneigten Leser eine eigene Meinungsbildung frei nach Stauffer zu erlauben.

\newpage

\subsection{\fraklines Di"sertation Florian Phelleas Ph"onixflug Deterministisches Schicksal und Chaos}

\yinipar{E}s war noch Winter in Grenzbrueck und nur zaghaft lie"sen sich die ersten Anzeichen des Fr"uhlings au"serhalb der Mauern der Akademie sehen. Florian war nach l"angerer Abwesenheit gerade mal eine Woche wieder an der Akademie und hatte sich zusehens ver"andert. Sein K"orper wirkte schw"achlich und ausgezehrt und seine Haare waren grau-wei"s geworden, aber sein Geist war umso sch"arfer. 
Er hatte mit der Akademieleitung geredet und ihm wurden volle Ehren und W"urden eines Magus Minor zugestanden. Nat"urlich noch vorbehaltlich seiner Di"sertation, die auch gleichzeitig seine Antrittsvorlesung an der Akademie werden sollte. Eine der "altesten und wichtigsten Traditionen der Akademischen Welt lag nun vor ihm. Bei seiner Di"sertation und der darauf folgenden wi"senschaftlichen Disputation w"urde sich entscheiden, ob die Magister und Senatoren ihn als einen Gleichen unter Gleichen aufnehmen w"urden, oder ob ihm die Zukunft eines unbedeutenden Wandermagiers bl"uhte.

Der alte Florian h"atte wohl vor Angst geschlottert, obwohl ein akademischer Disput sein Fachgebiet gewesen w"are, aber der neue Florian war kalt wie Eis, als er beobachtete, wie sich langsam der H"orsaal f"ullte. Vor den Toren und in den Hallen stand seit Tagen sein Vortrag angeschlagen ``Deterministisches Schicksal und Chaos'' alle waren eingeladen, aber es w"urde auf die ankommen, die in der ersten Reihe sitzen w"urden.
Die Magister und Senatoren, seine zuk"unftigen Kollegen, wenn alles glatt lief.
Mit seiner Magnifizienz rechnete er nicht, D \`{A}shencourt w"urde au"ser Hauses sein, genauso, wie der vermi"ste Palantor, aber wer von den Hohen Damen und Herren Senatoren w"urde kommen? Wenn der Vortrag Eindruck machte konnte er vielleicht sogar A"sistent von einem der Excellencen werden.
Ein kleines L"acheln stahl sich "uber sein Gesicht als er daran dachte, da"s Inaris vielleicht hier sein k"onnte, aber die junge Adeptin, die er so oft geh"anselt hatte, weil sie ohne Robe und mit einem Schwert bewaffnet durch die Weltgeschichte zog, h"atte sich lieber den linken Daumen abgeschnitten, als sich in eine Theorievorlesung zu setzen.

Der Saal f"ullte sich zusehends. Wie es schien waren vielen Mitglieder der ehrwuerdigen Hallen den Aushaengen gefolgt. Freilich war eine Disputatio eines angehenden Magus Minor nichts ungew"ohnliches und doch eilte Phoenixflug ein Ruf voraus... ein Ruf aus alten Tagen.

Langsam f"ullten sich auch die ersten Reihen. Ganz vorne nahm Senatorin Lauschenberg Platz. Ihres Zeichens Praetor Primus f"ur die Magia Galadae und damit von besonderem Intere"se f"ur Phoenixflug. Sie verf"ugte "uber die Administratio bez"uglich der Vorlesungen der Magia Galadae und wie bekannt war, war ihr Einflu"s im Hohen Rat - dank des Krieges der im Reiche herrschte - ein besonderer. Alle hier wu"sten, da"s Sie auf der Suche nach einem neuen A"sistentus gewesen war. Wenig gl"ucklich war ihre Hand in diesen Angelegenheiten bislang gewesen. Magus Minor Bremerius... Sein Name, als derjenige einer der ersten Verr"ater am Reiche, geisterte noch immer durch die Halle, auch wenn niemand wagte ihn offen auszusprechen. Aber dies war nun schon Jahre her.

Nun schritt Cornelius Luchtenfels durch die Bankreihen. Hinter dem Senator Aeneus und zweitem Stellvertreter seiner Magnifiziencz neben Senatorin Lauenson zog sich ein ganzer Schweif Magier, die sich in den B"anken hinter der ersten Reihe niedersetzten. Alles hervorragende Wi"senschaftler auf ihrem Gebiete.

Gerade schlo"s sich die Pforte als der Page sie noch einmal f"ur zwei weitere G"aste "offnete. Ihre Excellenz Lauenson blickte "uber die Reihen und zu dem Rede bereiten Phoenixflug hin"uber. Sie wartete kurz und beendete dann ihr Gespr"ach mit einer weiteren Gestalt. Phoenixflug vermochte nicht zu erkennen, wer es war, als die Gestalt schnell Platz nahm in einer der hinteren Reihen.

Ihre Excellenz schritt langsam die Reihen entlang, sich der Aufmerksamkeit bewu"st, die ihre die schweigende Ma"se zuteil werden lie"s. Dann nahm sie neben Luchtenfels Platz, den sie kurz freundlich begr"u"ste.

Welch eine Ehre. Luchtenfels und Lauenson, dazu noch Senatorin Lauschenberg, ein formidableres Publikum h"atte es kaum geben k"onnen und auch die Liste an anderen bedeutenden Magiern war "uberw"altigend. Einige wahren auch schon bei seiner Verhandlung vor Jahren dabei gewesen, als er verurteilt worden war und der Markgraf Milde hatte walten la"sen. Macht und Potential, aber keine Kontrolle, das hatten sie bestimmt damals ver"achtlich gedacht. Vollkommen zurecht. Doch die Zeiten hatten sich ge"andert. Keine Unsicherheit mehr, kein Stottern, kein Blick, der nur auf die Schuhspitzen gerichtet war. Nach all dem, was er erlebt hatte war nur noch eines in seinem Blick "ubrig geblieben, als er sich nun mit klarer Stimme an sein Publikum wandte, pure, kalte Effizienz ...

Dann wandte sich Luchtenfels dem Redner zu und nickte ihm freundlich zu, so als ob nun alles bereitet sei zu beginnen. Die T"ure am Ende des Saales wurde leise zugezogen und das Klacken der Pforte bedeutete den Zuh"orern und Phoenixflug, das es nun an der Zeit war... \\

``Sehr geehrte Excellencien, verehrte Senatoren, gesch"atzte Magii und Magae, werte Adepten und Intere"sierte. Vielen Dank f"ur Ihr zahlreiches Erscheinen zu meinem Vortrag "uber Deterministisches Schicksal und Chaos. In meiner Rede werde ich durch ein wichtiges Teilgebiet tiefe Einblicke in meine aktuelle Forschung geben und Bez"uge zu aktuellen Themen herstellen. In der anschlie"senden Disku"sion wird genug Zeit sein auf alle Fragen Bezug zu nehmen, so da"s ich sie bitte nur Fragen von e"sentieller Wichtigkeit w"ahrend des Vortrags zu "au"sern.''\\

Florian lie"s eine kleine Pause und kam hinter dem Pult hervor, so da"s er nun bar jeden Schutzes direkt vor dem Publikum und ihren bohrenden Blicken stand.

``Zu erkennen was die Welt im innersten zusammenh"alt. Das ist das h"ochste und wichtigste, das einen jeden von uns antreibt und uns zu Magiern macht. Maewon H"uter des Schicksals und Beherrscher der Magie lehrt uns auf seinen Pfaden zu wandeln und uns seine Kr"afte zu Nutze zu machen, die Magie und die Zeit.
Was unsere eigene Zukunft anbelangt sind wir Menschen oft zweierlei Meinung und das oft zur selben Zeit. Zum einen w"unschen wir uns unser Schicksal selbst in die Hand nehmen zu k"onnen und unser Gl"uckes eigener Schmied zu sein zum anderen sehnen wir uns nach Stabilit"at und der Sicherheit eines vorbestimmten Lebens, vorallem in so schweren Zeiten, wie diesen.
Wenn wir die fundamentalen Kr"afte der Zeit nutzen wollen, so m"u"sen wir uns zuerst fragen in wie weit das Schicksal, also Vergangenheit und Zukunft, vorherbestimmt sind, oder be"ser wie weit wir Menschen sie vorau"sehen k"onnen. Dies soll Thema dieses Vortrags sein.
Mein Ziel ist dabei nicht die Praeconition, sondern die transtemporale Dislocation mit ihren mannigfaltigen M"oglichkeiten.
In Raulfels bin ich selbst Zeuge geworden, wie eine gesamte Burg, vom Feind auf diese Weise geschliffen wurde. Mit Hilfe von Zeitmagie alterte sie um Jahre und zerfiel binnen einer Nacht zu nichts als Tr"ummern. Unserem Schutz soll diese Forschung dienen.

Wie in den Wi"senschaften "ublich fangen wir mit dem einfachsten System an und wenden dann unsere Erkenntni"se auf kompliziertere Ereigni"se an. Wie jeder Richtsch"utze wei"s beschreibt das Gescho"s eines Katapultes oder einer Kanone eine paraboloide Flugbahn und man kennt den Einschlagsort, bevor man da"s Gescho"s "uberhaupt auf seinen Weg gebracht hat. Also ist besagter Richtsch"utze in der Lage nur durch die Kraft des Verstandes und der Logik, sogar ohne Zuhilfenahme der Ars Magica die Zukunft vorher zu sagen. Nat"urlich wird man nun entgegnen, da"s dies mitnichten Pr"acognition ist sondern lediglich eine mehr oder weniger genaue Sch"atzung. Dem halte ich allerdings die Verschwommenheit und unklaren Au"sagen allgemein akkreditierter Wahrsagen entgegen.
Nat"urlich kann ein Richtsch"utze nicht exakt den Aufschlagsort seines Gescho"ses bestimmen, aber es ist trotzdem m"oglich in einem bestimmten Rahmen die Zukunft, also in unserem Beispiel den Zielort, vorherzusagen.
Betrachten wir zum Beispiel ein kurzes St"uck der Flugbahn des Gescho"ses. Nun wird mir sicher jeder zustimmen, da"s ich die Bewegung des Gescho"ses f"ur eine kleine Zeitspanne, sagen wir eine viertel Pendelschwingung, vorau"sagen kann. Nun ja, dies stimmt nat"urlich auch wieder nicht exakt, denn einem renomierten Magier fallen sicherlich mehrere M"oglichkeiten ein, besagtes Gescho"s in seinem Flug zu stoppen, sei es durch eine magische Wand oder eine Aus"ubung der Galadae.
Aber wenn ich nun diese Zeitspanne weiter verringere, werden die M"oglichkeiten den Flug zu unterbrechen oder abzu"andern immer geringer bzw. die Vorhersage wo sich das Gescho"s aufhalten wird immer pr"aziser. Ebenfalls kann ich durch die Beobachtung der Umgebung M"oglichkeiten den Flug des Gescho"ses zu unterbrechen ebenfalls erkennen und in meine Vorhersage mit einbeziehen. Will hei"sen in letzter Konsequenz wird f"ur ein infinitesimales Zeitinterwall die Vorhersage absolut korrekt.

Dies ist eine schockierende und mit nichten triviale Erkenntni"s, denn anders ausgedr"uckt bedeutet es nicht mehr und nicht weniger, als das unter bestimmten, wenn auch sehr restriktiven, Vorau"setzungen das Schicksal, also die Zukunft dieses Steines, vorherbestimmt ist.
Vorallem m"u"sen wir bei dieser Erkenntni"s nicht exemplarisch bei dem Gescho"s bleiben sondern k"onnen es auf jegliches Objekt verallgemeinern und wenn ich in diesem Kontext von Objekt spreche, dann meine ich nicht nicht nur jegliches unbelebte Objekt, sondern ebenfalls jedes Lebewesen bis hin zu uns Menschen.

Nun, da wir festegestellt haben, da"s die Zukunft, also das Schicksal, vorherbestimmt und festgeschrieben ist k"onnen wir uns die Frage stellen, wie wir Menschen dies nutzen k"onnen.
Dabei stehen wir vor dem gro"sen Problem, welches auch unser exemplarischer Richtsch"utze hat, n"amlich der Ungenauigkeit unserer Einsch"atzung. Der tats"achliche Einschlagsort des Gescho"ses weicht von dem theoretisch errechneten ab, weil "au"sere Einfl"u"se ihn von seinem Weg abbringen. Regen und Wind m"ogen allt"agliche sein, w"ahrend "ubernat"urliche Intervention, wie zum Beispiel das hypothetische Eingreifen eines Magiers, eher au"sergew"ohnlich sind.
Ein guter Richtsch"utze kann nun seine Berechnung aufgrund der "au"seren Einfl"u"se modifizieren, w"ahrend diese Praxis bei Wind und Regen noch eher gel"aufig und praktikabel ist wird sie sp"atestens bei "ubernat"urlicher Intervention praktisch unm"oglich sein. W"urde er aber alle "au"seren Faktoren kennen, so w"ahre es ihm mit absoluter Sicherheit m"oglich den pr"azisen Einschlagsort zu bestimmen. Leider sind selbst bei etwas so trivialem, wie einem Kanoneschu"s die "au"seren Einfl"u"se von unendlicher Mannigfaltigkeit, so da"s unser endlicher Geist sie niemals begreifen kann.
In einem unserer Probleme, also vorherzusagen wie eine Schlacht endet, oder auch nur den n"achsten Schachzug unseres Feindes vorauszusehen w"ahren die "au"seren Einfl"u"se noch um ein gewaltiges gr"o"ser.

Daher kommt es, da"s wir, wenn wir Pr"acognition betreiben, uns stets auf einem Gebiet bewegen, welches aufgrund seiner unendlichen M"oglichkeiten nicht vom menschlichen Geist in seiner F"ulle erfa"st werden kann. Lediglich die ewigen G"otter haben Einblick in diese Gefilde und k"onnen das Schicksal in seiner G"anze begreifen.
So wir Menschen mit unserem endlichen Geist auf diesen Pfaden wandeln bleibt uns nichts anderes "ubrig, als ungenau zu werden und die Unendlichkeit der M"oglichkeiten auf ein endliches, f"ur uns erfa"sbares Ma"s zu reduzieren.
So mu"ste selbst die Thesys, unsere st"arkste Waffe im Kampf gegen den Feind, von Ytalas in einer unklaren Sprache verfa"st werden, die Platz f"ur Interpretationen l"a"st, obwohl er von Maewon pers"onlich instruiert wurde. 

Aber obwohl das menschliche Unverm"ogen pr"azise Vorhersagen "uber die Zukunft zu treffen durchaus Teil meiner Forschung ist will ich mich im Rahmen dieses Vortrages nicht weiter mit diesem Thema auseinander setzen sondern zu konkreten L"osungen in Bezug auf unser Problem kommen und Methoden anbieten mit denen wir die Zeitmagie nutzen k"onnen.

Wir identifizieren zwei verschiedene M"oglichkeiten Einflu"s zu nehmen auf unser Systhem. Welche, deren Auswirkungen exponentiell ansteigen und welche die in linearem Ma"sstab unser Problem beeinflu"sen. Erstere sind die Regel und im Allgemeinen ist ihre Anzahl unendlich. Ihre Auswirkungen, oder tats"achlich die Wahrscheinlichkeit ihres Auftretens, was in unserem Kontext die gleiche Wertigkeit hat, sind meist verschwindend gering. In unserem Beispiel w"are der Fl"ugelschlag eines Schmetterlings eine solche Einflu"snahme. Der Wind, den der Fl"ugelschlag des Schmetterlings bewirkt, beeinflu"st das Gescho"s auf eine solch geringe Weise, da"s wir es nicht einmal wahrnehmen k"onnen. Allerdings gibt es eine solch gro"se Anzahl dieser Faktoren, da"s sie eben nicht vernachl"a"sigt werden k"onnen.
Ferner k"onnte der Fl"ugelschlag eines Schmetterling vor einigen Tagen einen Windstrom in seinem Winkel um nur den Millionsten Teil des Bogenma"ses abgelenkt haben, so da"s der Wind zu just der Zeit unseres Schu"ses unseren Stein beeinflu"st. Die Wahrscheinlichkeit dieser Einflu"snahme ist ebenfalls sehr gering, aber ihre Zahl ist sehr gro"s und falls sie eintrifft gilt dies ebenfalls f"ur ihre Auswirkungen.
Der Grad der Einflu"snahme steigt mit der Zeit exponentiell an und auch wenn wir zu einem infinitesimalen Zeitpunkt vernachl"a"sigbar kleine Auswirkungen haben, so werden diese mit der Zeit allerdings exponentiell gro"s und unsere Vorhersage maximal Ungenau.

Die anderen M"oglichkeiten der Einflu"snahme sind von linearer Natur und auch wenn uns ihre Auswirkungen auf den ersten Blick gewaltig erscheinen m"ogen, so wird mit der Zeit jedoch jedwede noch so gro"se lineare Steigung schw"acher werden als die Exponentiellen.
Um nun die Exponetiellen von den Linearen zu trennen bedienen wir uns der selben Methode, die wir Wi"senschaftler immer als erstes benutzen, wenn ein Sachverhalt zu gewaltig ist, als da"s wir ihn mit unserem Geist begreifen k"onnen. Wir Visualisieren. Wie bei der Analyse eines Artefaktes oder des Raumes bedienen wir uns einer transdimensionalen Analysis und visualisieren das Problem in einem Kristall oder einer Linse. Dabei ist es schon bei diesem Schritt "au"serst wichtig die Dimension der Zeit genau solcherart in die Betrachtung mit einzubeziehen, wie jegliche andere Dimension, seinen es die Elemantarebenen, oder der Astralraum, auch.
Bei einer normalen Invocation des Effektes suchen wir nach Strukturen und Mustern in Form und Farbe, die uns R"uckschl"u"se auf die zu analysierende Materie erm"oglichen.
Nun wollen wir uns aber auf die Trajektorie der multidimensionalen Abbildung selbst foku"sieren und so m"u"sen wir uns eines anderen Verfahrens bedienen.
Wir projizieren das erhaltene Bild in seiner G"anze in den Phasenraum um es be"ser analysieren zu k"onnen. Dort bedienen wir uns der Poincarr\`{e}schen Phasenraumschnitte um zweidimensionale Querschnitte der Vieldimensionalen Trajektorie zu erhalten.
Unser Geist, befreit von der Unendlichkeit kann nun auf den einzelnen Fl"achen den Verlauf erkennen und die linearen Auswirkungen als Geometrische Muster, meist Ellipsen, wahrnehmen, w"ahrend sich die Exponetiellen Auswirkunegn als einzelne chaostisch verteilte Punkte darstellen.
Da wie gesagt die Exponentiellen Abweichungen die Regel sind, ist das normale Bild ein Meer des Chaos mit bestenfalls vereinzelten Inseln der Stabilit"at.
So wir diese einzelnen Inseln der Stabilit"at identifiziert haben, k"onne wir einen linearen Zusammenhang zwischen ihnen herstellen und solcherart einen Pfad durch die Zeit bestimmen, bis zu unserem Designierten Endpunkt.
Das Verfahren hierbei ist das gleiche, wie bei der Anwendung von Occuli, den Portalen, oder den Nodice Causalis, nur da"s statt einer Abk"urzung durch die R"aumliche Dimension eine Abk"urzung durch die Zeit benutzt wird.
Der Kraftaufwand der hierbei ben"otigt wird ist linear zur Abweichung, kann also bei hohen linearen Abweichungen durchaus immens und dadurch nicht praktikabel werden.

Verglichen mit den gewaltigen causalen Kr"aften mit denen wir uns hierbei auseinander setzen ist das Verfahren geradezu simpel obwohl viele Magier es gar nicht erst werden anwenden k"onnen. Diese extreme Manipulation ist uns "uberhaupt nur m"oglich weil wir uns bei dem ganzen Weg nicht explizit f"ur die Linearen Auswirkungen intere"sieren und lediglich ihr Vorhandensein ausn"utzen.''\\

\textit{Ph"onixflug wirft einen Blick auf den Tisch neben dem Rednerpult, auf dem eine Schale mit Wa"ser steht, "uber deren Wa"seroberfl"ache eine kleine Kugel in der Luft schwebt.}\\

``Nun bin ich ihnen nat"urlich eine praktische Demonstration schuldig, so wie es au"sieht sind wir tats"achlich auch bald so weit.''

\textit{Er nickt einem A"sistenten zu, der in der Ecke des Vorlesung"sahles steht und nun damit beginnt einen Vorhang, der zuvor die rechte Seite der B"uhne verdeckt hatte zur"uckzuziehen. Auf den schwarzen Marmorboden ist dort mit Kreide, Sand und verschiedenen Kri"stallpulvern ein "au"serst komplexer magischer Zirkel gezogen worden in deren Mitte ein kleiner Topf steht, wie er doch "ofters f"ur Pflanzen benutzt wird.
Der Magier stellt sich in Position hinter den Kreis mit dem Blick zum Publikum und beobachtet die keine Kugel "uber der Schale, die sich immer mehr der Wa"seroberfl"ache n"ahert.}

``Eigentlich wollte ich noch eine Erkl"arung vor der Demonstration geben, aber leider eilt uns die Zeit davon'', bekommt er gerade noch mit einem am"usierten L"acheln auf den Lippen hervorgebracht, bis die Kugel die Wa"seroberfl"ache ber"uhrt und der Ritualkreis zu leuchten anf"angt.'

\textit{Direkt in Erwiderung f"angt Florian an die Arme auszubreiten und mit tiefer beschw"orender Stimme zu intonieren:}\\

\textbf{vas daemoni creo sanctum kal aragh eterna mortis perdo discrim vas lor intellex movo magia ignitia creatura mentem an muto temporis}\\

\textit{Dabei h"alt er ein Pendel "uber den "au"sersten Kreis, da"s am Anfang mit jeder Pendelschwingung ein rotes Leuchten an seiner Spitze hinter sich herzieht. Doch mit jeder Pendelschwingung scheint das Leuchten schneller zu werden und dem Pendel vorauszueilen, bis es nur noch eine durchgehende Leuchtende Schnur bildet, die die Pendelspitze immer entlang eilt.}\\

\textit{Mehrere verschieden stark bl"aulich gef"arbte Kugelhalbschalen bilden sich um das Zentrum und in der innersten Schale knistert ein feines Gespinnst aus hellblauen filligranen F"aden. Vor den Augen des Publikums w"achst aus der Erde im inneren der Vase eine kleine Pflanze}\\

\textbf{disjungite aenas}\\

\textit{, bildet einen ca. zwei Finger langen St"angel }\\

\textbf{disjungite temporis}\\

\textit{und vergeht und verschrumpelt dann zusehens.}\\

\textbf{Vas magia opprimo magia tym intellex creo muto mani tym}\\

\textit{Bis sie nur noch so au"sieht wie ein k"ummerliches Pfl"anzchen, das eine Woche lang nicht gego"sen wurde.}\\

\textit{Das Licht am Ende des Pendels wird langsamer und folgt wieder der Pendelspitze, bis es schlie"slich ganz aufh"ort.}

\textit{W"ahrend die Kreise glimmend verl"oschen und nur noch ein leichtes Knacken und ein Geruch nach Ozon in der Luft bleibt, schleppt sich Ph"onixflug kreidebleich hinter das Rednerpult, st"utzt sich schwer ab und bekommt einen Hustenanfall, der allerdings im allgemeinen Beifall untergeht, der jetzt im Raum unter den Adepten ausbricht und auch einige der H"oher gestellten mit einstimmen l"a"st. }

\textit{Als die Menge sich des Zustandes von Phoenixflug bewuszt wird, h"ort der Beifall fast schlagartig auf. Ein Raunen und Murmeln geht durch die Menge. Luchtenfels Blick schweift zu Phoenixflug A"sitenten, der etwas rat- und hilflos daneben steht. Ein weiterer Blick l"a"st zwei Pagen herbeieilen, die Phoenixflug helfen wollen...was dieser jedoch ablehnt. Der Husten und Schmerz zieht vorr"uber und nun vermag er den Moment seines Triumphes vollends auszukosten, als die Menge, erleichtert, wieder in Beifall einstimmt. Luchtenfels nickt ihm zustimmend zu, anzeigend, da"s es an der Zeit sei die Demonstratio und Explicio fortzufuehren und zu Ende zu bringen, um sich der Disputatio der Anwesenden zu stellen... }

\textit{Florian Phelleas Ph"onixflug erwiedert Luchtenfels ermunterndes und aufforderndes Nicken seinerseits mit der gebotenen Demut und wendet sich erneut dem Publikum zu. Ein Stofftaschentuch mit dem er sich den Mund abgewischt hatte verschwindet hinter dem Pult und etwas leiser beginnt er mit der Erkl"arung.}

``Die Demonstratio, die sie soeben sahen ist die Nachbildung des temporalen Effektes mit dem Burg Raulfels vor gut einem Jahr geschliffen wurde. La"sen sie sich von der Einfachheit der Vorf"uhrung nicht t"auschen. Es hat einen doch sehr signifikanten Aufwand bedeutet die Wirkung entsprechend vorzubereiten und nur ein kleiner Teil war die Ausl"osung, derer sie gerade angesichtig wurden. Trotzdem konnte ich lediglich eine Bohnenpflanze um das Drittel eines Mondes altern la"sen und habe daf"ur fast meine ganzen pers"onlichen Re"sourcen aufbringen m"u"sen, obwohl ich Silberkraut zur Unterst"utzung benutzte.
Um mit dem von mir entwickelten Verfahren einen milit"arischen Effekt zu erzielen w"urde die Macht aller Magier dieser ehrw"urdigen Hallen nicht ausreichen.
Daher wird ein Gro"steil meiner zuk"unftigen Forschung darin bestehen die Methodik der Schatten und ihrer Lakaien zu erforschen um herauszufinden auf welchem Wege sie ebenjene temporalen Effekte sehr viel einfacher invocieren k"onnen.
Momentan halte ich die n"ahere Erforschung der E"senzen des Schattenprinzen im Sinne des Leibarztes des Protautian f"ur sinnvoll. Er identifizierte drei Elemente und a"soziierte sie sowohl mit Attributen, herk"ommlichen Elementen, als auch Verk"orperungen, welche Ratmath An Burgia, Hohepriester Acrulons einst prophezeite.

Explizit die Di"sonantia halte ich bei meinem mometanen Kenntni"sstand dazu in der Lage die Kr"afte des Chaos zu ordnen, so wie es f"ur eine solcherart m"achtige Anwendung der Magia Temporalis von N"oten ist.
Ich beabsichtige nach einigen theoretischen Auffrischungen in Elementartheorie nach Mythodea zu reisen um dort in den Bibliotheken der Quai die verfehmten Elemente der zweiten Sch"opfung zu studieren, deren elemetare Konsistentia Parallelen zu den Elementen der E"senz des Schattenprinzen aufweist.
Au"serdem trachte ich danach den Begriff des Narrativums enger zu fa"sen und mit der eben demonstrierten Methode in Einklang zu bringen.
Ich Danke ihnen vielmals f"ur ihre Geduld und hoffe auf eine gute Zusammenarbeit hier in der Clavis Mundi.
F"ur Fragen sowohl zu diesem Themenschwerpunkt, als auch zu meiner generellen Arbeit stehe ich ihnen nun zur Verf"ugung. ''

\textit{Nachdem Ph"onixflug geendet hatte, erklang f"ur wenige Augenblick zustimmendes Klopfen. Dann erhob sich Senator Luchtenfels und schritt zum Pult, w"ahrend ein zweites Stehpult f"ur Ph"onixflug herangetragen und im Hintergrund die Instrumentes seiner Experimentatio beiseite geschafft wurden.}

\textit{Luchtenfels erhob seine Stimme. Wie es sich fuer den Praesidius der Commisio Disputatiae gerhoerte, erklang seine Stimme gleichmuetig...}

\grqq Zunaechst moechte ich Euch, verehrter Candidatus Magiae Minor fuer diese eindrucksvolle Praesentatio dancken. Selten genug geschieht es, dasz die theoretischen Ausfuehrungen der Candidates mit einer solch hervorragenden Praesentatio verknuepft werden. Bevor die Commisio sich ob der Bewertung Eurer Disputatio zurueckziehen wird, soll allen Zuhoerern - zuvoerderst selbstverstaendlich den Mitgliedern der Commisio - jedoch Gelegenheyt gegeben werden, Fragen an Euch zu richten und - hoffentlich - eloquente Antworten Eurerseyts - woran ich im uebrigen keynen Zweyfel hege - zu stellen. Der traditio dieser Hallen entsprechend erteyle ich daher der Beysitzerin der Commisio, Senatorin Erszbet Lauschenberg das Worth... \grqq

\textit{Langsam erhob sich Lauschenberg von ihrem Stuhl. Die kleyne Frau trug die dunckelrothen Roben der Commisio Disputatio, an ihrer linken Seyte trug sie einen Bund Schlue"sel, Zeychen der Aemter und Wuerden, die sie innehatte. Auf der rechten Seyte aber baumelte ein weyterer, kleyner gueldener Schlue"sel, eben jener welcher dem Candidatus im Falle des Bestehens - dann Magus Minor - ueberreycht werden wuerde.
Neben ihr hatte sich ein Page erhoben, welcher nun einen gewaltigen Folianten hielt. Die \grqq Inscriptiones Magistres \grqq , in welchem die Einzelheyten jeder Pruefung, die seyt der Erbauung dieser Hallen stattgefunden hatten, festgehalten wurden.}

\textit{Die Senatorin schritt nach vorne zum Pult, waehrend Luchtenfels wieder Platz nahm... }

\textit{Seinen schwachen K"orper auf den Stab st"utzend ging Florian hinter das f"ur ihn bereitgestellte Pult. Mit einer angedeuteten Verbeugung zollte er den hohen Senatoren Respekt, als er sich f"ur die Fragen wappnete.
Jetzt war er doch aufgeregt. Tief in seinem Inneren begehrten die wenigen Lebenskr"afte, die sein schw"achlicher K"orper noch zur Verf"ugung hatte auf und lie"sen sein kraftloses Herz flattern wie ein kleines F"ahnlein im Wind.}

\textit{Selten wurden praktische Pr"asentationen in theoretischen Vortr"agen gemacht, das hatte Luchtenfels gesagt. Meinte er damit, da"s es un"ublich war eine praktische Vorf"uhrung einzubringen? H"atte er es lieber nicht machen sollen? Er hatte sich vorher kundig gemacht und ihm wurde gesagt, da"s es m"oglich w"are, aber er hatte nicht gefragt wie "ublich es ist. Verdammt. War er etwa nach seinen ganzen Reisen und der langen Zeit in Arkenwald zum Praktiker geworden? Seine Freunde h"atten dar"uber gelacht, da"s ihm so ein Gedanken kommen k"onnte.
Egal, es war vorbei nun stand die Disputation an. Darauf hatte er sich vorbereitet. Er wu"ste, da"s er an vielen Punkten angreifbar war. Aber er hatte lieber diese Thema gew"ahlt, das mehr wi"senschaftlichen Ruhm einbrachte, als eines, welches sicher w"are.
Also noch mal kurz durchgehen. Die Sicherheit, die G"otter, Paradoxen, der Einflu"s der E"senzen, was k"onnte man ihm noch vorwerfen, ja genau die Portalproblematik, die Occuli, wieder E"senzen. Vielleicht dachten sie auch vollkommen an etwas anderes, mag sein, er war vorbereitet ... }

\textit{Lauschenberg Stimme durchbrach die Stille.}\\

``Folgendes will ich nun anmerken:
Eure Argumentation ist von bestechender Einfachheit und Anschaulichkeit, richtet doch ein jeder vern"unftige Mensch, seine Taten an der Erwartung kausaler Zusammenh"ange aus. 
Die Regelhaftigkeit von Ereigni"sen anzuzweifeln hie"se das eigene Denken und Handeln als Ergebnis unstrukturierter, zuf"alliger, ja sinnloser Proze"se darzustellen. Eine solche Behauptung w"urde durch sich selbst zunichte gemacht, da Sie Ihre eigene Grundlage und Rechtfertigung verneinen w"urde.

Dies schl"o"se einen freien Willen in so umfa"sender Form aus, da"s eure eigene Erkenntnis ebenfalls Ergebnis einer kausalen Kette sein m"u"se. 
Zum ersten Mal in dieser Disputatio m"u"sen an dieser Stelle systemtheoretische "uberlegungen zum Tragen kommen. 
Kann ein Element des Systems eine solch umfa"sende Kenntnis des selbigen erlagen, wenn es doch durch Kausalzusammenh"ange von deren Wechselwirkungen abh"angig ist?

Das zweite, sehr "ahnliche, systemtheoretische Problem ist in der Analyse des zu manipulierenden Ereigni"ses zu suchen. Kann ein offenes System, damit ist jedwedes Ereignis gemeint, das nicht durch den gesamten Kosmos abgebildet wird, hinreichend erfa"st werden? 

Ein weiterer h"ochst intere"santer Punkt, von scheinbarer Trivialit"at, sind die nicht linearen Einfl"u"se. Zum einen stellt sich die Frage, ob lineare Einfl"u"se nicht auch mittelbar exponentielle Einfl"u"se ausl"osen, die gemeinsam mit fr"uheren Ereigni"sen, die ebenfalls als exponentielle Einfl"u"se zu bezeichnen sind, den Status Quo in unberechenbarer Form determinieren. Dies w"urde eine komplexe Typisierung erfordern, die Ihr uns sicherlich noch zu erl"autern bereit seid.

Quod futurum e"se demonstrandum ''

\textit{Darauf hatte Florian gehofft. Wenn er den Vortrag auf diese Weise hielt wurde, bei einem aufmerksamen Publikum, nat"urlich diese Frage gestellt. Es sprach sehr f"ur die herausragenden analythischen F"ahigkeiten von Senatorin Lauschenberg, da"s sie dieses Thema ansprach und auch sehr daf"ur, da"s sie Florian wohl einige Sympathie entgegen brachte in zuerst mit diesem Thema zu konfrontieren, von dem sie ausgehen konnte, da"s er eine gute Antwort parat h"atte.}

\grqq Vielen Dank Senatorin Lauschenberg.

F"ur die Beantwortung der systhemtheoretischen Fragen mu"s ich einen Bereich meiner aktuellen Forschung zitieren, der, wenn auch stark verwandt, nicht Teil dieses Vortrags war. Es handelt sich um das theoretische Unverm"ogen f"ur unseren endlichen Geist exakte Prognosen "uber Zukunft oder Vergangenheit zu machen. Begr"undet liegt dies, wie sie in der Fragestellung so treffend feststellten, sowohl in der Wechselwirkung, der wir als Teil des zu untersuchenden Systems unterworfen sind, als auch der Unendlichkeit, des in sich geschlo"senen Kosmos beziehungsweise der disjunkten Untermannigfaltigkeiten.
Die Herleitung w"urde sicherlich den Rahmen dieser Disku"sion sprengen, trotzdem sei in aller Klarheit angemerkt, da"s ein Element eines Offenen Systhems, also wir als Beobachter, nie vollst"andige Erkenntni"s erlangen k"onnen. Dies liegt an der Unm"oglichkeit der exakten Prognose, die durch eine Unsch"arferelation zwischen Narrativum, also der \grqq M"achtigkeit \grqq des Ereigni"ses und dem Fehler der Erkenntni"s beschrieben werden kann.
Sehr vereinfacht ausgedr"uckt hei"st dies f"ur unsere Problemstellung, da"s wir mit sehr unvollkommenen Informationen arbeiten m"u"sen. Trotzdem k"onnen wir, wie es ein Kollegus einst ausdr"uckte, auch mit schmutzigem Wa"ser Geschirr waschen. Deshalb erlaubt es meine Methode sich linearer Ungenauigkeiten, die wir hinreichend beschreiben k"onnen, zu bedienen und uns solcherart "uber die Inseln der Stabilit"at durch das Meer des Chaos leiten zu la"sen. Wir ben"otigen mit diesem Verfahren keine umfa"sende Kenntni"s des Systhems um daraus Gewinn ziehen zu k"onnen. Daher ist es dann nat"urlich f"ur die Pr"acognition unbrauchbar, aber f"ur die temporale Dislocation umso wertvoller.

Die zweite Frage bezog sich auf lineare Einfl"u"se und da"s sie exponentielle Einfl"u"se ausl"osen k"onnten. Ich bitte vielmals um Entschuldigung, da"s ich in meinem Vortrag nicht n"aher auf die exakte Analysis an diesem Punkt eingegangen bin und danke Ihnen f"ur die Aufforderung das Verfahren zu konkretisieren. Nun nat"urlich l"osen lineare Einfl"u"se exponentielle aus und anders herum. Ein Ereignis beeinflu"st das n"achste und zusammen bedingen sie wiederum ein weiteres und so weiter, und so weiter ... . Somit besteht schon die einfachste Trajektorie, die wir uns vorstellen k"onnen aus, sagen wir 10 Ereigni"sen, die jeweils wieder 10 Ereigni"se bedingen, die wiederum 10 weitere Ereigni"se bedingen. Jede dieser 1000 verschiedenen Wege kann nun auf seinen jeweils 2 Teilst"ucken linearer, oder exponentieller Natur sein. Mittels der Fourieranalyse zerlegen wir die Trajektorie in ebenjehne einzelnen Wege und brechen so die sehr komplexe Funktion auf die zwei M"oglichkeiten exponentiell oder linear herunter und 
w"ahlen unseren Weg.``

\textit{Senatorin Lauschenberg gab sich sichtlich M"uhe ein ebenso zufriedenes wie auch anerkennendes L"acheln zu unterdr"ucken. Sie ben"otigte einen kurzen aber merklichen Augenblick um die gew"unschte analytische K"alte in Ihre Mimik zur"uckkehren zu la"sen.}

\grqq Der Euch eigene Pragmatismus ist gleicher Weise erfrischend wie auch selten unter den Gelehrten und so sehe ich alle meine Fragen durch eine Antwort von "uberraschender Simplizit"at beantwortet.
Und so richte ich nur noch die beil"aufige Frage an Euch, ob Ihr den Begriff unendlich in einer Beschreibung eines unbegrenzten Volumens oder einer Ermangelung an Grenzen im Sinne einer Grenzenlosigkeit gebrauchtet.''

\textit{Als wolle sie demonstrieren, da"s die Frage so selbstverst"andlich sein, da"s ihre Antwort nicht lange auf sich warten la"sen k"onne, verharrte Sie in leicht angespannter, nach vorne gebeugter Haltung. Ihr Tonfall war von offensichtlicher Beil"aufigkeit gewesen, doch Ihre K"orpersprache formulierte eine klare Aufforderung zur Eile und vermittelte dadurch eine gewi"se Dringlichkeit. }

\grqq Ah ja, der Begriff der Unendlichkeit ist an vielen Stellen von e"sentieller Bedeutung. Oft k"onnen wir nicht mit einem undendlichen Volumen arbeiten, sondern m"u"sen uns mit einer geringeren Unendlichkeit, wie der Grenzenlosigkeit begn"ugen.
Wie beim trivialen Beispiel der geometrischen Reihe. Wir addieren eins, einhalb, dann ein viertel und ein achtel und so weiter. Jedes addierte Folgeglied ist die H"alfte des vorherigen. Die komplette Summe n"ahert sich mit jedem addierten Folgeglied immer weiter der Zwei an, ohne sie jedoch jemals zu erreichen. Wir haben also undendlich viele Folgeglieder und undendlich viele Summen im Bereich zwischen Eins und Zwei. Es existiert aber keine Grenze, sondern lediglich ein Supremum, eine kleinste obere Schranke.
Auch wenn wir also nicht alle Folgeglieder addieren k"onnen, weil es unendlich viele sind, k"onnen wir trotzdemeine Au"sage "uber die Summe der ganzen Reihe machen, wie sie sich im Unendlichen verh"alt, sie konvergiert gegen Zwei. \grqq

Und so folgten weitere Anmerkungen und Fragen zu seinem Vortrag und es zog sich doch noch f"ur lange Zeit hin. Die Hohen der Akademie schauten mit Wohlwollen auf den angehenden Magus Minor herab und ihre Fragen wahren zwar wohlgew"ahlt und herausfordernd, aber nicht destruktiv und vernichtend, so wie es doch durchaus schon in einigen dieser Disputationen gewesen war. Sei es, da"s sie die Einstellung der Hohen sp"uhrten, oder sei es da"s sie keinen Angriffspunkt fanden, so wahren auch die Fragen der regul"aren Magi und Magae harmlos und meist lediglich intere"sierter Natur.

Als die Fragen geschloszen wurden danke Lauschenberg dem Kandidatus Phoenixflug und die Kommi"sion zog sich zur Beratung zur"uck, w"ahrend sich die Versammelten aufl"osten und nur Florian allein zur"uck blieb.

Die folgende Zeit, bis zur Nachricht der Kommi"sion m"u"sen die h"artesten f"ur ihn an der Clavis Mundi bisher gewesen sein, aber alle M"uhen hatten sich gelohnt, als wir einen Diener mit einer Marmortafel zum Eingang eilen sahen, auf dem simpel stand „Florian Phelleas Ph"onixflug Magus Minor Academia Clavis Mundi Grenzbrueckensis“

Die offzielle Verleihung des Schl"u"sels war um so prunkvoller, aber soll nicht an dieser Stelle berichtet werden.“\\

\vspace{10mm}
Answin aus Tibur, Apreius 258 nach Jeldrik um die Engonische Zeitrechnung zu benutzen



\newpage

\section{Lix Dissertation}



Untersuchung von Entstehung und Natur der Globulen und Subglobulen bekannter Sphären anhand einer elementaren Subglobule


Von der Academia Cantus Harmoniae zu Tharemis zur Erlangung des akademischen Grades einer Vicaria Vita genehmigte Thesis


Vorgelegt von Adepta Exempta Vitam Salix Phönixflug


\subsection{Kapitel I: Bisher bekannte Theorie und Objekt der Untersuchung}

Der Begriff „Globule“ beschreibt ein geschlossenes System, das im Gegensatz zu einer Sphäre keine perfekte, in sich 
abgeschlossene Kugel darstellt. Ein solches System kann nicht spontan entstehen und so ein weiteres Kriterium einer 
Sphäre erfüllen: es ist allerdings in der Lage, energetisch und physisch unabhängig von seiner Muttersphäre zu existieren. 
Die Entstehung oder Abspaltung eines Systems von seiner Sphäre ist immer durch ein Lebewesen großer Macht bedingt, welches 
das Bedürfnis, sich ein sicheres Heim zu schaffen perfektioniert, indem es seine eigene Subsphäre, also eine Globule 
schafft. Bildlich gesehen spinnt es sich einen Kokon, dessen innere Bedingungen nur von ihm kontrolliert werden können. 
Stirbt die Entität, bricht in manchen Fällen auch die Globule in sich zusammen. Hat sich allerdings ein stabiler 
Energiekreislauf innerhalb der Globule etabliert, ist die Anwesenheit des Erschaffers nicht mehr nötig um das System vor 
dem Kollaps zu bewahren. 

Das Innere einer Globule kann nur aus Elementen seiner Muttersphäre zusammengesetzt sein. Der Beweis wurde in einem 
Feldversuch erbracht: in keiner bekannten Globule wurden jemals außersphärische Elemente angetroffen und jegliche 
außersphärischen Einflüsse werden mit dem exakt gleichen Impuls wie in der Muttersphäre abgestoßen. Außerdem ist der 
Kreislauf der belebten Materie in Globule und Sphäre ähnlich bis identisch. Es ist sehr wahrscheinlich, dass eine Globule 
nur das beinhalten kann, was innerhalb des Vorstellungshorizontes seines Erschaffers liegt.

Der Austausch von Masse und Energie zwischen Sphäre und Globule ist prinzipiell möglich, allerdings muss hierfür die Hülle 
der beiden Systeme gewaltsam durchbrochen werden. Die Eigenschaften dieser Hüllen sind bisher unbekannt und deshalb ist es 
bislang nur möglich, mit brachialer Gewalt und großem Kraftaufwand ein nicht weiter kontrolliertes Loch zu reißen, um 
belebte und unbelebte Gegenstände oder Energie zwischen den Systemen auszutauschen.

Man unterscheidet solche Subsysteme in vollwertige Globulen und Subglobulen. Während eine vollwertige Globule eine intakte 
Hülle hat und, wie oben beschrieben, in völliger Unabhängigkeit von seinem Ursprungssystem existieren kann, ist bei einer 
Subglobule eine partielle Abhängigkeit von der Muttersphäre zu beobachten. Meist ist es nur ein einzelner Faktor, wie eine 
künstliche Verbindung zu einem Ort, einer Person oder einem Gegenstand in der Muttersphäre oder eine „dünne Stelle“ in der 
Hülle der Globule. Dieser Faktor aber hat immer die Eigenschaft, Kraft zwischen Muttersphäre und Globule auszutauschen: 
meist in Richtung der Globule, die ohne diese Energie nicht aufrechterhalten werden kann. 
In den seltenen Fällen, in denen Masse und Kraft zurück in die Muttersphäre fließen, wird diese Energie in allen 
beobachteten Fällen dazu verwendet, ein magisches Konstrukt oder ähnliches zu speisen. Es kann also davon ausgegangen 
werden, dass diese Subglobulen alleinig zu diesem Zweck von magiebegabten Humanoiden erschaffen wurde. 
Im Prinzip gibt es also zwei Formen der Subglobulen: Die unterlegene Subglobule, die auf Kraft von außen angewiesen ist und 
die überlegene Subglobule, die in der Lage ist, durch die Hülle Kraft an das unterlegene System abzugeben ohne dabei zu 
kollabieren.

Das Objekt dieser Untersuchung soll eine überlegene Subglobule sein, die von einem benannten Humus-Dschinn geschaffen 
wurde. Da Elementare sich normalerweise als undifferenzierte Einheit auf der jeweiligen Elementarebene bewegen, stellt es 
für sie ein Problem dar, wenn ein Wesen der gleichen Sphäre, das aber auf der physischen Ebene beheimatet ist ihm einen 
Namen und damit eine Existenz außerhalb der Elementarebene gibt. Ihre kollektive Existenz verbietet ihnen die Entwicklung 
einer Persönlichkeit auf der Elementarebene, während ein langfristiger Aufenthalt auf der physischen Ebene das energetische 
Gleichgewicht der materiellen Ebene in Richtung der Elementarebene verschieben würde. Eine Lösung für dieses Problem ist 
die Schaffung einer Globule, die eine Art Transitraum zwischen den beiden Ebenen darstellt. 
Bemerkenswert bei dieser elementaren Subglobule ist, dass sie je nach Jahreszeit und anderen natürlichen oder künstlichen 
Einflüssen an der Kontaktlinie entlang wandert. Bereits das Fällen einer Baumgruppe kann die Verlagerung der Kontaktstelle 
um mehrere hundert Schritt bewirken. Offensichtlich ist der Erschaffer sehr eng mit seiner Umgebung auf der physischen 
Ebene verbunden. Ob er genauso empfindlich auf Veränderungen in der elementaren Ebene reagiert, konnte leider nicht 
nachgewiesen werden. 


\subsection{Kapitel II: Wie man die Anwesenheit einer Globule verifiziert}

Ein Ort, der eine Kontaktstelle zu einer Globule darstellt, ist meist schon mit einfachen Beobachtungen durch die fünf 
natürlichen Sinne zu identifizieren. Flora und Fauna verhalten sich an diesen Orten meist entsprechend der Natur der 
Globule. Ansässige Menschen und Tiere empfinden diese Orte oft als speziell und sonderbar und viele, nicht mit den 
natürlichen Sinnen erklärbare Dinge ereignen sich dort. Häufig sind solche Orte deshalb Schauplätze lokaler Legenden: 
oder es werden wichtige Gebäude, wie zum Beispiel Tempel und Kultstätten auf diesen Kontaktstellen errichtet. 

Im Falle der untersuchten Subglobule, die dem Element Vita zuzuordnen ist, zeichnet sich die Kontaktstelle durch üppige 
Vegetation aus, die trotz des kargen, schieferhaltigen Untergrunds genügend Nahrung findet um sich zu erhalten. Da sie 
sich aufgrund von Dislocationen in den Kraftlinien Condras während der Untersuchungen verschob, konnte die Veränderung 
des Ortes, der innerhalb weniger Tage zur Kontaktstelle wurde, gut beobachtet werden. In einer quantitativen Bestimmung 
der Flora wurde eine Verdreifachung des ursprünglichen Bewuchses und eine Verdoppelung in der Vielfalt der Arten 
festgestellt.

Mit den übernatürlichen Sinnen ist die Kontaktstelle einer Globule nicht leicht zu erfassen, da sie in allen bekannten 
Fällen an starken Kraftlinien oder –knoten liegen. Die Kraftströme überlagern beinahe immer die einfachen Methoden der 
übernatürlichen Sicht, so dass der Forscher auf komplexere Messmethoden angewiesen ist um die Nähe einer Globule zu 
beweisen. Da durch die Verschiebung des Liniennetzes die Astralkarten Condras ohnehin überarbeitet werden müssen, wurde 
eine eher zeitaufwändige Kraftlinienvermessung nach Silbenschwenk durchgeführt. Mittels dieser Ergebnisse war es möglich, 
an verschiedenen, fest definierten Punkten der fraglichen Linie Klicks über eine bestimmte Zeit zu messen und damit die 
Oszillation zwischen den Kompartimenten der Linie zu berechnen.
Stößt man in der normalerweise konstanten Schwingung zwischen zwei Kompartimenten auf eine Fluktuation zwischen 40 und 
150 Klicks pro Pendelschwung, kann man davon ausgehen, eine Kontaktstelle gefunden zu haben. Wichtig ist hierbei, dass 
alle möglichen astralen und elementaren Phänomene geringe Fluktuationen (unter 3 Klicks Differenz pro Pendelschwung) 
hervorrufen, so dass die bloße Anwesenheit eines übernatürlich Begabten schon eine Abweichung produzieren kann. Um 
möglichst genaue Ergebnisse zu erzielen ist es nötig, die Impulse sämtlicher bekannter Faktoren der Messung getrennt und 
abgeschirmt zu bestimmen um dann die Summe der Klicks aller Störfaktoren von den tatsächlich gemessenen abzuziehen. 

Alternativ kann ein erfahrener Begabter, der in der Lage ist, Kraftlinien mit mindestens drei seiner übernatürlichen Sinne 
zu erfassen, sich in die Struktur eines Ortes einklinken und über die Kraftlinien den Zugang zur Globule erspüren. Ob der 
Zugang zum Kraftnetz durch Meditation oder einen physischen Übergang in die Astralebene erfolgt ist hierbei egal: jeder 
Magier oder Elementarist sollte die Methode verwenden, mit der er am Besten vertraut ist. Die häufigsten mit einer 
Kontaktstelle assoziierten Eindrücke sind der Geruch nach Kampher, das Gefühl, gegen den Strich über ein Brennnesselblatt 
zu streichen und der folgende visuelle Eindruck, der wie ein Fortsatz des Kraftlinienbündels wirkt und in Richtung der 
Globule abzweigt:

Bei dieser Methode darf niemals die Gefahr durch ein verunreinigtes oder gar verseuchtes Kraftnetz außer Acht gelassen 
werden. Sind die Bewohner der Globule feindlich gesonnen oder wird das Kraftnetz außerhalb der Globule durch ein 
feindliches Wesen kontrolliert, kann das Einklinken in ein astrales oder elementares Netz einen Angriff provozieren. 

\subsection {Kapitel III: Wie man die Natur einer Globule bestimmt}

Anhand der makroskopischen Einwirkungen der Globule an der Kontaktstelle kann meist schon eine grobe Aussage über das 
erschaffende Wesen und den Zweck der Globule getroffen werden. Bei elementaren Globulen beobachtet man beispielsweise 
immer einen Einfluss der primären Eigenschaften des entsprechenden Elements, wie Abweichungen im Mikroklima des Ortes oder 
Veränderung in Boden, Flora und Fauna. 

Eine sehr einfache Methode die Natur einer Globule zu erforschen ist es, den Erschaffer derselben zu kontaktieren und sich
mit ihm zu unterhalten. In den meisten Fällen handelt es sich um ein intelligentes Wesen das sich seiner Umgebung bewusst 
ist und sich oft sogar über eine gepflegte Konversation freut. „Wenn man die Konstruktion eines Gebäudes erforschen will, 
ist der Weg über den Erbauer meist der Beste“: dieser Merksatz gilt für Globulen ebenso wie für alles von intelligenten 
Wesen geschaffene. 
Ist es nicht möglich, mit dem Erschaffer der Globule zu kommunizieren, sind umständlichere Methoden vonnöten um Art und 
Nutzen einer Globule herauszufinden. Zunächst sollte man davon absehen, blind in die Globule hineinzustolpern ohne zu 
wissen, was für Bedingungen man vorfindet. Es ist wichtig, sich mit der Struktur der Hülle vertraut macht bevor man 
unkontrolliert Löcher in sie hineinreißt. Erst wenn die Hülle bekannt ist, kann man zunächst unter kontrollierten 
Bedingungen niedere Lebewesen (anfangs Pflanzen, später Tiere) in die Globule transferieren um zumindest eine 
Lebensfeindlichkeit des Globuleninneren auszuschließen. Sind diese beiden Vorraussetzungen – bekannte Hüllenstruktur und 
lebensfreundliche Bedingungen im Inneren-  erfüllt, kann sich auch ein höheres Wesen in die Globule hineinwagen um 
Messungen und Beobachtungen durchzuführen. 

\newpage

\subsection{Kapitel IV: Die Ergebnisse einer Untersuchung }

Die Kontaktstelle 
Aufgrund makroskopischer Beobachtungen (untypischer Pflanzenbewuchs auf sonst kargem Schieferboden, zufälliger direkter 
Kontakt zum erschaffenden Wesen durch bloße Meditation, großes Aufkommen minderer Humus-Elementargeister) konnte die in 
Frage kommende Kraftlinie und die ungefähre Strecke leicht eingegrenzt werden. Die direkte Verbindungslinie zwischen den 
Knoten in Schieferbruch und denen oberhalb von Tharemis wurde als Kontaktlinie ermittelt: der Ort der Messung waren je 800 
Schritt in jede Richtung vom Goldkrug-Dreieck aus. Dieser 1600 Schritt lange Kraftlinienabschnitt wurde in 32 Kompartimente 
aufgeteilt, die einzeln vermessen und verglichen wurden. Mittels einer einfachen Zählung der elementarmagischen Ausschläge 
(Klicks) über eine Zeit von 10 Pendelschwung, die mit einem 1:5-Pendel bestimmt wurde, konnte das Kompartiment, in dem sich 
die Kontaktstelle befindet ermittelt werden.
Als zusätzliche Referenz wurden der Impuls der ersten 8 Klicks und das dominierende Element aller Klicks mittels 
Farbendiagnostik bestimmt. Auffällig waren die zwei Nachbarkompartimente der Kontaktstelle: das dominierende Element war 
in beiden Fällen Cryo, was erstens selten und zweitens für einen Ort in direkter Nachbarschaft zu einer Vita-Globule sehr 
untypisch ist. 


Ein Auszug aus den Rohdaten der Messung:

\begin{table}[htbp]
\begin{center}
\begin{tabular}{|c|c|c|}
\hline
Kompartiment	&dominantes Element	&Impuls	\\
\hline
VI	&18	&90	\\
VII0	&20	&100	\\
VIII	&24	&120	\\
IX	&18	&90	\\
X	&20	&100	\\
XI	&19	&95	\\
XII	&16	&80	\\
XIII	&13	&65	\\
XIV	&39	&195	\\
XV	&15	&75	\\
XVI	&17	&85	\\
XVII	&21	&105	\\
XVIII	&18	&90	\\
XIX	&18	&90	\\
XX	&16	&80	\\
XXI	&20	&100	\\
XXII	&19	&95	\\
XXIII	&17	&85	\\
XXIV	&17	&85	\\
\hline
\end{tabular}
\label{Winkel der hochsymmetrischen Achsen zu (111)}
\end{center}
\end{table}

Die gemessene Kontaktstelle erwies sich als korrekt: es konnte an diesem Ort sehr einfach eine direkte Verbindung zur Hülle 
der Globule hergestellt werden.

\subsubsection{Die Hülle}
Das Studium der Hüllen von mehreren Globulen zeigt, dass eine Barriere aus Nichts, ein beinahe perfektes Vakuum die Globule 
von den Ebenen trennt. Selbst für die meisten Ebenen feste Dimensionen wie Raum oder Zeit sind in diesem Raum nicht 
vorhanden. Es ist für Materie, Zeit, Seele und Energie unmöglich, diese Barriere zu überwinden. Gleichzeitig ist sie von 
beiden Seiten von einem Film oder einer Membran aus sehr schweren Teilchen überzogen, die verhindern, dass der leere Raum 
kollabiert. Die Natur der Teilchen der Innenmembran variiert von Globule zu Globule, während die Außenmembran bei allen 
untersuchten Globulen gleich aufgebaut war. 
Die Untersuchung einer dem Chaosgott Slaneesh zugeordneten, klerikalen Subglobule ergab beispielsweise, dass die 
Innenmembran aus Seelenpartikeln -manchmal auch „Staub“ genannt- bestand, die mittels einer emotionsähnlicher Energie 
vernetzt waren, die leider nicht weiter untersucht werden konnte. 
Der Film auf der inneren Membran der untersuchten Humus-Globule erwies sich als weniger abstrakt: es handelt sich um wahres 
Element der Art Vita. 
Egal welcher Natur die Innenmembran ist, sie erlaubt den Austausch von Materie, Zeit, Seele und Energie zwischen der Globule und den anliegenden Ebenen. Mit etwas Krauftaufwand kann selbst ein Fremder, der die Globule weder bewohnt noch sie erschaffen hat die Partikel beugen und sie so anordnen, dass sie einen Durchgang bilden und das Ein- oder Austretende vor der Barriere aus Nichts schützen. 
Die Außenmembran bestand in allen untersuchten Fällen aus Orichalkum, der perfekten Vereinigung aller sechs Elemente. Da die Membranen von Sphärengrenzen oft ebenso aufgebaut sind, liegt es nahe, dass ein Wesen wie ein Humus-Dschinn, der von Natur aus nicht in der Lage ist Orichalkum zu erschaffen, eine Sphärengrenze beugt, sie zur Membran seiner Globule formt und die Innenseite nach seinen Bedürfnissen auskleidet. Wesen, die nicht einer elementaren Ebene entstammen, sind demnach nicht auf eine bereits vorhandene Sphärenmembran angewiesen, da sie –natürlich nur mit erheblichem Aufwand- theoretisch in der Lage sind Orichalkum herzustellen.
Versuche ergaben, dass eine kurzlebige und instabile Globule auch mit weniger komplexen Außenhüllen konstruiert werden kann. Eine Vernetzung von wahrem Element ergibt beispielsweise eine Hüllmembran, die je nach Güte der Vernetzung einige Stunden bis zu mehreren Tagen hält, bevor die Globule kollabiert. Allerdings kann in einer solchen Globule kein komplexer Energiekreislauf hergestellt werden ohne die Membran zu zerreißen. Masse und ideelle Faktoren hingegen stellen hierbei kein Problem dar. 

\subsubsection{Der Erschaffer}
Es handelt sich um einen alten, benannten Humus-Dschinn, dessen Name vor allem in Dokumenten auftaucht, die den Elfen 
zugeordnet werden. Es ist sehr wahrscheinlich, dass dieser Dschinn von den Elfen seinen heutigen Namen und seine Existenz 
in dieser Ebene erhielt: 


Eru Loa Rava

Vor und während der Zeit der nekanischen Besatzung wurde weder dem Dschinn noch der Subglobule viel Aufmerksamkeit zuteil: 
die Kontaktlinie war zwar schon immer eine recht wichtige Verbindung von Oben nach Unten, zwischen Schieferbruch und 
Tharemis, aber da EruLoaRava kaum Einfluss auf seine Umwelt nahm und selten mit Menschen in Kontakt trat wurde er 
schlichtweg nicht bemerkt. Erst mit der Verschiebung der Kraftlinien am Nachtwall und den daraus folgenden Neuvermessungen 
wurde die Kontaktstelle wieder entdeckt und zunächst nicht als solche erkannt sonder als vernachlässigbare Auffälligkeit 
vermerkt. Durch die Errichtung des Goldkruges in der Mitte dreier sich kreuzender Kraftlinien, zu denen auch die 
Kontaktlinie der Subglobule gehört („Goldkrugdreieck“), kam der Dschinn mit Menschen und schließlich auch mit 
Elementaristen in Kontakt. Die erneute Verschiebung des Netzes im Sturm auf Schieferbruch schob das Dreieck enger um den 
Goldkrug und verdichtete auch die Verbindung der Kontaktlinie zu dem vielgenutzten Knoten im Goldkrugdreieck. 

Die Kontaktaufnahme erwies sich als recht einfach, da EruLoaRava ein neugieriges und geduldiges Wesen ist, mit dem die 
Kommunikation geradezu ein Vergnügen ist. Bisher wurde noch keine physische Manifestation beobachtet. Vielmehr wurde die 
Präsenz eines kollektiven Geistes beobachtet, die in der Lage ist, auch menschliche Körper zu übernehmen. Hierbei geht der 
Dschinn allerdings recht rücksichtsvoll vor: wehrt sich der den Körper bewohnende Geist, zieht er sich sofort zurück. 
Außerdem kann der Autor aus eigener Erfahrung berichten, dass EruLoaRava immer sehr sorgsam mit geborgten Körpern umgeht 
und ihre Strukturen in der Regel intakter zurücklässt als sie es vor der Übernahme waren. Dieses Verhalten konnte auch nur 
dann beobachtet werden, wenn eine größere elementare Störung im Einflussbereich der Globule vorlag. Offensichtlich 
interessiert sich die Entität sehr für das, was in seiner Umgebung vorgeht, greift aber nur dann in der physischen Ebene 
ein, wenn es sein Heim gefährdet sieht und dann auch nur mit Hilfe fremder Körper, nie mit der eigenen physischen 
Anwesenheit. Tut er dies, zeigt er sich allen Phänomenen der physischen Ebene gegenüber sehr wissbegierig und freut sich 
über jede Geschichte, die ihm erzählt wird und jeden sinnlichen Eindruck der ihm gewährt wird. Dieses Verhalten und die 
kollektive Natur der Entität legen nahe, dass es für den Dschinn schwierig ist, eine singuläre Präsenz zu bilden und er 
deshalb nicht ohne weiteres auf der phyischen Ebene erscheinen kann oder will. 

Das Innere
Die Bedingungen in der Subglobule erlauben die Existenz von Pflanzen und Lebewesen der physischen Ebene. Allerdings neigen 
die Dinge im Inneren der Subglobule dazu, zu verschmelzen, sich zu vereinen und neue Dinge hervorzubringen, so dass Materie, 
Geist und Seele nur in einem stark beschleunigten, aber dennoch den natürlichen Gesetzen der physischen Ebene gehorchenden
Kreislauf existieren. Ein Wesen mit Bewusstsein kann in dieser Umgebung nicht ohne weiteres seine Identität bewahren und 
läuft Gefahr, sich in diesem Kreislauf zu verlieren. Bereits bei einem kurzen Besuch bemerkt man ein Ziehen in der Struktur, dem man sich nur durch absolute Kontrolle der eigenen Physis entziehen kann. Eine ungeübte Person verschmilzt beinahe sofort mit der Vegetation der Globule und kann nur mit Gewalt und den daraus resultierenden Strukturschäden wieder aus ihr entfernt werden. 
Diese Eigenschaften machen eine direkte Erforschung des Inneren sehr schwierig, erlauben aber interessante Rückschlüsse auf 
den oder die Bewohner der Globule. So decken sich die in der Subglobule gemachten Erfahrungen mit schriftlichen Berichten, 
in denen die Bedingungen auf der Elementarebene des Humus festgehalten wurden. Lediglich die distinktive Vegetation fehlt 
in den Berichten, in denen eine undefinierbare, ständig die Form wechselnde Masse beschrieben wird anstelle der Ranken, 
Bäume und Sträucher, die in der Subglobule zu erkennen waren. Man kann also einen direkten Vergleich zwischen der 
Subglobule des Dschinns und seiner ursprünglichen Heimat, der Ebene des Humus ziehen. Für einen geübten Elementaristen ist 
es also möglich, in der Subglobule Techniken und Methoden zu erproben, die ihm in der Elementarebene seine Identität 
bewahren könnten. Der physische Übertritt in eine Elementarebene bleibt dennoch ein nahezu wahnsinniges Unterfangen, auch 
wenn die Mechanismen einer solchen Umwelt mit Hilfe der untersuchten Subglobule nun besser bekannt sind. 


\subsection{Kapitel V: Die Schaffung einer Globule}

Mit den erhobenen Daten und den gemachten Erfahrungen verstehen wir nun die Natur von Globulen dergestalt, dass die Idee 
einer menschgeschaffene Globule in greifbare Nähe rückt. 
Die größte Schwierigkeit besteht in der Auswahl einer geeigneten Substanz, mit der eine stabile Hülle geformt werden kann. 
Natürlich ist Orichalkum die beste Wahl, aber seine Herstellung ist äußerst aufwendig und gefährlich, weshalb in einer 
Demonstration davon abgesehen wird, die Außenmembran aus Orichalkum zu fertigen. Für ein konkretes Projekt allerdings ist 
es unabdinglich, das kostbare Material zu nutzen, da eine Globule nur mit einer Hüllmembran aus Orichalkum dauerhaft und 
stabil ist. 

Es muss also eine Alternative zu Orichalkum gefunden werden. Diese Substanz muss großen Energien widerstehen können und 
darf weder starr noch weich sein. Sie darf durch Einflüsse von Kräften, Ideen oder Emotionen nicht veränderlich sein und 
muss gleichzeitig genügend physische Präsenz aufweisen um eine Barriere für Materie und materielle Energie darzustellen. 
Nach dem Vorbild der Humus-Globule im Goldkrugdreieck wurden Versuche mit wahrem Element Humus gemacht, die akzeptable 
Ergebnisse lieferten. Je reiner eine Substanz ist, desto starrer ist auch ihre Struktur, weshalb selbst die Verwendung von 
wahrem Element keine perfekte Lösung darstellt. Da Humus allerdings ein flexibles Element ist, funktioniert es als 
Schutzschicht der Hülle einer Globule, auch wenn vernetzte Strukturen mehrerer unterschiedlicher Komponenten 
(beispielsweise Verbindungen aus physischen und ideellen Partikeln) eine elegantere und vermutlich auch stabilere Lösung 
wären. Da wir vorerst nur beweisen möchten, dass eine Globule auch von Menschen geschaffen werden kann, nehmen wir mit der 
etwas plumperen, aber dafür weniger aufwändigeren Lösung vorlieb und nutzen wahres Element Vita um die Hülle abzuschirmen.

Um die Partikel in der gewünschten Form anzuordnen ohne dass sie sofort in die Umgebung diffundieren wird eine stützende 
Sphäre benötigt. Ob die Sphäre materiell oder energetisch aufgebaut ist spielt keine Rolle, solange sie leicht aus der 
fertigen Struktur entfernt werden kann. Hat man also eine Schutzhülle aus wahrem Element Vita um die Sphäre konstruiert, 
wird jegliche Materie und Energie aus dem Zwischenraum entfernt. Hierfür wird bei der Konstruktion der Schutzhülle eine Art 
Reuse eingebaut, die es erlaubt, Materie und Energie nach außen abzuführen und den Rückstrom gleichzeitig aufzuhalten. 
Schließt man die Austrittsstelle an ein System mit einer Energie- und Masseninsuffizienz an, reicht ein einziger kräftiger 
Impuls um den gesamten Inhalt der Hülle in dieses Hilfssystem abzuleiten.
Um eine perfekte Hülle zu schaffen ist es nötig, neben Energie und Masse auch Zeit und Seele aus dem Zwischenraum zu 
entfernen. Große Teile dieser abstrakten Substanzen heften sich zwar an die ausströmenden physischen Substanzen, aber eine 
komplette Leere zu schaffen ist einem Menschen mit seinen eigenen Mitteln nicht möglich. Hierbei machen wir uns eine 
Eigenart der Elemente Mitrasperas zunutze, einem Kontinent auf dem „die Leere“ ein Element darstellt, dessen Essenz von 
entsprechend sensitiven Wesen erfasst und genutzt werden kann. Ein winziger Teil der Essenz einer Viinshar, eingebaut in 
das Hilfssystem genügt, um auch die abstrakten und schwer greifbaren Energien abzuleiten. Um sicher mit dieser 
lebensfeindlichen Essenz umgehen zu können, ist es nötig diese stark zu verunreinigen. Dies schwächt den Effekt des 
Hilfssystems zwar ab,  genügt aber um eine stabile, undurchdringbare Hülle für eine kleine Globule zu erhalten. 


Es ist ratsam, den Inhalt der Globule vor der Konstruktion der Hülle zu implementieren, da es sehr kraftaufwändig ist, 
wahres Element zu beugen um einen stabilen Durchgang durch die Hülle zu erzeugen. Alternativ ist es möglich, die komplette 
Globule zu dehnen, die Hülle  kontrolliert kollabieren zu lassen und Dinge ähnlich einer Amöbe in die Globule aufzunehmen.
Danach muss der Innenraum der Hülle allerdings erneut entleert werden, außerdem ist eine kompakte, potente Kraftquelle 
nötig um die Globule gleichmäßig auszudehnen. 

Eine leere Globule zieht sich nach wenigen Stunden zusammen und implodiert spätestens nach zwei Tagen. Um eine Globule mit 
einer längeren Lebensdauer zu erschaffen ist es nötig, im Inneren einen stabilen Kreislauf zu etablieren. Es ist nicht 
zwingend notwendig, hierfür ein System mit einander ergänzenden Lebewesen zu konstruieren: ein simples abgeschlossenes 
System mit mindestens zwei Elementen genügt. Dieses System muss so lange funktionstüchtig sein, bis sich eine Kontaktstelle 
ausgebildet hat. Bei sämtlichen Versuchen erfolgte dies nach wenigen Stunden selbstständig und ohne Einfluss von Außen. 
Sobald die Kontaktstelle erkennbar ist, kann das System auch von außerhalb der Globule beeinflusst und der Kreislauf auch 
künstlich aufrecht erhalten werden. Es ist möglich, über die Kontaktstelle Kraft zuzufügen oder abzuziehen, Materie 
hingegen muss über einen aufwändigen Durchbruch in der Hülle hinzugefügt werden.

Eine solcherart gefertigte und stabilisierte Globule besitzt die gleichen Eigenschaften wie die in vivo beobachteten 
Subglobulen. In Größe und Komplexität des Innenlebens ist sie noch unterlegen, doch diese Faktoren sind lediglich eine 
Frage der eingesetzten Kraft. 


\section{ Narrativum}

\yinipar{M}"ogen sie mir verzeihen das ich in dieser Vorlesung ein wenig die ausgetretenen Fade der Naturbeschreibenden Wi"senschaften verla"se und auf die Geisteswi"senschaften der Philosophia und der Erkenntnistheorie zur"uckgreifen werde.
Narrativum ist keine tats"achlich existente Substanz, Mittel oder Energie vielmehr ein Hilfskonstrukt f"ur unseren Geist mit dem sich allerdings Tatsachen und Sachverhalte erkl"aren la"sen und beweisen la"sen. Narrativum befindet sich in allem und jedem, jedes noch so mikroskopische Objekt besitzt einen gewi"sen Wert an Narrativum und dem Prinzip der Superposition folgend besitzt jedes makroskopische Objekt, einen Narrativumswert, der der Summer der Werte seiner Einzelteile entspricht.
Narrativum ist ein Wert f"ur die Wichtigkeit eines Objektes im Raum Zeit Gef"uge. Das Wort selbst kommt von dem Verb narrare, erz"ahlen. Es ist ergo ein Ma"s f"ur die Tatsache ob es sich lohnt dar"uber zu erz"ahlen. Narrativum hat keine konkrete Ma"seinheit bzw. Dimension sondern wird bestenfalls mit Begriffen wie „viel“, „wenig“ oder „fast gar nicht“ geme"sen auch sind Vergleiche mit anderen Objekten und deren Narrativumsgehalt sehr schwierig bis gar unm"oglich und s"amtliche Einsch"atzungen absolut subjektiver Natur.
Ist diese vage Definition erst einmal verstanden ist das Arbeiten mit diesen Werten recht einfach man mu"s sich lediglich von der Vorstellung verabschieden objektive und absolute Werte und Au"sagen zu erhalten, in dieser Betrachtungsweise ist alles relativ.
Einige Bespiele zum Verdeutlichen:
Ein Stein der am Wegesrand liegt hat recht wenig Narrativum, denn keiner wird sich jemals an ihn erinnern und "uber ihr Erz"ahlen. Wenn nun allerdings ein Wagen "uber ihn f"ahrt und sich die Achse bricht, dann hat der Stein eine recht gro"se Menge an Narrativum, der H"andler wird sich nun die ganze Woche dar"uber "argern. Ist der Stein der Grund das die Achse bricht w"ahrend der H"andlern von Dieben auf der Flucht ist und die Diebe k"onnen ihn nur wegen des Steines stellen und seine ganze Ladung Rauben, dann wird sich der H"andler ein Leben lang "uber diesen Stein "argern und der Stein hat sehr viel Narrativum. W"urden die Diebe den H"andler t"oten w"urde er sich nicht "uber den Stein "argern, da er tot ist und den R"aubern w"ar es egal warum die Achse brach, also w"urde keiner "uber den Stein erz"ahlen, aber dennoch h"atte er einen sehr gro"sen Narrativumsgehalt weil er das Potential zu einer gro"sen Geschichte h"atte, zumindest aus unserer Sicht. Also noch mal zusammengefa"st letzterer Stein h"atte aus 
Sicht der R"auber und des H"andlers wenig aus unserer aber recht viel Narrativumsgehalt (da wir ja davon wi"sen das der Stein Schuld war), soviel zur Objektivit"at.
Ein Objekt hat schon einen objektiven Gehalt an Narrativum, doch wir verm"ogen ich nicht zu me"sen, da wir selbst subjektiv sind. Kaiser Jeldrik hat einen Immensen Anteil an Narrativum jeder erz"ahlt von ihm, ja wir benannten sogar unsere Zeitrechnung nach ihm, doch in anderen L"andern ist dieser Engonische Nationalheld recht wenig bis gar nicht bekannt, das schafft einen weiten Bogen an Variationen die Objektiv zu einem Mittelwert verarbeitet werden m"u"sten, doch da wir nicht wi"sen wie viele Menschen es in den Dimensionen gibt geschweige denn Tiere und Gegenst"ande (die ja auch durch Jeldriks Pr"asenz beeinflu"st worden sein k"onnten) kann unsere Sch"atzung nur subjektiv bleiben.

Wozu brauche ich nun einen Wert f"ur Narrativum, wenn er so unheimlich ungenau ist wird sich der geneigte Leser fragen, die Beispiele daf"ur sind recht zahlreich aber ebenfalls recht ungenau (dies scheint sehr in der Natur der Wahrscheinlichkeits und Sph"arenereigni"se zu liegen). Zum Beispiel die Wahrsagerei (nun sp"atestens werden viele Kollegen wieder mit den Augen rollen) und Hellseherei wird von gro"sen Ereigni"sen, sprich Ereigni"sen mit viel Narrativumsgehalt angezogen. Epische Schlachten und Schicksale von ber"uhmten Kriegern ziehen die Aufmerksamkeit der Zukunftsdeuter auf sich, so da"s man in unserer hermetischen Terminologie davon sprechen kann das das Narrativum die Hellseherei leitet und bestimmt was gesehen wird.
Die Fachgemeinde der Magier und Magierinnen wird aber ein konkreteres Beispiel fordern und ich werde es bringen, obwohl dies sehr kompliziert ist und einen Gro"steil der mir zur Verf"ugung stehenden Zeit beanspruchen wird, n"amlich die Zeitreisen.
Zeitreisen existieren, davon kann man nach Existenz des 5. Zirkel Zaubers „Zeitkontrolle“ als allgemein bekannte Tatsache ausgehen. Weiterhin ist es als gesicherte Tatsache anzusehen das die Zeit nicht linear verl"auft sondern Wirbel H"ugel und ebene Fl"achen aufwei"st.
„Die Zeit vergeht wie im Flug“ sagt man wenn einem ein langer Spaziergang mit seiner Liebsten im Park gut gef"allt oder „die Zeit stand still“ als sie sich sahen. Ebendies wird durch das Narrativum beeinflu"st, an schlimme Ereigni"se (gro"se Schlachten, viel Narrativum) erinnert man sich stark, „als w"are es gestern gewesen“ w"ahrend man die 5 Tage die man im Hafen auf ein Schiff wartet zwar in dem Moment als Lang empfindet, die aber nachher, da recht ereignislos als minderwertig und nicht erw"ahnenswert im Ged"achtnis erinnert werden. Diese Beispiele sollen aber nun keinesfalls einen linearen Zusammenhang zwischen Narrativumsmenge und Schnelligkeit der Zeit implizieren sondern vielmehr einen sehr viel komplexeren Zusammenhang.
Der weitaus weniger bekannte Zauber Cronocla"sis erlaubt es einen Gegenstand aus der Vergangenheit in die Gegenwart zu bringen z.B. Kaiser Jeldriks Schwert oder Hargats Glocke oder dergleichen, dabei kommt es sehr auf eine besondere Eigenschaft des Narrativums an n"amlich der Zeit/Raum Konsistenz. Der Narrativumsgehalt eines Gegenstandes ist Ort und Zeit unabh"angig uns bleibt immer gleich egal wann man ihn betrachtet. Ein nee geborenes Kind in einem kleinen Bauerndorf in Andarra w"are ja eigentlich nicht besonderes, aber der Narrativumsgehalt, dieses Kindes ist "uber alle Ma"sen hoch, so da"s jemand der zu diesem Zeitpunkt den Narrativumsgehalt feststellen k"onnte schon sagen kann das dieses Kind zum Andarrianischen Feldherrn und zuk"unftigen Herrscher "uber alle Engonier wird. Angemerkt sei hier auch der Selbstschutz des Narrativums vor Ver"anderung, denn sollte besagter Beobachter nun das Kind t"oten weil es ein so hohes Narrativum hat w"urde es nicht ber"uhmt werden und demzufolge ein geringeres 
Narrativum haben, dann aber h"atte der Beobachter das Kind nicht ge"otet, weil unwichtig, Paradoxon. Das Narrativum kennt, da es au"serhalb von Raum und Zeit steht, die komplette Geschichte eines Objektes und sch"atzt sich dementsprechend ein und bestimmt seine Gr"o"se, auf unser Beispiel bezogen hei"st das da"s das Narrativum schon vorher w"u"ste ob das Kind get"otet wird oder nicht.
Um auf den Cronokla"sis zur"uckzukommen hei"st dies das die magische Energie die ben"otigt wird um ein Objekt in unsere Zeitepoche zu versetzen direkt proportional zum Gehalt an Narrativum dieses Objektes ist. Ein Stein aus dem Wald fast ohne Narrativum bedarf fast keiner Energie um ihn herzuholen, besagtes Schwert des Kaisers aber sehr wohl.
Dies verdeutlicht einen weiteren Immensen Nutzen, den wir aus der Einf"uhrung von Narrativum ziehen k"onnen. Mit Hilfe des Gehaltes an Narrativum k"onnen wir die Auswirkungen von Zeitreisen auf das Raum/Zeit Gef"uge me"sen, verstehen und ihm vorbeugen. Sehen wir nun einmal von den sehr komplizierten Paradoxen ab k"onnen Zeitreisende durch ihre Handlungen in der Vergangenheit die Zukunft ver"andern. Eine g"angige Warnung in Verbindung mit Zeitreisen ist das auch die kleinste Ver"anderung in der Vergangenheit immense Auswirkungen in der Zukunft haben kann, man bedient sich oft der Analogie einen Stein in einen See zu werfen und die Wellen (wie durch die Zeit) sich immer weiter ausbreiten zu sehen. Dies ist mitnichten so einfach und pauschal zu sagen und soll an dieser Stelle durch Folgende Erkl"arung ersetzt werden, bis ich das Thema in einer zuk"unftigen Vorlesung noch einmal aufgreifen werde. Ein Zeitreisender ver"andert die Zukunft in dem Ma"se indem er Kontakt zu Narrativum in der fremden Zeitepoche hat 
und Objekte die hohes Narrativum tragen ver"andert. Will hei"sen ein Zeitreisender kann Gegenst"ande mit geringem Narrativum ver"andern soviel er Lust hat ohne das dies gro"se Auswirkungen (oder Irgendwelche) auf die Zukunft haben wird, sollte er aber Objekte mit hohem Narrativum aus ihrem Zeitrahmen rei"sen, wird dies besagten Domino bzw. Wellenkreis Effekt haben. Um sichere Zeitreisen zu erm"oglichen w"are es also recht angebracht einen „Narrativumsme"ser“ zu erfinden der einen Zeitreisenden sicher durch die Zeit geleiten kann.

Die Raum/Zeit Konsistenz erm"oglicht ein weiteres bedeutendes Gedankenexperiment, das „Diktatorische Schicksal“. Wir Menschen sehen unsere Zukunft oft gerne als Unbestimmt, das wir selbst unser Gl"uckes Schmied sind und das nichts in unserem Leben vorbestimmt ist. Andererseits sehnen sich viele oft nach einem vorbestimmten Leben einem Schicksal einem Pfad der ihnen vorbestimmt ist, so das sie an ihren Schicksal"schl"agen nicht selbst Schuld sind sondern es der vorbestimmten Zukunft in die Schuhe schieben k"onnen.
F"ur Vertreter beider Richtungen habe ich eine gute Nachricht, denn mit Hilfe des Narrativums kann man diese Frage entg"ultig l"osen.
Erinnern wir uns an die Eigenschaften des Narrativums, das Narrativum ist Raum/Zeit Konsistent also wei"s das Narrativum bereits im Voraus welche Zukunft dem Objekt bestimmt ist, oder be"ser ausgedr"uckt nicht im Voraus, denn das Narrativum steht au"serhalb der Zeit, es gibt f"ur das Narrativum kein „Vorher“. Das hei"st also das ein kleines Kind in einem Bauernhof in Andarra geboren mit einem immensen Gehalt an Narrativum kein Bauer werden und ein bescheidenes Leben f"uhren wird, egal f"ur was es sich entscheidet, das Schicksal wird zuschlagen und ihn auf einen Weg zur Ber"uhmtheit f"uhren.
Nun kommt die gute Nachricht f"ur Diejenigen die die Theorie der frei w"ahlbaren Zukunft bevorzugen, einfach aus der Tatsache, das wir Narrativum nicht genau me"sen k"onnen. Wir k"onnen Narrativum nicht genau me"sen nicht weil wir nicht die Ger"ate und Methoden dazu haben, sondern weil es schlicht und einfach nicht perfekt zu bestimmen ist.
Also kann man zusammenfa"send sagen das das Schicksal von jedem von uns zwar festgeschrieben ist wir es aber unter keinen Umst"anden me"sen oder bestimmen k"onnen.

Dieses Gedankenexperiment "uber das Diktatorische Schicksal liefert uns zusammen mit der Unbestimmtheit des Narrativumswertes das wir von nun an Unsch"arferelation nennen werden eine wi"senschaftlich fundierte Erkl"arung zur Hell- und Wahrsagerei. Der Grund warum Hell- und Wahrsagerei in der Fachwelt einen so schlechten Stand haben ist ihre Ungenauigkeit. „Wenn sie die Zukunft vorhersagen k"onnen wieso kleiden sie sich dann in nebul"ose Phrasen und Metaphern, die mannigfaltige Interpretationen erlauben“ h"ore ich so manche Magi und Magae sprechen. Manche sehr akademische Magier und Magierinnen gehen sogar so weit zu behaupten, Wahrsagerei w"are ganz und gar unm"oglich und die Ereigni"se w"urden nur manchmal eintreffen, weil die Vorau"sagen so wage formuliert sind das sie sich zuf"allig auf ein Ereignis beziehen. 
Mit dem nun gelernten k"onnen wir diesen Sachverhalt aber nun ebenfalls kl"aren. Wahrsagerei, eine Kunst von der ich selbst nur die rudiment"arsten Kenntni"se aus meiner Grundausbildung besitze, ist die Kunst oder Methodik zu beschreiben und zu me"sen, was nicht exakt beschrieben oder geme"sen werden kann. Wie eben bereits beschrieben steht die Wahrsagerei in direkter Beziehung zum Gehalt des Narrativums in einer bestimmten Substanz/Objekt, doch aufgrund der Unsch"arferelation kann man diesen konkreten Wert des Narrativums nicht me"sen, Also mu"s ein unbestimmter Wert geme"sen, oder be"ser gesagt etwas "uber ihn ausgesagt werden und daraus entstehen die nebul"osen und zweideutigen Phrasen auch bei den seri"osesten Wahrsagern und Wahrsagerinnen. 
Ein Zahlenbeispiel verdeutlicht dies: Angenommen das Narrativum h"atte einen Wert im abgeschlo"senen Intervall von 8 bis 12 irgendeine der Zahlen 8,910,11 oder 12, eine bestimmte, die sich nicht "andert, aber die keiner kennt. Nun m"u"ste man dieses Intervall me"sen und wie wir Wi"sen hat jede Me"sung eine Ungenauigkeit. Sagen wir ein guter Wahrsager kann mit einer Genauigkeit von +- 1 wahrsagen und ein schlechter Wahrsager mit einer Genauigkeit von +-3. Also w"urde der gute Wahrsager ein Intervall von 7 bis 13 und der schlechte ein Intervall von 5 bis 15 angeben, also beides sehr vage Au"sagen, die allerdings beide mit absoluter Sicherheit zutreffen w"urden. Beide wollen nun ihr Ergebnis konkretisieren um in der magischen Fachwelt Anerkennung zu erhalten und beschr"anken ihr Intervall auf drei bzw. f"unf Ergebni"se jeweils am oberen Ende ihres Spektrums. Nun hat der gute ein Intervall von 11 bis 13 und der schlechte ein Intervall von 11 bis 15, also der gute sich auf einige Metaphern beschr"ankt, w"ahrend 
der schlechte noch recht viele nebul"ose Beschreibungen verwendet. Wir wi"sen das der Wert des Narrativums irgendwo zwischen 8 und 12 (inklusive) zu finden ist, also eine Chance von 20\% f"ur die 8: 20\% f"ur die 9 usw. besteht. Im Intervall der Wahrsager liegen nun die Werte 11 und 12, die je eine Wahrscheinlichkeit von 20\% also zusammen 40\% haben. Das hei"st das der gute Wahrsager mit seiner recht konkreten Vorhersage eine Chance von 40\% hat das Richtige vorhergesagt zu haben und der Schlechte mit seiner recht nebul"osen auch nur eine Wahrscheinlichkeit von 40\% hat. Hat der gute Wahrsager nun Gl"uck wird die Fachwelt erstaunt aufhorchen und auf reproduzierbare Ereigni"se warten um die Daten zu verifizieren, hat er Pech und die 60\% Chance schl"agt zu werden ihn wieder nur alle auslachen und sagen „das h"atten wir ihnen vorher sagen k"onnen“. Der schlechte Wahrsager hat in jedem Fall verloren, denn trifft sein Ereignis ein (40\%) dann werden alle sagen „na deine Vorhersage war so ungenau wenn Ereigni"se 
13, 14, 15 eingetreten w"aren h"attest du ja auch recht gehabt, obwohl sie nicht eingetreten sind bzw. h"atten k"onnen“ und wenn er falsch liegt w"are das f"ur die Wi"senschaftler zu erwarten gewesen. Im besten Fall, also das der Gute Recht hatte (was nur bei 4 aus 10 Versuchen der Fall ist) hatte die Fachwelt auf reproduzierbare Ergebni"se gewartet (wie man es immer bei Experimenten macht, Ein Experiment mu"s bei gleicher Durchf"uhrung immer wieder zu dem selben Ergebnis kommen um etwas empirisch zu Beweisen) doch das n"achste Experiment w"urde wieder nur zu 40\% klappen, und irgendwann wird der gute Wahrsager auch Pech haben und seine Vorhersage trifft nicht zu (was bei den Prozentsatz eigentlich die Regel sein sollte), dann wird sich die Fachwelt auf ihn st"urzen und seine Theorie und Me"smethoden f"ur unbrauchbar erkl"aren und wieder behaupten Wahrsagerei w"are Unm"oglich. 
Also zusammengefa"st egal wie gut der Wahrsager ist, er wird aufgrund der Struktur des Narrativums nie eine Wi"senschaftlich Hieb und Stichfeste Au"sage treffen k"onnen. An dieser Stelle sei dann noch angemerkt das wenn der schlechte seine Au"sage so konkretisiert h"atte wie der Gute also auf einen Intervall von 13 bis 15 er gar keine Chance gehabt h"atte das sein vorhergesagtes Ereignis eintrifft, und abgesehen davon gibt es in dieser Zunft auch noch jede Menge Hochstapler, die gar nichts vern"unftig vorhersagen k"onnen und somit diese Zunft noch tiefer in Diskredit bringen als die sowieso schon der Fall ist.

Eine ausf"uhrliche Erkl"arung zur Unsch"arferelation bin ich ihnen bis jetzt schuldig geblieben und habe sie lediglich als gegeben vorausgesetzt um andere Sachverhalte zu beweisen, dieses Defizit werde ich nun nachholen.
Pauschal l"a"st sich sagen je mehr Narrativum ein Gegenstand hat, desto schwerer ist seine Menge zu bestimmen, beziehungsweise der Umkehrschlu"s, je weniger Narrativum ein Gegenstand hat desto genauer kann man de"sen exakten Wert eingrenzen. Dies l"a"st sich durch folgende Formel leicht ausdr"ucken  N * delta N = h quer . N  ist der Wert den das Narrativum annimmt,  delta N  das Intervall in dem sich der Wert des Narrativums befindet und h quer ist eine einfache Konstante, die recht unwichtig ist und auf die ich hier nicht n"aher eingehen werde. Einige Beispiel um das zu verdeutlichen ein einfacher Stein irgendwann irgendwo im Wald der nichts besonderes erleben wird hat einen sehr geringen Anteil an Narrativum keiner nichts und niemand intere"siert sich f"ur ihn, er ist lokal und global vollkommen unwichtig. Daraus folgt er hat sehr wenig Narrativum das sehr genau den Wert sehr wenig hat. Auf der anderen Seiten haben wir Kaiser Jeldrik, der von immenser Bedeutung f"ur das Engonische Volk ist, jeder kennt ihn 
und jeder erz"ahlt sich von seinen Ruhmestaten, er hat einen sehr hohen Wert an Narrativum ohne Zweifel. Doch au"serhalb von Engonien ist er nur m"a"sig bekannt und in weit entfernten L"andern und fremden Dimensionen vielleicht gar nicht, was f"ur sich alleine gesehen einen geringen Wert an Narrativum andeuten w"urde. Also ist er Lokal von Bedeutung und hat sehr viel Narrativum aber da Narrativum Raum/Zeit Konsistent ist m"u"sen wir einen globalen Wert angeben und da wir nicht wi"sen wie gro"s „alles“ ist ist dies unm"oglich. Daraus folgt das Kaiser Jeldriks immenser Wert an Narrativum nur sehr sehr grob angegeben werden kann.
Weder die St"arke des Narrativums noch die Abweichung vom Exakten Wert kann Null betragen, denn hquer ist eine positive Zahl ungleich Null. Das hei"st es gibt kein Objekt was kein Narrativum enth"alt, auch der unwichtigste Stein ist ein ganz, ganz, ganz klein wenig wichtig, irgendwann und es gibt kein Objekt, keinen Gegenstand de"sen Wert an Narrativum absolut genau geme"sen werden kann, es gibt immer einen kleinen Fehler und sei er noch so klein.
\newpage
Nun noch ein kleiner Ausflug in die fundamental Elementartheorie und dann n"ahert sich diese Vorlesung auch schon dem Ende. Jeder Objekt enth"alt f"ur sich Narrativum und sein das Objekt auch noch so klein. Spaltet man ein Holzscheit in zwei Teile enth"alt jedes Teil wiederum Narrativum spaltet man nun eins der Teile, erleben wir wiederum das gleiche, diesen Proze"s setzen wir weiter fort und spalten die immer eins der Teile bis das Holzscheit so klein istdas man es nicht mehr spalten kann. Nun befinden wir uns auf der Elementarebene, jedes Objekt und jedes Wesen besteht aus den fundamentalsten Bausteinen des Seins, den Elementen Feuer, Wa"ser, Erde, Luft. Sie sind die unteilbar kleinsten Teilchen aus denen alles besteht. Das hei"st jedes Element hat einen eigenen Wert an Narrativum und sie kombinieren all ihre Narrativumswerte um dem Holzscheid, das sie bilden einen festen Wert an Narrativum zu geben und alle Holzscheite kombinieren wiederum ihre Werte um dem Lagerfeuer das aus ihnen besteht einen festen 
Narrativumswert zuzuschreiben. Dieses Prinzip nennt man Superposition, alles steht mit allem in Wechselwirkung und bedingt sich untereinander. So werden alle mikroskopischen bis hin zu allen makroskopischen Dingen, Objekten, Subjekten, Wesen und was man sich sonst noch so alles vorstellen kann mit einem einzigartigen Wert ausgestatten, der ihre komplette Existenz bestimmen wird, den nichts und niemand aber so recht fa"sen kann.

Ich hoffe ich habe ihnen mit dieser Vorlesung einige neue Blickwinkel aufgezeigt und M"oglichkeiten in die Hand gegeben ihre Erkenntnis der Erde des Himmels und dem ganzen Rest ein wenig zu erweitern. Querverweise und meine anderen Vorlesungen finden sie wie "ublich in der Bibliothek der Akademie Ayd Owl in Engonien.
\vspace{10mm}
Mit freundlichen Gr"u"sen
Florian Phelleas Ph"onixflug, Bund zu Ayd Owl
im Jahre 254 nach Jeldrik oder 1502 der alten Zeitrechnung

\newpage

\section{ Transdimensionale Dislocation}

\yinipar{W}ir beginnen die transdimensionale Dislocation mit einer Abhandlung "uber Deterministisches Schicksal und Chaos. Was unsere eigene Zukunft anbelangt sind wir Menschen oft zweierlei Meinung und das oft zur selben Zeit. Zum einen w"unschen wir uns unser Schicksal selbst in die Hand nehmen zu k"onnen und unser Gl"uckes eigener Schmied zu sein zum anderen sehnen wir uns nach Stabilit"at und der Sicherheit eines vorbestimmten Lebens.

Nun zu ergr"unden wann die Zukunft vorherbestimmt ist und wann nicht ist Subjekt dieser Vorlesung. Der Sinn dieser Fragestellung ist an dieser Stelle allerdings weder die Zukunft vorherzusagen noch theologische Fragen aufzuwerfen, sondern lediglich eine Vorarbeit f"ur Zeitreisen zu leisten. An dieser Stelle m"ochte ich ebenfalls noch einmal ausdr"ucklich darauf hinweisen, da"s der Begriff Chaos, den ich im Zusammenhang mit dieser Vorlesung gebrauchen werde lediglich eine gro"se Unordnung beschreibt und in keinerlei Zusammenhang mit G"ottergestalten oder zurecht verbotenen Spielarten der Magie steht.

Wie wir aus grundlegenden Kenntni"sen der Dimensionstheorie wi"sen gibt es viele, wenn nicht gar unendlich viele Dimensionen die alle mit der Dimension der Zeit verkn"upft sein k"onnen. Die Zeit me"sen wir durch einen periodischen, sich immer wiederholenden Vorgang, der seine Periode im zu me"senden Zeitraum nicht "andert. Das bedeutet, da"s wir die Zeit durch eine einzelne Zahl beschreiben k"onnen. Da wir allerdings nicht wi"sen ob die Zeit einen Anfang oder ein Ende hat, zumindest nicht innerhalb dieser Vorlesung, m"u"sen wir einen beliebigen Punkt als Fixpunkt erw"ahlen und von dort aus sowohl vorw"arts, als auch r"uckw"arts z"ahlen. In der Praxis ist dies die Thronbesteigung Kaiser Jeldriks, seit der nun 255 Jahre vergangen sind, respektive die alte Zeitrechnung, nach der wir das Jahr 1503 schreiben. Diese Zahl ist eine einzige Gr"o"se, weswegen der Raum der Zeit auch nur eine einzige Dimension besitzt. Soweit h"ort sich die Theorie recht einfach an, allerdings sei zu bedenken, da"s die Zeit nicht linear 
ist und unter anderem als imagin"ar aufgefa"st werden kann, was allerdings ebenfalls nicht Teil dieser Vorlesung sein soll und bitte an anderer Stelle gekl"art wird.

Zu jedem dieser Zeiten kann ein Ereignis in den anderen Dimensionen eintreten, d.h. es existiert eine eindeutige Abbildung von einem oder mehreren Punkten im Zeitraum in den der anderen R"aume mit deren Dimensionen und wieder zur"uck. Zum Beispiel trinke ich Tee heute, morgen und "ubermorgen, damit w"aren einem Ereignis drei Punkte auf der Zeitschiene zugeordnet. Dies erlaubt nat"urlich keine Au"sage "uber die Vollst"andigkeit, es ist ja nicht dar"uber bekannt ob ich gestern auch Tee getrunken habe.
Um uns in unserer Fragestellung, die uns am Anfang besch"aftige weiter zubringen m"u"sen wir uns fragen inwieweit ein Ereignis in der Zukunft feststeht, also ob ich morgen wirklich Tee trinken werde zum Beispiel. Dabei stehen wir vor einem gro"sen Problem, denn das ich morgen Tee trinken werde wird subjektiv von immens vielen Faktoren bedingt. Der Vorsatz morgen Tee zu trinken reicht ja nicht aus es tats"achlich war werden zu la"sen, ich k"onnte ja zum Beispiel morgen keine Teebl"atter finden, oder meine Teekanne k"onnte zerst"ort werden oder ich k"onnte durch noch unangenehmere Gr"unde daran gehindert werden.
Mit anderen Worten es gibt immens viele Ver"anderungen von meiner Planung die eintreten k"onnen. Die Anzahl dieser Ver"anderungen sprich die Abweichung von meinem geplanten Weg gebe ich mit exp Lambda an also einer Exponentiellen Steigerung.
Das Verfahren das ich nun beschreiben werde wird es uns erm"oglichen Ereigni"se zu finden, f"ur die sich die Zukunft vorhersagen l"a"st und wir werden lernen sie von denjenigen zu unterscheiden f"ur die uns dies, im Rahmen der f"ur die Vorlesung zur Verf"ugung stehenden Mittel, nicht m"oglich ist, dieses Verfahren werde ich Ph"onixflug Dimensionsweg Verfahren nennen.

Zuallererst betrachten wir eine Abbildung aller Dimensionen in den Phasenraum. Dadurch erhalten wir einen 2 mal f mal n dimensionalen Phasenraum mit f gleich der Anzahl der Freiheitsgrade in den betreffenden Unterr"aumen und n gleich der Anzahl der Unterr"aume, sprich Nebenwelten, Paralleldimensionen und so weiter. Dies ist nat"urlich eine viel zu gro"se Menge ist, als da"s sie unser Geist fa"sen k"onnte, aber das braucht er ja auch nicht, denn nun bedienen wir uns der Poincaré Phasenraumschnitte und zerlegen den Phasenraum in jeweils zweidimensionale Unterr"aume. Das geschieht einfach in dem wir alle Dimensionen bis auf zwei negieren, vergleichbar mit einem Querschnitt durch eine Torte. Diese Phasenraumschnitte sind f"ur uns ohne weitere Probleme zu betrachten, da sie z.B. auf einem Pergament simpelst dargestellt werden k"onnen.
Nun betrachten wir die Pr"asenz eines Ereigni"ses, dieses hinterl"a"st im Phasenraum eine Trajektorie, also die Punkte, die es auf andere Abbildet. Da die Poincaré Schnitte lediglich zweidimensional sind betrachten wir von der vieldimensionalen Trajektorie nur die Durchsto"spunkte. Diese k"onnen nun im Phasenraumschnitt entweder als zusammenh"angende Figuren oder chaotisch wild verstreute Punkte auftauchen. Zusammenh"angende Punkte implizieren einen Zusammenhang zwischen den einzelnen Teilereigni"sen, sind also vorhersagbar, w"ahrend wild verstreute Punkte wie ich oben bereits angedeutet habe sich absolut chaotisch, also unvorhersagbar verhalten. Wie wir an den ersten Experimenten in dieser Hinsicht gesehen haben stellen die verstreuten Teilereigni"se die Regel und die Figuren eher die Ausnahme dar, daher bezeichnen wir die Figuren, zumeist Ellipsen, als Inseln der Stabilit"at und die sie umgebenden verstreuten Punkte als Meer des Chaos.
Der Praktische Nutzen liegt nun darin, das die Teilereigni"se auf den Inseln der Stabilit"at in einer Beziehung zu den anderen stabilen Teilereigni"sen stehen, auch wenn diese mitunter sehr kompliziert werden kann. Also die Abweichung von meinem geplanten Weg ist was diese Teilereigni"se 
anbelangt nur noch linear und nicht mehr exponentiell. Wie sie wi"sen k"onnen lineare Entwicklungen selbstverst"andlich zu bestimmten Zeiten gr"o"ser sein als exponentielle, aber betrachtet man sie beim Gang ins Unendliche wir die exponentielle immer gr"o"ser werden als jeder noch so gro"se Faktor. Wie sie sehen haben wir uns einen Faden geschaffen, dem wir durch alle Dimensionen, die in Verbindung mit der Zeit stehen folgen k"onnen, weil wir seinen Weg zu jedem beliebigen Zeitpunkt in jeder mit der Zeit verkn"upften Dimension kennen k"onnen. Um eine Au"sage "uber das Ereignis als solches zu treffen ist dieses Verfahren nat"urlich total ungeeignet, da wir immer noch keine Au"sage "uber die Teilereigni"se im Meer des Chaos 
machen k"onnen aber f"ur eine Zeitreise ist dieser Faden durch das Chaos unabdingbar. Mit einfachen Worten ausgedr"uckt hei"st das wir w"u"sten zwar immer noch nicht ob ich morgen Tee trinken werde, aber wir k"onnten dann hingehen und nachsehen.
Um eine einfache Notation einzuf"uhren bedienen wir uns des Liapunow bzw. f"uhrenden Liapunow Exponenten. Die Liapunow Exponenten sind eine Menge von Zahlen, deren Summe gleich Null ist, als Reihe geschrieben mit monoton fallenden Elementen. Wir identifizieren sie mit den exponentiellen Lambdas aus unserer vorherigen Betrachtungsweise der Abweichung von unserem geplanten Weg.
Sie werden nat"urlich an dieser Stelle aufmerken, da"s es allein eine M"oglichkeit gibt einen Linearen Anstieg der Abweichung oder Ungenauigkeit zu erreichen, als die da w"are den f"uhrenden Liapunov Exponenten, und damit nat"urlich auch die folgenden, gleich Null zu setzen.
Um nun eine Zeitreise praktisch durchzuf"uhren identifizieren wir durch das Ph"onixflug Dimensionswegverfahren ein Element mit einem Liapunov Exponenten gleich Null. Danach bestimmen wir eine Abbildung durch de"sen Funktionswert nach dem von uns gew"unschten Zeitpunkt und erhalten eine wunderbar lineare Funktion. Wir Identifizieren den linearen Faktor als Tau und kompensieren ihn durch sein Inverses.

Die so entstehende Matrix dient uns f"ur die hermetische Anwendung als Weg zwischen den designierten Punkten in der Raumzeit. Dabei ist nat"urlich die Dimensionlatit"at der Abbildung von immenser Wichtigkeit und wir ben"otigen eine komplette weitere Einheit um dieses Verfahren n"aher zu bringen.

Stellen sie sich eine Karte eines beliebigen Landes vor. Es existiert eine gegebene Topologie mit Bergen, Wegen W"aldern und so weiter. Ferner seien auf dieser Karte zwei St"adte zu finden, die durch einen l"angeren Weg voneinander getrennt sind. Nun stellen sie sich weiter vor sie w"urden einen H"andler, einen klugen, halbwegs gebildeten Mann fragen, wie man am besten von Stadt A zu Stadt B k"ame.

Nun, dieser H"andler w"urde nun mehrere Faktoren in seine "uberlegung mit aufnehmen. Zum einen w"urde er die Distanz der St"adte absch"atzen. Danach w"urde er die Topologie mit in Betracht ziehen. Wenn die beiden St"adte nun durch einen Wald oder ein Gebirge getrennt w"aren, es keinen Weg g"abe, der das Gebirge "uberspannt, aber einen, der um es herum f"uhrte, dann w"urde der H"andler zu folgender Schlu"sfolgerung kommen. Obwohl der direkte Weg k"urzer ist, so w"urde es doch l"anger dauern ihn zu benutzen, da man auf der Stra"se wesentlich schneller reisen k"onne. Also w"urde er den l"angeren Weg nehmen, der ihn in k"urzerer Zeit ans Ziel bringt.
Zu den gleichen Schlu"sfolgerungen w"urde auch ein Bibliothekar kommen, der dies als theoretisches Modell verstehen w"urde. Der H"andler hingegen stoppt nicht an dieser Stelle, sondern w"urde weiter "uberlegen, welchen Weg er w"ahlen sollte. Nun w"urde er andere "au"sere Gesichtspunkte mit in diese Wertung einflie"sen la"sen. Ob es R"auber gebe, wie viele Herbergen es auf dem Weg gibt und so weiter.

Nun gebe ich die gleiche Karte einem Magier und frage ihn, wie er idealerweise vom einen Punkt zum anderen kommen w"urde. Dieser Magier w"urde die Karte nehmen und sie mittig in der direkten Distanz von Stadt A zu Stadt B falten. Dann w"urden beide St"adte nur marginal voneinander entfernt liegen, weil A und B identisch sind.

Dies bringt uns direkt zur nicht euklidischen Geometrie, die ich ein einem weiteren, f"ur uns relevanten Beispiel erl"autern m"ochte. Wie sie alle wi"sen ist die Winkelsumme eines Dreiecks gleich Pi, also einem Halbkreis. Will hei"sen, wenn ich 1000 Schritte in eine Richtung gehe, mich dann um das Drittel von Pi, also ein sechstel eines Vollkreises nach rechts drehe, wiederum 1000 Schritte gehe, mich wiederum um das Drittel von Pi nach rechts drehe und wiederum 1000 Schritte gehe und mich wiederum um das Drittel Pi nach rechts drehe, dann stehe ich am gleichen Punkt, an dem ich angefangen habe und blicke in die gleiche Richtung, in die ich am Anfang geblickt habe.
Das ist bekannt, das ist simpelste Geometrie.

Nun gehe ich einen Schritt weiter und beschreibe folgende Szenario. Der Bobachter steht an einem Punkt und geht 1000 Schritte in eine Richtung. Dann dreht er sich im rechten Winkel nach rechts und geht wiederum 1000 Schritte, nach denen er sich wiederum im rechten Winkel nach rechts dreht und wieder 1000 Schritte geht und im rechten Winkel nach rechts dreht. Nun ist er wieder am Ausgangspunkt und blickt in die gleiche Richtung. Wie ist das denn m"oglich, denn dieses Dreieck, das er beschrieben hat h"atte nun ja eine Winkelsumme von drei halben Pi.
Zudem behaupte ich, das dieser Beobachter, h"atte er zweitausend Schritte getan sich in einem beliebigen Winkel h"atte wenden k"onnen und w"are nach weiteren zweitausend Schritten ebenfalls wieder an seinem Ursprungspunkt angekommen.
Wie kann das sein?

Die Antwort ist ganz simpel. Unser Beobachter ist eine Ameise auf einem Ball von viertausend Ameisenschritten Umfang. Geht sie nun tausend Schritte ist sie von oben an die Seite des Balles gewandert. Dreht sie sich nun um das halbe Pi, dann l"auft sie die n"achsten tausend Schritte an der Seite des Balles entlang. Dreht sie sich nun wieder um das halbe Pi, dann erklimmt sie den Ball wieder und l"auft wieder zur Spitze. Wir haben also ein Dreieck mit einer Winkelsumme von 270 Grad.
Wenn sie zweitausend Schritte geht ist sie von oben nach unten gewandert und klebt nun unter dem Ball. Egal in welche Richtung sie sich nun wendet. Wenn sie von dort aus weitere Zweitausend Schritte tut findet sie sich wieder oben auf dem Ball wieder.

Diese topologischen Spielereien bringen uns zu der Art Abbildung, die unser Zauber bewerkstelligen mu"s, um einen Effekt zu bewirken, den wir im allgemeinen Sprachgebrauch als Teleportation beschreiben. Wir stellen uns nun die kleinste nicht-orientierbare Fl"ache von, die in der Literatur, heutzutage den Namen Kleinsche Flasche erworben hat.
Dies ist ein dreidimensionales Konstrukt, wie eine Flasche. Doch ihr Hals verl"angert sich „durchst"o"st“ die Seite der Flasche und bildet innen den Boden, indem sie sich wieder auseinander st"ulpt.
Dies ist eine zweidimensionale Fl"ache, ohne Rand und Orientierung. Wenn wir sie in h"oheren Dimensionen (mit einem Minimum von vier Dimensionen) einbetten, dann durchst"o"st der Hals eben nicht die Seite der Flasche.
Eine ebensolche Topologie brauchen wir auch, wenn wir eine transdimensionale Dislocation vornehmen wollen, nur, da"s wir derart drei Dimensionen haben, die wir abbilden. Aber zum Gl"uck steht uns ja mindestens der Astralraum mit allen seinen Dimensionen zur Verf"ugung.

Die Reise durch den Astralraum ist mit nichten Einfach und wenn sie den Weg verla"sen, den das Ph"onixflug Dimensionswegvehrfahren uns beschrieben hat, dann kann es zu nicht trivialen Beeintr"achtigungen kommen.

Wie wir bereits festgestellt haben liegt das Problem in der nicht-Orientierung unseres entsprechenden Volumens. Einzelne Punkte sind eben nicht Permutationsinvariant, sondern orientieren sich chaotisch. Dadurch kann unseren Reise im schlimmsten Falle eine chaotische Ver"anderung in allen beteiligten Dimensionen nach sich ziehen deren Tragweite wir nicht im Geringsten einsch"atzen k"onnen.
Zu dieser Problematik konsultieren sie bitte auch meine Schriften zum Amonlonde Parallelweltproblem 258 nach Jeldrik.

\vspace{5mm}

Als abschie"sende Bemerkung erlauben sie mir noch die Bemerkung, da"s ich Ihnen in ihren eigenen Anwendungen viel Erfolg w"unsche und hoffe einige Anregungen gegeben zu haben. Nat"urlich sehr an einem wi"senschaftlichen Disput intere"siert und w"urde mich sehr "uber Ihre Korrespondenz freuen.

\vspace{10mm}

Mit akademischen Gr"u"sen
Florian Phelleas Ph"onixflug
Magus Minor Akademia Clavis Mundi Grenzbrueckensis 258 nach Jeldrik

\newpage

\section{ Komponentensubstitution}

\yinipar{D}ie Verwendung von Komponenten oder Paraphernalia hat in der Zauberei sowohl eine lange Tradition als auch einen sehr praktischen Nutzen. Viele Spielarten der Magie von Ritualmagie bis hin zum Schamanimus sind ohne Komponenten schlicht unm"oglich, in dieser Abhandlung werde ich mich allerdings au"schlie"slich auf die Verwendung von Komponenten in der Hermetischen oder Thaumathurgischen Spruchmagie beschr"anken.
Die Spruchmagie benutzt vorher genau Ausgew"ahlte Abl"aufe um immer den gleichen Effekt hervorzurufen sprich standardisierte Abl"aufe, die es unserem Geist einfacher machen die Magie zu formen und in die gewollte Form zu pre"sen. Der Vorteil liegt auf der Hand, durch Spruchmagie wird es uns Magiern und Magierinnen erm"oglicht schnell und relativ Unkompliziert (zumindest im Gegensatz zu manch anderen Spielarten der Magie) zu beabsichtigten, uns vorher bekannten, Effekten zu gelangen. Der Nachteil besteht in der Einschr"ankung auf bestimmte Zauberformeln, Gesten und eben Komponenten. Formeln und Gesten variieren je nach Schule bzw. Lehrmeister und Kulturkreis, w"ahrend die Gesten oft von der Notwendigkeit bestimmter Bewegungen gepr"agt sind. Die Behauptung liegt also nahe, da"s Formeln, Gesten und Komponenten innerhalb bestimmter Toleranzgrenzen variiert werden k"onnen, wozu ich bereits einen empirischen Beweis erbracht habe.
Verehrte Kommilitonen la"sen sie mich das an einem Beispiel zun"achst n"aher erl"autern. Beim Zauber Windsto"s (aera elementum movo discrim) dem zweiten Zirkel Spruch aus der Via Pugna benutzt man Standardisiert die Schwungfedern drei verschiedener V"ogel um einen Lufthauch zu erzeugen, den man dann mit Hilfe des Spruches und der einem innewohnenden Kraft soweit steigert, das er zu einem Windsto"s wird, "ahnlich einem Stein der einen Abhang herunterollt und eine Steinlawine ausl"ost. Beschr"anken wir die Wirkungsweise der Federn auf ihr Minimum liegt ihr Sinn und Zweck einzig und allein darin besagter Stein des Ansto"ses zu sein, also die bewegte Luft herzustellen, die dann vervielf"altigt werden kann. Wieso also drei verschiedene Vogelfedern benutzen die zudem noch recht hochwertig (Schwungfedern) sind? Die Antwort liegt in der Struktur der Magie, besondere sprich hochwertige Gegenst"ande enthalten mehr magische Energie als allt"agliche, langweilige Gegenst"ande (dies wiederum ist ein Teil meiner 
Narrativums Vorlesung, ich bitte den geneigten Leser diese Behauptung vorerst als bewiesen anzusehen und zu ihrer weiteren Disku"sion meine andere Vorlesung zu h"oren) wodurch auch f"ur die Magie schon ein Stein vorhanden ist, der ins Rollen gebracht werden kann.
Beleuchten wir nun die beiden Extrema, will hei"sen um den Spruch noch leichter zu machen k"onnte ich Federn benutzen, die von sich schon einen hohen Teil von magischer Energie beinhalten (z.B. Ph"onixfedern wie ich es in meinem Experiment getan habe) und/oder sehr gro"se Federn, die mehr Wind machen nach dem Prinzip, je gr"o"ser der Stein des Ansto"ses desto wahrscheinlicher und gr"o"ser der Erdrutsch. Praktische Experimente haben bewiesen das dies in gewi"sen Schranken funktioniert. Zauber werden leichter auszuf"uhren, vergr"o"sern ihre Wirkung aber nur unwesentlich bis gar nicht.
Das andere Extrem w"are wohl ohne Komponenten zu zaubern, nach dem Prinzip „Luft ist sowieso immer in Bewegung, wieso sollte ich sie dann noch in Bewegung versetzen m"u"sen?“ und „in dem Anwender steck genug Magie, wieso sollte ich sie noch aus irgendwelchen Federn holen?“. Obwohl dies theoretisch m"oglich erscheint ist es mir noch nicht gelungen es Experimentell zu best"atigen, was in mir die Vermutung hegt, das die Verwendung von Komponenten nicht nur eine Tradition ist, sondern tats"achlich aus einer Notwendigkeit heraus erwachsen ist. Tats"achlich l"a"st dich die Wirkungsweise durch eine Wurzelkurve ann"ahern, was den schwachen Anstieg des Wirkungsgrades ab einem bestimmten Punkt erkl"art. Folgt man dieser Darstellungsweise w"are das Zaubern ohne Komponenten unm"oglich da die Funktion im Nullpunkt einen Wert von Null annimmt.
Worauf ich mich in dieser Vorlesung allerdings konzentrieren will ist die Komponenten Substitution, also die Vertauschung oder Auswechselung von Komponenten.
Um auf unser Anwendungsbeispiel zur"uckzukommen, ich kenne einige Magier u.a. auch Hochrangige Vertreter einiger Akademien, die besagten Windsto"s mit einem F"acher, anstatt mit Federn Zaubern und wenn wir uns auf unsere Minimalen Anforderungen zur"uckbesinnen k"onnen wir nur sagen wieso nicht? Denn wir brauchen lediglich einen Gegenstand, der Luft in Bewegung setzt und „etwas Besonderes“ ist, diese Beschreibung trifft auf den F"acher genauso zu wie auf die drei Schwungfedern von verschiedenen V"ogeln. Verallgemeinern wir diese Argument f"ur den Windsto"s hei"st das das man ihn mit jedem Gegenstand, der Wind machen kann und der „etwas besonderes ist“ herbeirufen kann. Erfolglose Versuche mit diversen K"uchenger"aten, Buchdeckeln und "ahnlichem haben allerdings gezeigt, das es noch eine weitere Anforderung an den Gegenstand gibt: Er mu"s dazu da sein Luft in Bewegung zu versetzen. So klappte der Zauber mit einem Blasebalg versagte aber kl"aglich mit einem Buchdeckel. (Obwohl bei dem Blasebalg der Wiederstand 
der Technisierung mir ein wenig zu schaffen machte, aber die ist ein Teil der Vorlesung "uber technische Magie und soll hier ebenfalls keine weitere Erw"ahnung finden.). Diese, dritte, neue und an und f"ur sich wichtigste Vorau"setzung hat ihre Grundlage in der Struktur der Zaubermatrix in der Wirkungsweise. Sich "ahnelnde Wirkungsmatrizen erleichtern die Wirkungsweise, ein an und f"ur sich bekanntes Prinzip, Stiefel erleichtern Bewegungsmagie, Schmuckst"ucke Beherrschungszauber usw.. Diese Strukturen sind allerdings jedem Hermetiker seit seiner Grundausbidung bekannt und werden zumeist selbstverst"andlich und ohne besondere Beachtung Ber"ucksichtigt.
Nun haben wir einen bestimmten Rahmen abgesteckt in dem wir unsere Komponenten von der Lernvorlage abweichend variieren k"onnen, allerdings sei an diesem Punkt darauf hingewiesen das dies die Unfallwahrscheinlichkeit und Fehleranf"alligkeit teils extrem steigert und au"schlie"slich erfahrenen Magiern und Magierinnen zu empfehlen ist.
Die K"onigsdisziplin der Komponentensubstitution ist das ersetzen der Komponenten durch Zauber. Zur Erkl"arung behelfe ich mich wiederum mit einem Beispiel. Der 1. Zirkel Spruch der Via Pugna „Waffe erhitzen“ (galad fulumbar) wird Standardm"a"sig mit einem Bregah"olzchen als Komponente ausgef"uhrt, indem man die Hitze der Flamme des H"olzchens auf die Zielwaffe projiziert. Um wieder unsere Betrachtungweise der minimalen Anforderung zu bem"uhen hei"st dies wir brachen lediglich ein Fl"ammchen, de"sen Hitze wir auf die Waffe lenken k"onnen. Nun jeder von ihnen meine Damen und Herren wei"s, wie man ein Fl"ammchen in seiner blo"sen Hand ohne gro"sartige Anstrengung entstehen l"a"st, der Lehrling"spruch Fulumbar auch Feuerfinger genannt.
Diese Annahme hat mich zu folgendem Experiment inspiriert, das ich bereits mehrfach erfolgreich durchgef"uhrt habe. Statt der Flamme des Bregah"olzchens benutzte ich die kleine Flamme des Feuerfinger als Komponente f"ur den galad fulumbar um die Hitze zu projizieren was sogar noch leichter und einfacher funktionierte als der urspr"ungliche Zauber. Da nun der Feuerfinger seinerseits keinerlei Komponenten erfordert konnte ich die beiden Zauber zu einem einzigen kombinieren, der f"ur sich als ganzes genommen keinerlei Komponenten ben"otigt, der fulumbar galad fulumbar. Zu den zus"atzlichen Kosten an Konzentration und magischen Energien f"ur den Feuerfinger k"onnen wir also g"anzlich auf alle Komponenten verzichten, oder be"ser ausgedr"uckt wir substituieren sie, in diesem Fall durch einen Zauber.

Diese Theorie ist sicherlich f"ur einige Zauber anwendbar und ich werde in n"achster Zeit noch einige konkrete Vorschl"age ver"offentlichen. Die Geschichte der Zauberei zeigt deutlich das bei der Entwicklung vieler Zauber bewu"st oder unbewu"st solch ein Wi"sen mit eingeflo"sen ist. Zum Beispiel der Zirkel 3 Spruch Feuerball l"a"st solche Strukturen erkennen. In der Fromel ballisto fulumbar vas perdo magia mortis ist wiederum das zweite Wort der altbekannte Zauber Fulumbar Feuerfinger. Um den Gedanken noch weiter zu spinnen ist das erste Wort der Formel Ballisto, wie in Aragh Ballisto, die magische Terminologie um ein magisches Gescho"s zu erzeugen. Also liegt die Vermutung nahe das in der Geschichte dieser Zauber aus der Verschmelzung verschiedener Zauber entstanden ist nun aber nur noch als eine Einheit gelehrt und ausgef"uhrt wird und nicht mehr wie der fulumbar galad fulumbar in seine Einzelzauber zerlegt werden kann.

Einen kleinen Exkurs in das Gebiet der Alchemie kann man ebenfalls f"uhren in dem man bestimmte magisch hoch potente Komponenten betrachtet, wie z.B. Drachenz"ahne und Schuppen, Ph"onixfedern und dergleichen. Da gerade diese Gegenst"ande schwer zu erhalten sind, sind ihre Auswirkungen auf die Zauberei recht unerforscht, doch im Allgemeinen kann man sagen das sie sowohl die Wirkung des Zaubers steigern, als auch daf"ur sorgen das sie leichter auszuf"uhren sind. Auch erm"oglichen sie die Grenzen der Komponentensubstitution, die ich weiter oben als relativ eng angegeben habe weit "uber das vermutete Ma"s hinaus zu strecken. Ich hoffe sie in n"achster Zeit mit neuen n"utzlichen Informationen bez"uglich dieses Themas beliefern zu k"onnen, doch diese Forschungen stehen gerade erst am Anfang.

\vspace{10mm}

Zusammenfa"send kann man sagen das uns die Komponentensubstitution viele neue M"oglichkeiten er"offnet und feststehende Grenzen etwas flexibler macht. Es ist keine revolution"are Erneuerung noch soll sie die Sorgfalt ersetzen die bei dem Studium von Spr"uchen und der Auswahl der Komponenten an den Tag gelegt werden soll. Es ist jedem Scolarius und jeder Scolaria angeraten hermetische Spr"uche erst so zu lernen wie sie im Lehrplan vorgesehen und von ihren Lehrmeistern als Sinnvoll erachtet wurden und sie erst wenn sie den Status eines Adepten oder einer Adepta erreicht haben eigenst"andig zu modifizieren. Es hat durchaus einen Sinn das die hermetischen Zauber seit Jahrtausenden gelehrt werden wie sie gelehrt werden, aufgeschrieben werden wie sie aufgeschrieben werden und gezaubert werden wie sie gezaubert werden doch sollen wir als eigenst"andige und intelligente Mitglieder der magischen Gemeinschaft diese althergebrachten Traditionen hinterfragen und logisch analysieren und sie nicht einfach aus stumpfem 
Obrigkeitsdenken und Unterw"urfigkeit heraus "ubernehmen, sondern weil wir ihren Nutzen erkannt haben und wir uns objektiv entschieden haben diese wirkungsvolle und Effektive Methode zu w"ahlen.

\vspace{10mm}

Florian Phelleas Ph"onixflug, Scolarius des Bundes zu Ayd`Owl im Jahre 254 nach Jeldrik

\newpage

\section{ Animagie}

\yinipar{D}ie Animagie ist eine Form oder Schule der Magie, wie Hermetik oder Elementarismus, sie begr"undet sich auf die Interaktion mit Geistern. Animagie ist sehr eng mit dem Schamanimus verkn"upft und in der Regel wirken Schamanen und Schamaninnen Spruchzauberei mit Hilfe von Animagie, allerdings sollten diese beiden Schulen nicht miteinander verwechselt werden, sind sie doch zwei f"ur sich gesehen selbstst"andige Zweige ohne notwendigen Bezug zueinander.

Die Kraftquelle der Animagie sind Geister, sowohl Elementar- und Naturgeister, als auch Ahnengeister Verstorbener und die Idealisierten Vorstellungen von Tieren also Tiergeister. In der animagischen Spruchmagie werden diese durch standardisierte Gesten, Formeln und Komponenten dazu gebracht einen bestimmten Effekt in der die"seitigen Welt zu verursachen, der unseren hermetischen Spr"uchen bis aufs Haar gleichen kann, aber nicht mu"s. Zum Beispiel ruft ein Animagier einen Windgeist herbei um den Wind zu entfachen und auf seinen Gegner zu werfen, die Wirkung ist identisch mit unserem Windsto"s. Was aber viel verbl"uffender ist, ist das sich der Zaubervorgang an sich ebenfalls sehr "ahnelt. Besagter Schamane bei dem ich diesen Zauber beobachten konnte hei"st Chop A und kommt aus Silvanaja, er wirkt diesen Zauber in dem er mit einem Fetisch, an dem sich unter anderem auch viele Federn befinden, in Richtung des Gegners wedelt und folgende Worte rezitiert: „Rad susak vad jak“ (rein phonetische Wiedergabe) die laut 
seiner Au"sage soviel hei"sen wie „komm Wind auf Feind“. Um nicht zu sehr in die Magietheorie abzugleiten werde ich an dieser Stelle nicht weiter auf Gemeinsamkeiten eingehen sondern "uberla"se dieses Thema einer zuk"unftigen Vorlesung. Wie der geneigte Leser aber dennoch entnehmen kann k"onnen viele bekannte Spruchzauber aus der Hermetik ebenfalls in gleicher Wirkungsweise von der Animagie nachgeahmt werden. Ich pers"onlich wurde bereits Zeuge von erfolgreichen Anwendungen "aquivalenter Zauber zu „mentaler Dolch“, „Magiespiegel“, „Furcht“ und einem Feuerball wobei erstere zwei wohl zu den kompliziertesten Spr"uchen z"ahlen die die hermetische Spruchzauberei zu bieten hat und somit die Vermutung nahe liegt das solch eine Nachahmung mit allen Spr"uchen m"oglich ist.

Die Ritualmagie der Animagier ist weitaus komplizierter, wie es in jeder Magierichtung der Fall ist und da ich bis jetzt noch nicht die Ehre hatte einem reinen Animagus bei einem Ritual beizuwohnen halten sich meine Kenntni"se dar"uber auch sehr in Grenzen. An diesem Punkt sei lediglich angemerkt das Animagie sich sehr of mit anderen Traditionen "uberschneidet, teilweise sogar mit Fachbereichen der Hermetik. Elementargeister, sind z.B. die Schnittstelle zwischen Animagie und Elementarismus, Naturgeister, die zum Druiden und Hexentum und zur allgemeinen Naturmagie. Ahnengeister schlie"slich schlie"sen die Br"ucke zur Nekromantie, einem Weg unserer Hermetik. Das hei"st das jede Tradition wohl ihren Weg hat in gewi"ser Weise durch Rituale Animagie zu wirkenk und mit den Geistern in Verbindung zu treten.

\vspace{10mm}

Florian Phelleas Ph"onixflug, Bund zu Ayd`Owl im Jahre 254 nach Jeldrik

\newpage


\chapter{ Condra}

\section{ Bericht "uber die Reise in die S"udlande von Condra.}

Sehr geehrter Prytanus Andariel Dagonett,

vor einigen Tagen beteiligte ich mich an einer Expedition der Sturmfalken in die Gebiete s"udlich von Axnom. Dabei ist ein Daimonisches Wesen aus einem Jahrhunderte alten Gef"angnis befreit worden und streift seit dem durch die Retekberge. Da Walpurga von Auenbruch offensichtlich zu einer der ehemaligen Wirte des Ythid Darathai geh"orte wollte ich euch zus"atzliche Informationen, additiv zu denen, die ihr sicher von den Sturmfalken erhalten werdet, zukommen la"sen. Au"serdem stehe ich euch nat"urlich gerne f"ur jedwede Fragen bez"uglich der magischen Analysis dieses Vorgangs zur Verf"ugung, sollten die der Armee nicht ausreichen.

Mit freundlichen Gr"u"sen
Florian Phelleas Ph"onixflug, Adeptus des Bundes zu Ayd Owl

\newpage

\textbf{Bericht "uber die Reise in die S"udlande von Condra.}\\

\yinipar{E}rste Woche des siebenten Mondes des Jahres 256 nach Jeldrik. Ihre Markgr"afliche Hoheit Jerevan von Arkenwald schlo"s sich mit seiner Frau Gemahlin und seinem Gefolge einer Expedition der Condrianischen Armee in die Lande s"udlich von Axnom in Condra an. Soweit ich mich erinnern kann war der Sinn der Expedition einen alten Posten der Armee, Sternwacht gehei"sen, wieder zu finden und neu zu bemannen, doch er sollte im Laufe der Dinge nicht weiter intere"sant werden.
Froh wieder in Condra zu sein und bald meine Freunde wiederzusehen geno"s ich die Reise, bis wir auf die erste Leiche am Wegesrand stie"sen. Es war ein Mann in fremde Kleidung geh"ullt, mit einem vermummten Gesicht, wie ich es bis jetzt nur in Samarkant gesehen habe. Er wurde mit einer Handvoll Heimaterde und einer Schriftrolle, die ein Gedicht enthielt, in einen Baum gelegt. Offensichtlich der Totenritus dieser Leute. Die Expedition reiste weiter und erreichte eine kleine H"utte, die an einem Felshang erbaut worden und offensichtlich das einzige "uberbleibsel der gesuchten Grenzwacht war. In dieser H"utte lebten zwei kleine M"adchen ganz f"ur sich alleine, die lediglich von einem eher tumben Dorftrottel besucht wurden, der ihnen Windspiele „gegen die b"osen Geister“ schenkte.
Wie wir nach einiger Resch"arsche herausfanden waren diese M"adchen die T"ochter eines m"achtigen Magiers und einer Hexe, die vor Jahrhunderten hier lebten. Der Magier widmete sein Lebenswerk dem Versuch den D"amonen Ythid Darathai zu bannen, was ihm schlie"slich auch gelang. Doch bei dem Bann verlor er sein Leben und kehrte nur noch als Geist jeden Abend zur"uck, um sich um seine beiden unsterblichen, alterslosen T"ochter zu k"ummern.
Der Mann hatte den D"amonen mit f"unf Ketten und einem Ritual in einen Felsen gebannt, da er eine Exvocation f"ur unm"oglich hielt. Als wir dort eintrafen fanden wir nur noch eine dieser Ketten aus einer Adamantium/Gold/Stahl Legierung vor, welche sp"ater allerdings ebenfalls abhanden kam. Die Turbantr"ager begehrten wohl ebenfalls diese Ketten und entf"uhrten zwei unserer Mitreisenden. Ein unvorsichtiger Krieger "ubergab ihnen dann unserer Kette im Austausch gegen das Leben der Geiseln. Eine andere Kette hatten sie wohl schon fr"uher in ihre H"ande gebracht, indem sie sie ein paar streunenden Goblins abgenommen hatten.
Diese Leute drangen unter Waffengewalt in das provisorisch errichtete Lager am Fu"se des Felsens ein und warfen einen Blick in den magischen Kerker des Ythid Darathai, wohl um sich zu "uberzeugen, da"s sie den Daimonen dort tats"achlich vorfanden und das die Ketten ihre Wirkung taten. Sie zogen nach der Sichtung des Daimonen voller Panik gepackt wieder von dannen, ohne, da"s wir sie angriffen.
Orks hatten sich den starken Nodice Astralis des Gef"angni"ses schon vor Wochen zu Nutze gemacht um ihm Kraft zu entziehen. Diese Kraft lenkten sie um und nutzten sie um Fr"uchte wachsen zu la"sen, in denen diese Magische Macht konzentriert war. Aus den Kernen dieser Fr"uchte lie"s sich ein hochpotentes Gift, gegen Astrale Wesen gewinnen, was wir auch taten um dem Daimonen entgegenzutreten.
Durch das Tagebuch des Magiers und den Ausk"unften der kleinen M"adchen erkannten wir nun das wahre Ausma"s des Problems und machten uns bereit ihm entgegenzutreten. Unter gro"sen Opfern gelang es uns die zwei verbleibenden Ketten in unseren Besitz zu bringen. Doch es erschienen zwei Geister, die sich uns als Schutzgeister der M"adchen vorstellte und unser Vertrauen gewannen. Doch in Wahrheit waren sie vom D"amonen geschickt worden um die Orks, oder be"ser gesagt deren Schamanen unter ihre Kontrolle zu bringen. Wir f"uhrten die Beiden Geister, weil wir es nicht be"ser wu"sten, zu den Orks und sie beherrschten deren Schamanen. Dann zogen die Orks gegen uns in den Kampf und befreiten Ythid Darathai aus seinem Gef"angnis. Wir mu"sten hilflos mitansehen, wie eines der beiden M"adchen get"otet und das andere Verst"ummelt wurde. Auch mehrere Mitstreiter von uns lie"sen ihr Leben auf diesem Schlachtfeld.
Geschlagen floh die Expedition aus den Landen S"udlich von Axnom zur"uck nach Condra und wir mu"sten zusehen wie der Daimon seine Freiheit wiedererrang.

\newpage


Anbei eine Abschrift des Tagebuches, des besagten Magiers:

\vspace{5mm}

Dies ist ein Traktat "uber den Einen, der „Verr"ater, „Bruderm"order“ und Ythid Darathai gehei"sen wird, eine d"amonische Wesenheit gro"ser Macht.
Der Bruderm"order erscheint nur selten in gestaltlicher Form eines von Knochen ummantelten Wesens mit einem entarteten Kopf, der einer widernat"urlichen Kreuzung aus Mensch und Insekt gleicht. Wenn er dies tut, so ist er jedoch von gro"ser Kraft und beinahe unbesiegbar, da es aus jeder gegen ihn aufgebrachten negativen Emotion oder Handlung neue Kraft beziehen kann.
Sein Vorgehen ist stets "ahnlich: Immer versucht er, die Kontrolle "uber eine Minorit"at aufzubauen, die er dann gegeneinander und gegen Freunde, Verwandte und Wohlt"ater zu hetzen. Aus dieser Ballung von Verrat, Mi"strauen und Ha"s zieht er seine Kraft.
Der Daimon erschien zum ersten Mal in dokumentierter Form auf einem Inselreich im Norden, einem Land namens Neka und dort in einem kleinen Dorf an der Grenze zweier „Provinzen“. Der Daimon trat in Form einer Stimme in Erscheinung „die einen jungen Mann mit falschen Versprechungen und daimonenhaften Kr"aften“ unter ihre Kontrolle brachte. Morde und Diebst"ahle waren an der Tagesordnung und bald schon konnte er seine Kr"afte auf die ganze Dorfgemeinschaft ausbreiten. Nachdem alle, die sich seinem Wirken in den Weg gestellt hatten aus demselben ger"aumt waren, begannen die Raubz"uge in der Umgebung. Hier wurden die Priester des dortigen Feuergottes und einige Magier der gr"o"sten Akademie auf das Ph"anomen aufmerksam, die das ganze Dorf mit Feuer und Schwert ausmerzten, nicht jedoch ohne eine Alnalyse der Kreatur vorzunehmen und ihre Pr"asenz in ganzen Land auszuradieren.
Ich spreche von „dokumentiert“ doch haben wir erfahren, da"s auch den Elfen und Zwergen ein Daimon mit "au"serst "ahnlicher Machart bekannt ist.
Die n"achste Sichtung fand in Jashra statt, einer Stadt in Betheuer. Dort beeinflu"ste er den dortigen Grafen dazu, sich zu einem Despoten aufzuschwingen, der nur durch einen Aufstand der B"urgerlichen gest"urzt wurde. Offenbahr war sich der Verr"ater nicht sicher, wie weit er gehen k"onnte. Ger"uchteweise wurde die Kreatur sowohl identifiziert, wie auch ausgeschaltet, durch einige Elfen, die sich dem Aufstand angeschlo"sen hatten.
All dies habe ich von meinem Meister erfahren, der einen Gro"steil seines Lebens in diesem Neka verbracht hat und als Schiffbr"uchiger die Gestaden Condra erreichte. Auch er hatte nie von diesem Land hier geh"ort. Bald war er als f"ahiger Kenner der magischen Kunst bekannt.
Nach ihrem R"uckschlag in Betheuer mu"ste sich der Bruderm"order anscheinend einige Jahre erholen und sich am Ha"s der Welt m"asten um wieder zu Kr"aften zu kommen. Sieben Jahre sp"ater, ich war inzwischen Lehrling meines Meisters geworden , startete der Daimon seinen bisher letzten Versuch der Korruption.
Die Akademie Condras, das Lied der Harmonie, war erst wenige Jahre zuvor aus einer Kooperative von Hexen und Druiden hervorgegenagen. Die Hexe Walpurga von Auenbruch war eine verlockende Beute. Soweit wir herausfanden war dies das erste Mal, da"s die Kreatur direkt von einer Person besitzt nahm. Entweder werden seine Kr"afte immer st"arker, oder er pa"st sein Vorgehen immer mehr der menschlichen Kultur an. "uber die Leiterin konnte er die Aktionen der jungen Akademie lenken und war durch die Magi be"ser gesch"utzt als in Betheuer. Mein Meister, der wohl schon l"anger auf der Jagt nach dem Bruderm"order ist, erkannte jedoch de"sen Wirken. Gemeinsam mit einigen abtr"unnigen Mitgliedern der Akademie gelang es uns, den Daimon in die Enge zu treiben. Es kam zum Kampf in den Mauern der Akademie als unsere kleine Gruppe bis in die Gem"acher von Auenbruch vordringen konnte, wo wir den Verr"ater stellten.
Die Magier sollten ihn besch"aftigen, w"ahrend mein Meister die Bese"sene von der verderbten Pr"asenz reinigen wollte. Meine Aufgabe war es, die Muster des Daimons zu erforschen, um so zu dem wichtigsten Schl"u"sel im Kampf gegen ihn zu kommen: Sein wahrer Name.
Doch irgendwie gelang es dem D"aimonen, sich aus dem K"orper der Hexe heraus zu manifestieren und uns anzugreifen. Mein Meister war das erste Opfer, doch mit letzter Kraft konnte er ein Portal in eine andere Sph"are "offnen, welches den K"orper des Verr"aters verschlang. Ich selbst kam mit einem verkr"uppelten Bein davon, doch drei unserer Helfer wurden ebenfalls fortgeri"sen. Das schlimmste Bild bot sich jedoch mir, da ich mich der Magia Clarobservantia astralis bediente: Auch wenn der K"orper der Kreatur vernichtet wurde, so weilt ihr verdorbener, wenn auch geschw"achter Geistk"orper immer noch in dieser Welt und wird sich bald neue Macht verschaffen wollen. Dieses Monster mu"s unter allen Umst"anden davon abgehalten werden diese Welt wieder zu verpesten, denn ich f"urchte, diesmal werden wir ihn nicht mehr so leicht aufhalten k"onnen. Aber bereite dich auf einen harten Kampf vor, Bruderm"order, denn ich kenne dich jetzt: (hier fanden wir seinen Namen)

\vspace{10mm}

So kam es, da"s der Bruderm"order einen Leib schuf, in dem gebunden sein faulender Geist weilen sollte.
So m"achtig dieser Leib auch sein mochte, so offenbahrte er doch auch zgleich die Schw"ache des Verdorbenen.
In Fleich gebunden war er verwundbar und konnte durch simple Formen der Magia Combativa attakiert werden.
Diese Wunden schlo"sen sich jedoch viele Male, immer schneller verheilte sein brennendes Fleisch. Aus diesem Wi"sen erwachte ein Plan in mir, um diesen schlimmsten unserer Feinde endg"ultig zu "uberwinden. Um dieses Ziel zu erreichen m"u"sen drei Dinge geschehen:
Ad Prima: Ythil-Darathai m"u"ste wieder in seinen fauligen Laib gezwungen werden.
Ad Secundo: M"u"ste sein verr"aterischer Geist von den Amni Astralis getrennt werden.
Ad Tertio: Sowohl sein Geist als auch sein Leib m"u"sen am Ort gehalten werden.

\vspace{10mm}

Alle mir bekannten Schurken der Magia Invocatia bedienen sich hierbei einer gro"sen Anzahl von Munstern, welche Portale und Br"ucken zwischen den Ph"ahren zu schlagen verm"ogen. Abgesehen von der gro"sen Gefahr, die diese fragw"urdige Methode in sich tr"agt und dem gro"sen Schaden, die dem Sph"ahrengef"uge durch diese Praxis zugef"ugt wird, w"are sie in diesem Fall auch nicht sonderlich erfolgreich. Der Verr"ater weilt schon seit langem in dieser Welt, so da"s eine Invocatio aus einer anderen Sph"are zum Scheitern verurteilt w"are.
Ihn zu rufen, wie es einige der fraglichen Hexen- und Druidenzirkel empfehlen birgt ebenfalls Gefahren, da wohl schon das Au"sprechen seines Namens seine Aufmerksamkeit erregt, wie es dem Bruderm"order auch einen Pfad in den Geist des armen Teufels "offnen mag, der unbedacht seinen Namen nennt.
So wir also den Verr"ater nicht auf dem uns bekannten Wege der Invocation rufen k"onnen, so plane ich, mich einer, wenn auch sicheren , so doch wesentlich aufw"andigeren Prozedur zu bedienen:
Der Convocatio Maioris!
Wenn ich den Einen nicht dazu zwingen kann zu erscheinen, so m"u"sen eben alle Alle gerufen werden, falls man dieser Kreatur habhaft werden will.
Bei diesem Ritus wird die Bereitung des Platzes die wohl gr"o"ste Aufmerksamkeit erfordern. So mu"s der Magier zun"achst einen Ort finden, der abseits genug von allen astralen Steigwellen liegt, jedoch einen direkten zugriff zum Intervallum Nodicum erm"oglicht. Die coherrenten Amni der topographischen Substanzschwelle m"u"sen exakt nach 3 entsprechen, wogegen keines der Str"ohmungswerte "uber dem Maximal Werten der Lucianischen Portoaethesica liegen darf. (Care!)
Nach der gr"undlichen Reinigung des Platzes kann mit der Arretierungder Amni begonnen werden.
Die systolischen Spannungswellen m"u"sen jedoch jederzeit die Option haben auf ihrer Maximalamplitude zu schwingen, um eine Anpa"sung von nicht mehr als 375 Omikronparametern pro Mondlauf "au"serst unwarscheinlich zu machen.
Wichtig sei hier zu erw"ahnen, da"s man einene der diametralen Amni zwar verst"arkt, jedoch nicht akkumuliert, da dieser zur sp"ateren Apesation des Ableitung"strohmes genutzt werden mu"s. Nach erfolgreichen Fixierung und Akkumulation kann mit dem Retiskonstrukt begonnen werden, welches dem Gef"uge angeglichen und mit dem Amnis und de"sen diastolischen Ladunswert syncronisiert werden mu"s.
Sobald die Syncronisierung begonnen wurde, m"u"sen wir uns dem zentralen Konstrukt zuwenden, reicht es schlie"slich nicht, die Kreatur nur zu rufen. Um jedoch die Matrizen der Fixation in die mannigfaltigen differenten Mani zu integrieren, mu"s zun"achst ein zentraler, am besten leicht erh"ohter Fokuspunkt gefunden werden, an dem die Amni non cogerient werden.
Hier kommt nun das Detis oder Netzkonstrukt ins Spiel, welches mit Hilfe der metaphysischen nominalen E"senzkomponenten, dem „wahren Namen“, des gesuchten Astralleibs der Ma"se der Convocierten heraus zu binden vermag. Als hilfreich zeigt sich ein Fokus, der auch in unserer Sph"are bindende Kr"afte h"atte. Ich habe mich f"ur ein paar Ketten entschieden, welche die vier Gliedma"sen und den Hals der Kreatur binden werden.
Es wurde eine 15:4:1 Legierung aus Stahl, Gold und Adamatium verwendet, welche ich mir von den Zwergen in Axnom fertigen lie"s.
Nach der Artefaktbindung von Augustus Runethal mu"s die E"senzkomponente (Der „Name“) in die bindenden Foki gepr"agt werden, da ein astraler Leib in jedem Fall von seiner eigenen Struktur gehalten werden kann.. Auch sollte in diesem Arbeit"schritt die Kommunikationsverbindung zum immigrierenden Amnus geschlagen werden, verbunden mit einer Matrix, die de"sen Aktivit"at einstellt, sobald die Ketten eine posititve Resonanz abgeben. (Verdammt jetzt mu"s ich auch noch die Tinte wechseln!)
Danach mu"s der Magier eine manuelle Abschaltung vornehmen, welche den emmigrierenden Amnus betrifft, da sonst m"oglicherweise noch andere Dinge in der N"ahe des Fokus an das Gef"uge der Gegend gebunden werden k"onnten.
Durch die Trennung wird jeder innerhalb des Kreises vom astralen Raum abgeschnitten und der D"amon gefangen. Au"serdem werden durch das Zusammenspiel der Ketten und des Astralkonstruktes (s.u.) die Heilungskr"afte der Kreatur eingeschr"ankt. Ich bin zwar nich sicher, ob man genug arkane Macht oder Waffengewalt aufbringen kann um ihn zu zerst"oren, aber gottseidank wird das nicht n"otig sein, sobald das Konstrukt seine Arbeit tut.
Die Ketten sollten sowohl astral als auch physisch in einer Weise angebracht werden, da"s sie in der Lage sind da Wesen zu halten, wobei sich der Astrale Aspekt mit dem Gewebe von selbst verbinden sollte.
So der Bruderm"order also im Fokus Amnium, welches in Form eines Schutzkreises von ausreichender Gr"o"se vorliegt sollte gefangen ist, kann mit dem zweiten Teil der Vernichtung begonnen werden.
Es scheint hilfreich einen Kreis zu w"ahlen, der tats"achlich k"orperliche, wenn m"oglich nat"urliche Komponenten enth"alt, da dies die Bindung noch weiter verst"arkt.
Doch wollen wir nun zur schwierigsten Komponente kommen:
Ich habe seit langem einen weg gesucht, den Geist des Verr"aters zu zerst"oren ohne seinen K"orper zu zerst"oren und den Geist damit wieder freizula"sen. Jetzt endlich habe ich eine M"oglichkeit gefunden. Man mu"s ihm die Gelegenheit geben sich selbst zu vernichten!
Eine implodierende, durch temporale und astrale Reizung ausgel"oste Astraltasche, welche gleichzeitig als Leiter fungiert! Jede Kraft, die die Kreatur im Inneren anwendet f"uhrt dazu, da"s sich das Gef"uge doppelt proportinal zur eingesetzten Kraft schlie"st und zusammenzieht, und gleichzeitig jegliche Energie aus dem inneren harmlos ableitet. Je mehr sich die Bestie als wehrt, desto eher wird sie sich selber ausl"oschen!
Um aber sicherzugehen, da"s Ythid Darathai auch wirklich vernichtet wird, wird das Konstrukt auch von alleine enger, was den letzten und schwierigsten Teil der Herstellung darstellt.
Sind alle vorbereiteten Ma"snahmen vollendet, so mu"s der Fokus Amnium an der Stelle der Fe"selung plaziert werden. Der Magus steht in einem kleineren Kreis, welcher mit dem Pentagramma gezeichnet ist und sich zwei und einen halben Schritt vor dem Siegel befindet.
Die Initiation des Ritus ist nun recht einfach, mu"s doch nur der convocative Kraftflu"s gerufen werden und zwar mit folgenden Worten:
Alii aliam in partem feruntur!
Der zweite Annus wird aktiviert mit den Worten:
Spititi timebant, ne circuirentur!
Alles was die Str"ome nun noch hemmt, ist der Fokus selbst und mit folgenden Worten kann endlich das Ende des Bruderm"orders eingel"autet werden:
Portus Annis arcaniterem odenbar 

\newpage

\section{ Von Miasaris Toreville}

\yinipar{M}iasaris Toreville war eine Adepta der Akademie von Nektor. Sie hatte ebenfalls den Kampf gegen die Schatten Grenzbruecks zu ihrem eigenen gemacht und ist im Sp"atsommer 257 nach Jeldrik bei der Errichtung des Walles gestorben.
Statt eine Laudatio zu halten m"ochte ich hier, an dieser Stelle Ihre akademischen Werke ver"offentlichen, da sie zu Lebzeiten nie die Gelegenheit dazu hatte es selber zu tun. Es ist eine Abhandlung "uber die V"olker Condras und eine "uber die Sprache der Orks in Grenzbrueck. Letztere hatte sie in Grenzbrueck stets mit einer Leidenschaft verfolgt, wie ich es selbst bei Sprachwi"senschaftlern selten gesehen habe. Ich hoffe sie ruht in Frieden und das wir sie durch ihre Schriften auf ewig im Ged"achni"s behalten werden.

\subsection{ "uber die V"olker und Wesen Condras}

Condra ist, vergleicht man es mit den K"onigreichen der Mittellande, Engonien oder Neka, ein kleines Land, in wenigen Tage durchquert man es zu Pferde, vom sumpfigen Norden zu den Retekbergen, von den Elfenw"aldern im Westen zur Steilk"uste im Osten. Nichtdestotrotz leben auch hier eine Vielzahl der verschiedensten Wesen, die ich, Miyasaris Thoreville aus Nektor, kurz vorstellen will. 

\newpage

\subsubsection{ Die Menschen}

\yinipar{D}as zahlenm"a"sig gr"o"ste der kulturschaffenden V"olker Condras (die meisten Gelehrten gehen der Frage, inwiefern man Orks zu diesen z"ahlen kann, wohl absichtlich aus dem Weg – mehr dazu sp"ater). Nach vorsichtigen Sch"atzungen sind weit "uber 90 pro centum der Lebewesen Condras menschlich, das entspricht in absoluten Zahlen etwa 40.000 K"opfen – mangels eines allgemeinen Zensus seit der Nekanerzeit. Da sich zur Siedlungsgeschichte in der fr"uhen vornekanischen Zeit keine Quellen mehr finden la"sen, die uns verr"aten, ob es eine urt"umliche Bev"olkerung dieser rauhen Lande gab, bin ich hier auf eine Extrapolation meiner Forschungen zur Sprachenverbreitung angewiesen. Zuallervorderst sei auf das Faktum hingewiesen, da"s Condra und Betheuer wie auch andere L"ander der weiter entfernten Mittellande sich eine Sprache teilen. Meine Vermutung ist, da"s diese Sprache mit den dranaischen Fl"uchtlingen aus Neka hierher kam. In Betheuer gibt es sogar schriftliche Quellen, die die These, 'Condrianisch' w"
are die Sprache der Nekaner gewesen, unterst"utzen. M"ogen die patriotischen Zweifler auch jede Behauptung, wir w"aren mit den Feueranbetern verwandt, weit von sich weisen, so bleibe ich doch dabei. Denn auch das Gegenargument, Neka h"atte doch fr"uher “Altnekanisch” gesprochen (uns als Sprache der steinernen Inschriften und pomp"osen Schriftst"ucke bekannt) und w"are erst durch Reisende aus den Mittellanden zu einer anderen Sprache gekommen, halte ich f"ur Unfug. Ein auf die eigene Kultur bis zum Fanatismus stolzes Land wie Neka, das niemals Einwanderung kannte, soll von Fremden etwas "ubernommen haben? Lachhaft! Aber ich schweife ab.
Lokale Dialekte haben sich in dieser kurzen Zeit nicht herausgebildet, vereinzelt h"ort man noch Ankl"ange an die singende Sprechweise der Alineser und unter der Landbev"olkerung vor allem der abgelegenen Wei"squelld"orfer hat eine gewi"se Verschleifung stattgefunden, die in der Gegend um Widdau am st"arksten zu sp"uren ist. Auch regionale Unterschiede im Gem"ut und Alltagskulture sind (noch) nicht festzustellen. Der s"udliche Condrianer ist, wohl durch die in der Vergangenheit h"aufigen Orkangriffe und ein Leben in st"andiger Verteidigungsbereitschaft, wortkarger und verschlo"sener, der n"ordliche Condrianer der Wei"sspitzen und Venne ist eigenbr"otlerisch aber herzlich.
Die Statur ist gleich, m"annliche Condrianer werden im Durchschnitt 175 Zentimeter gro"s, Frauen 165 Zentimeter (Stadtbewohner gr"o"ser). Im Durchschnitt werden Stadtbewohner 70 Jahre alt, doch 80 pro centum aller Menschen leben als Bauern und Handwerker auf dem Land, der Rest verteilt sich auf die St"adte Tharemis, Schieferbruch und Nektor. Das Hauptnahrungsmittel ist Getreide (Hafer, Weizen, Gerste) vor allem aber auch Erdknollen aller Art, die der Witterung entsprechend be"ser wachsen und im Winter eingekellert werden k"onnen. Da Condra keine nennenswerten Bodensch"atze sein Eigen nennt (einige kleinere Vorkommen in den Wei"sspitzen, die Minen der Retekberge sind fest in Zwergenhand), mu"s Metall importiert werden – was sich auch in der Au"stattung der “Armee” niederschl"agt, die nur dicke Tuchr"ustungen und einige wenige Kettenhemden aufbieten kann. Plattenpanzer sind teure Importware aus Betheuer oder den Mittelanden. Der technische Stand Condras ist der, da"s Buchdruck, Uhrwerke, Kompa"s und andere 
Linsensysteme bereits Verbreitung gefunden haben, es aber  immer noch eine kleine Sensation war, als Sendrian Schneeloh vor einem Jahrzehnt in Tharemis eine Walkm"uhle f"ur Wolltuche er"offnete. Die meisten Condrianer leben ein einfaches Leben, angepa"st an die Jahreszeiten und stark lokal gebunden, doch - nicht zuletzt verst"arkt duch den Hydracorglauben – sind sie freiheitsliebend, aufrecht, wenn auch nicht immer wahrheitsliebend und vor allem von einer schroffen Direktheit, die in H"oflichkeitskulturen wie der Betheuers undenkbar w"are.
Der nekanische Versuch, Condra ein hierarchisches System aufzudr"ucken, kann auch in dieser Hinsicht als gescheitert angesehen werden. Condrianer folgen nur denen, die etwas f"ur sie tun und die sie respektieren k"onnen. Solchen Respekt kann die Hydracorkirche einfordern, denn der Hydracorglaube ist de facto die Staatsreligion. "uber das F"ur und Wider einen Personalunion von Regierung und geistigen Herrschaft werde ich mich hier nicht ausla"sen, nur soviel, da"s der Klerus sehr viele F"aden in der Hand h"alt und auch, wie ich erfahren mu"ste, viel Wi"sen unter Verschlag. 
Hydracor ist Gott des Wa"sers, Wandels und der Welt in ihrer wilden Form, daher ist es nicht verwunderlich, da"s Condra sein Augenmerk niemals auf die Bezwingung der Natur richtete wie Neka mit seinen Monumentalbauten. F"ur eine genauere Beschreibung dieser Glaubenslehre verweise ich auf die Schriften des nekanischen Forschers und gesch"atzten Kollegen Jakobius Sonnau. 

Nachtrag: Mir ist zu Ohren gekommen, da"s es im tiefen S"uden, noch unterhalb der Orklande, in w"armeren Gefilden, ein Eingeborenenvolk geben soll, das helle Kleidung tr"agt und eine andere Sprache spricht. Aber ob dieser Menschenstamm ebenfalls eingewandert ist oder nicht, kann ich mangels Informationen nicht sagen. 

\newpage

\subsubsection{ Orks}

\yinipar{I}n Condra waren einmal zwei gro"se Orkst"amme beheimatet: Die Gr"unorks und die Braunorks.  Nachdem die Gr"unorks jedoch aus dem Kernland vertrieben worden waren und von den Braunorks 
aufgerieben wurden, finden sich nur noch wenige Gr"une, die meist in kleinen Gruppen durch die W"alder schweifen oder sich bei Menschen als billige Arbeitskr"afte verdingen. Die Braunorks jedoch sind umso gef"ahrlicher geworden. Zwar haben sie nicht, wie damals in den Orkkriegen, den gesamten S"uden "uberrannt, aber bis vor kurzem waren die S"udrouten und alles Land unter den Retekbergen orkverseucht. Geholfen hat ihnen bei der Verw"ustung des S"udens, da"s ihre Schamanen, die einzigen die Magie verwenden k"onnen, niedere D"amonen unter ihre Kontrolle brachten. Diese D"amonen werden wohl in Blutritualen mit ordin"aren Braunorks verschmolzen, woraus dann die monstr"osen Rotorks entstehen, deren Geist durch das Ritual zerst"ort, zu w"utenden Berserkern werden, die auch unter ihren Orkkumpanen eine Spur der Zerst"orung hinterla"sen. Im Zustand der Raserei sind sie nur noch durch erfahrene Schamanen zu beherrschen. Sie sind erkennbar an der roten F"arbung ihrer lederartigen Haut, ihren breiten Schultern, ihrer 
Unf"ahigkeit zu sprechen und ihrer f"urchterlichen Kraft. Im Vergleich mit ordin"aren Orks k"ummern sie sich noch weniger um Kleidung, soda"s sie meist nur den ungegerbten Pelz von erlegten Wildtieren am Leib tragen. Gr"une und braune Orks kommen in allen Gr"o"sen und Konstitutionen, die meisten sind allerdings st"ammiger und st"arker als durchschnittliche Menschen. Sie sterben fr"uher, meist bereits nach 20-30 Jahren. Orks ern"ahren sich gr"o"stenteils von Fleisch, dazu von Milch ihrer Viehherden und von wild wachsenden Pflanzen. Ihre Kleidung besteht aus Leder, Pelz und groben Stoffen, Schmuck aus Knochen, Federn und anderen tierischen (und humanoiden) K"orperteilen. Zivilisiertere Gr"unorks bevorzugen allerdings menschliche Kleidung. Ger"ustet sind sie meist entweder mit kr"uden Ansammlungen von Metall, gefundenen menschengefertigten R"ustungen oder Naturmaterialien. Ausgeglichen wird dies, da"s ihre Haut dicker ist als die der Menschen. Ihre Behausungen bestehen bestenfalls aus Erdh"utten und Lederzelten.
 Sie haben eine eigene Sprache, die aus Knurr-, Plosiv und Kehllauten besteht. Eine formelle Schrift kennen nur die Schamanen und einige H"auptlinge. Ger"uchteweise sind die condrianischen Orks degenerierte Nachfahren eines einstmals kulturschaffenden Volkes, das sogar m"achtige Bauwerke erschaffte und eine Bilderschrift sein Eigen nannte, doch sind die Quellen nicht zuverl"a"sig – eine Gruppe Abenteurer, die mir von einem Zusammentreffen mit einem solchen zivilisierten Ork erz"ahlte, sehe ich nicht als Beweis, vor allem angesichts der Tatsache, da"s es keinerlei Grabungsfunde gibt. 
Ein humanoides Volk, von dem ich nicht wei"s, ob es einfach nur orkische Mi"sgeburten sind, sind kleine gr"unh"autige Wesen, mit langen, zur Seite h"angenden Ohren. Sie kommen nur im S"uden vor, werden "au"serst selten gesichtet, daher bin ich inzwischen der Meinung, da"s sie ebenfalls nicht aus Condra stammen. Es soll hier ja auch Feen und Kobolde geben, aber ich habe in meinem ganzen Leben noch keinen zu Gesicht bekommen. 

\newpage

\subsubsection{ Zwerge}

\yinipar{E}ines der alten V"olker, das sich nicht lange nach Beginn der Nekanerherrschaft aus den menschlichen Angelegenheiten zur"uckgezogen und in seinem Bergk"onigreich Axnom abgeschottet hat. Zur selben Zeit wurden auch die letzten Minen in den Wei"sspitzen aufgegeben, die, wie der Hohe Rat durch Untersuchungen herausfand, allerdings auch ersch"opft sind. Wie man erkennen kann, sind die Zwerge ein Volk, das ber"uhmt ist f"ur seine handwerklichen F"ahigkeiten. Vor allem in Metallverarbeitung, Bergbau und Feinmechanik (aber auch in der Braukunst) sind ihre Werke ohnesgleichen. Ihre Verbundenheit mit dem Erdboden "au"sert sich darin, da"s sie Wohnh"ohlen oberirdischen H"ausern vorziehen und da"s auch ihre Nahrung subterrane Pflanzen umfa"st.  Zwergische Kleidung unterstreicht ihre Kunst: Harte Formen, die Farben von Stein und Metall, geometrische Muster und Metallst"ucke als Verzierung. Zwerge erreichen eine maximale K"orpergr"o"se von 165 Zentimeter, sind korpulent und kr"aftig gebaut. Ihr gro"ser Stolz 
sind ihre langen B"arte, die bei m"annlichen Zwergen eine stattliche L"ange erreichen k"onnen. Viele, aber nicht alle weiblichen Zwerge haben einen Kinnbart. Beide Geschlechter tragen kunstfertig geflochtene Z"opfe.  Ihre Sprache ist die der Menschen, ob es  einmal eine andere, alte Sprache gegeben hat, ist unbekannt, doch ihre Schrift ist eine andere. Es ist, wie ich herausgefunden habe, eine phonographische Buchstabenschrift, die eckige Runen verwendet, die leicht in Stein zu mei"seln und in Metall zu gravieren sind. Zu den Runen, die Laute repr"asentieren, kommen noch spezielle Zeichen, denn jeder zwergische Titel hat eine Rune, genauso wie die einzelnen Bingen – Reste einer Bildschrift? Die Gesellschaftsform ist stark hierarchisch, mit der K"onigsfamilie an der Spitze, den Meisterschmieden und Meisterbergbauern als Elite. Dazu gibt es noch die Priester der Schmiedegottheit Fulgor, die einzigen, die imstande sind, etwas zu verwenden, was wir Menschen „ klerikale Magie“ nennen w"urden. Das restliche 
Zwergenvolk ist so magisch wie – ein Stein. 

\newpage

\subsubsection{ Elfen}

\yinipar{N}icht viel ist bekannt "uber dieses alte Volk, da sich bereits vor der Nekanerherrschaft die Elfen vollst"andig aus der Menschenwelt zur"uckzogen. Und auch jetzt, da die n"ordlichen Handelsrouten wieder ge"offnet sind, ist der Kontakt eher sp"arlich. Einige wenige junge Elfen haben die W"alder verla"sen und leben unter den Menschen. Was sich "uber die Elfen sagen l"a"st, ist, da"s sie  hochgewachsene Wesen, von schlanker Gestalt, ohne Gesichts- und K"orperbehaarung und mit leicht l"angeren, spitz zulaufenden Ohren sind. Sie sind Meister in der Herstellung feiner leichter Stoffe und Schmuckst"ucke aus weichen Metallen mit organischen Formen.  Ihre Affinit"at zu Magie scheint um einiges h"oher als die der Menschen zu sein, denn unter ihnen gibt es nur wenige, die sie nicht zu gebrauchen scheinen. Elfen altern sehr langsam und ihre Jahre sind ihnen auch nicht anzusehen. Eine eigene Schrift und eine wohlklingende Sprache haben sie, doch hatte ich bis jetzt keine Gelegenheit, mehr als ein paar Wort zu 
erlernen, da ich noch nie einer Elfe begegnet bin, sondern nur Fuhrleuten, die von ihnen durch den Wald geleitet wurden. "uber ihre Geschichte konnte ich ebenfalls nichts in Erfahrung bringen, doch scheinen sie wie die Zwerge bereits vor den Menschen in diesen Landen gelebt haben, da es auch in Betheuer Reste einer elfischen, ebenso naturverbundenen Zivilisation gibt. 

\newpage

\subsubsection{ Nazgash}

\yinipar{A}uch wenn einige meiner gesch"atzten Kollegen abstreiten, da"s diese aufrecht laufenden menschengro"sen Echsen mehr sind als Tiere, mu"s ich sie doch erw"ahnen, da sie meiner Ansicht nach zumindest ein paar Anzeichen einer Kultur haben. Erstens tragen sie Kleidung als Schutz gegen die Sonne, die von Menschen gestohlen zu sein scheint, zweitens verwenden sie rudiment"are Werkzeuge aus Holz und Stein sowie menschengefertigte Waffen und drittens scheinen sie zumindest rudiment"ar die menschliche Sprache zu verstehen. Ihre eigene Sprache ist allerdings f"ur uns Menschen absolut unverst"andlich, sie besteht nur aus Zischlauten. Handwerkliche F"ahigkeiten, die "uber den Bau von Nestern und groben Schuhwerk hinausgehen, scheinen sie nicht zu kennen. Wie andere Reptilien legen sie Eier, allerdings wenige pro Weibchen, die danach im Stammesverband um jeden Preis bewacht werden. 'Angef"uhrt' werden die Verb"ande von den "altesten Weibchen, da die M"annchen eine k"urzere Leben"spanne haben. Ihre Lebensr"aume 
sind die S"umpfe der Fenne im Norden und Westen. Den Gebrauch des Feuers scheinen sie nicht zu kennen, daher ist es zumindest mir ein R"atsel, wie sie in der winterlichen K"alte der S"umpfe "uberleben. Sie sind sehr scheu und bevorzugen den Aufenthalt im Schatten, daher wird man selten einen Nazgash zu Gesicht kriegen. Die Farbe ihrer Schuppen geht von gelbgr"un bis zu braungr"un, ob dies ein Zeichen einer Alterung ist, ist mir unbekannt. Ihre Nahrung besteht aus Pflanzen, aber vor allem Fischen, kleineren Reptilien und Insekten. 

\newpage

\subsubsection{ Drow, Trolle, Kender und andere seltsame Wesen}

Vorweg der Hinweis, da"s alle vorgenannten Wesen nicht urspr"unglich in Condra vorkommen. 

Wilde Trolle, gro"s wie zwei Menschen, wandern manchmal aus dem S"uden ein, aber die letzte Sichtung war von 120 Jahren, auch wenn Reisende schon mal Trolle als Lasttiere mitbrachten. 

Drow kommen aus dem einen weiten H"ohlensystem in Betheuer, wo die Armee selbst mit Hilfe der Sturmfalken nicht eindringen konnte. Zum Gl"uck f"ur die Menschen bleiben diese dunklen Elfen – schwarze Haut, wei"se Haare - meist unterirdisch, da sie Tageslicht nur unter Drogen und unter Schleiern ertragen. Wenn sie an die Oberfl"ache kommen, sind sie grausame, hinterlistige Wesen, den Menschen feindlich und den Elfen ha"serf"ullt gegen"uber, daher werden sie gejagt, wo immer man sie in Betheuer sieht, doch wehe den Menschen, die von ihnen gefangen und versklavt werden. Ihre Gesellschaft ist matriarchalisch, mit Priesterinnen ihrer Spinnengottheit als Elite. "uber ihre Lebensweise und Kulturf"ahigkeiten ist nicht viel bekannt, sie scheinen zumindest Waffen und dunkel gef"arbte Tuche selbstt"atig herzustellen, ihre Gifte sind schrecklich und ihre Magie ist so stark wie die ihrer lichten Verwandten, der Elfen. Ihre Sprache ist h"arter als Elfisch, aber leider hat noch niemand sie erforschen k"onnen. 

Kender, Hobbits, Kobolde und Feenwesen sind in Condra nur als Durchreisende und fremdartige Besucher zu finden. Kender und Hobbits sind kleine spitzohrige Menschen"ahnliche. W"ahrend die ersten die Gestalt von schlanken, aber kleinw"uchsigen Menschen haben, sind die letzteren eher rundlich, mit lockigen Haaren und pelzigen F"u"sen. Kender haben die unschuldige Neugier und das Staunen von Kindern, Hobbits einen uners"attlichen Appetit. Beide V"olker sind nicht daf"ur bekannt, Magie zu kennen, sind aber daf"ur friedfertige, fr"ohliche Wesen. 
Im Gegensatz dazu kann das, was man Kobolde nennt, laut den Sagen fiese Gesellen sein. Die Magie dieser buntgekleideten gr"unh"autige Spitzohren ist von einer anderen Art als die der Menschen, daher k"onnen sie als Feen im Gegensatz zu menschlichen Magiern einfach zwischen ihrer eigenen Welt und der der Menschen hin- und herwechseln. Andere Feenwesen kommen in allen m"oglichen Formen und Farben vor, daher ist  eine allgemeine Au"sage "uber sie nicht m"oglich, doch im Gegensatz zu den Kobolden verla"sen sie ihr Reich nur unter ganz besonderen Umst"anden. Ich zumindest habe in meinem gesamten Leben noch nie eine Fee oder einen Kobold gesehen, daher w"are ich geneigt, sie ins Reich der M"archen zu verbannen, wenn meine Freunde mir nicht als Augenzeugen von Treffen mit solchen erz"ahlt h"atten. 

\vspace{10mm}

Miyasaris Toreville

\newpage

\chapter{Grenzbrueck und die Schatten}

\yinipar{I}n Grenzbrueck tobt seit Jahren der Krieg gegen die Schatten. Ich wurde bereits in den Jahren meines Adeptentums an der Akademia zu Ayd'Owl in diesen Konflikt verwickelt und seit dem immer tiefer darin verstrickt. Ich habe viele Bekannte und Freunde an Morbus verlohren und auch mehr als einmal habe ich selbst an der Schwelle des Todes oder noch Schlimmerem gestanden. Hier ver"offentliche ich das Wi"sen, da"s wir in den Jahren gesammelt haben, um es nachfolgenden Generationen zu erhalten sollte ich einmal nicht mehr in der Lage sein es zu teilen.


\section{ Von Miasaris Toreville}

\yinipar{M}iasaris Toreville war eine Adepta der Akademie von Nektor. Sie hatte ebenfalls den Kampf gegen die Schatten Grenzbruecks zu ihrem eigenen gemacht und ist im Sp"atsommer 257 nach Jeldrik bei der Errichtung des Walles gestorben.
Statt eine Laudatio zu halten m"ochte ich hier, an dieser Stelle Ihre akademischen Werke ver"offentlichen, da sie zu Lebzeiten nie die Gelegenheit dazu hatte es selber zu tun. Es ist eine Abhandlung "uber die V"olker Condras und eine "uber die Sprache der Orks in Grenzbrueck. Letztere hatte sie in Grenzbrueck stets mit einer Leidenschaft verfolgt, wie ich es selbst bei Sprachwi"senschaftlern selten gesehen habe. Ich hoffe sie ruht in Frieden und das wir sie durch ihre Schriften auf ewig im Ged"achni"s behalten werden.

\newpage

\subsection{ Grenzbr"ucker Orkisch – Die ersten Worte}

\yinipar{D}iese kurze Liste entstand w"ahrend meiner letzten Reise nach Grenzbr"uck. Wieder einmal hatten wir Morbus Pl"ane vereitelt und auch wenn ich beinahe dabei gestorben w"are, als mich der Fluch seiner Dienerin Lechata, der Frau ohne Gesicht traf, war es doch wert, denn zum ersten Mal hatte ich einen – tempor"ar friedfertigen - Ork namens Tullamok gefunden, der bereit war, mir einige Worte beizubringen. Ich bedaure zutiefst, da"s dieser mein Lehrer nach nicht einmal einer Stunde von Kriegern eines feindlichen Braunhautstammes umgebracht wurde und ich ihn nicht retten konnte. Die Orks, die wir davor getroffen hatten, waren entweder die von Morbus pervertierten Lok Ashtar gewesen oder hatten uns angegriffen und waren von den menschlichen Verteidigern sofort niedergemacht worden. 

Sobald sich mir die Gelegenheit erneut bietet werde ich diese Liste hoffentlich zu einem kleinen Wortschatz ausarbeiten k"onnen. Die F"ahigkeit, mit den Orks zu kommunizieren, darf nicht zu gering gesch"atzt werden, hat sie doch schon einmal Leben gerettet. Sei es, um mit den Orks zu verhandeln, sie und Morbus zu entzweien oder sie zu belauschen – die Kenntnis  ihrer Sprache wird ein Schritt auf dem Weg zur Vernichtung des "ubels der Schattenlande sein.

Miyasaris Thoreville, Tharemis im Jahre 5 n.d.B. (entspricht 256 n Jeldrik Anmerkung Ph"onixflug)

Die Sprache der Orks ist, wie man sehen wird, voller harter Laute, vor allem -k und -r. Wenn sie gesprochen wird, sollte auch darauf geachtet werden, diese Laute zu rollen, knurren und herauszusto"sen. 

Grenzbr"ucker Orkisch scheint eine aglutinierende Silbensprache zu sein, jede Silbe ist ein Wort und  tr"agt  eine Bedeutung. Weitere Bedeutungen entstehen durch Zusammensetzungen der Silben/Worte als Komposita und durch Anhang von Silben. Verben werden wohl nicht flektiert, sondern durch Hinzunahme weiterer Worte wie 'fr"uher/gestern' oder 'morgen/zuk"unftig' in verschiedene Tempi gesetzt. Eine festgelegte Wortreihenfolge gibt es nicht, aber meist werden handelnde Personen vor die Verben gesetzt. 

\newpage

Pronomen:\\

ar =  ich \\
ur = du \\

('sie' wird durch eine Benennung der Gruppe oder einzelnen Person gebildet, "uber die geredet wird)\\

-nan - mehrzahl, alle, viele \\
arnan = wir \\
urnan = ihr \\

artok = mein \\
urtok = dein \\
\newpage
Verben:\\

- tok = Verb des Haben und Besitzens.\\

kramm = fre"sen \\
nirkramm = trinken \\

- nir =  ! verneinung/umkehrung de"sen, was davor oder dahinter steht\\

hosch = k"ampfen \\
rakk - jagen \\

\newpage

Nomen: \\
ork = Ork\\ 
homm = Mensch\\
kat = Katze \\
nirkat = Ratte/Maus  \\
mago = Hexe \\
pak = Waffe\\
gorr(o) = gro"s \\
schakk = Beh"alter? Ding?\\
Sulk = Feuer \\
gach = Rache \\
nakk = Krieg \\
hok = Feind  \\
nirhok = Freund\\
kar = Geist/Magie\\ 
karoschakk = Magie(ding)\\
Mok Gakhan = Die Maske (Morbus' Artefakt)\\
ham = Holz \\
tul? tuul? = Dorf? H"utte? Blut? Tod? \\

Wie man sieht, k"onnen Worte durch ein zus"atzlich eingef"ugtes 'o' be"ser sprechbar gemacht werden, sowohl die Form 'karschakk' als auch 'karoschakk' sind richtig. \\
Miyasaris Toreville
\newpage

\section{ Auszuege aus der Thesys}

Vorwort:\\
\yinipar{D}ie Thesys ist ein Dokument, da"s uns Wi"sen ueber vergangene und zukuenftige Geschehni"se in Verbindung mit dem Schatten verspricht, der Grenzbrueck befallen hat. Wann immer uns ihre Anwesenheit offenbahr wurde und ein neues Kapitel in ihr zu oeffnen war schrieben wir ihren Inhalt nieder um sie der Nachwelt zu erhalten. Wenige Paszagen und Stuecke der Thesys wurden uns offenbahr und dies sind die Abschriften, die von wenigen Mutigen gemacht wurden.\\
Sie repaesentiert unser Wi"sen im Sommer des Jahres 258 nach Jeldrik und erhebt keinen Anspruch auf Vollstaendigkeyt, da sie aus vielen verschiedenen Quellen und unter besonderen und wiedrigen Umstaenden zusammen getragen wurde.
\newpage

\begin{center}
\begin{Huge}Im Gedenken an:\\\end{Huge}

\vspace{10mm}

Aaron - Diener des Ewigen\\
Felian - Held\\
Harkon - Knappe Grenzbruecks\\
Varn Chamounde – Adeptus Cantus Harmonae\\
Andor - Adeptus Cantus Harmonae\\
Grunas – Wolf der Armee Condras\\
Tarik – Soldat der Armee Condras\\
Jonan – Soldat der Armee Condras\\
Kira – Heldin\\
Illayda Wolfslauf – Archontin Condras\\
Fisken – Unnabhaengiger Condras\\
Miyasaris Toreville – Adepta der Academia zu Nektor\\
\end{center}

ebenso wie die zahllosen Soldaten und Kriegern aus Grenzbrueck, Engonien, Condra und vielen anderen Laendern, die ihr Leben im Kampf gegen die Schatten verlohren haben.

\newpage

\subsection{ Von den Lakaien}

\yinipar{M}ein Name ist Conradus Phelan Phileas\\
Ich diene treu meinem Koenig.\\
Ich sah die Diener des Schatten.\\
Zahlreich sind ihre Scharen.\\
Doch nur vier Lakaien thronen in seinem innersten Circel. Von diesen will ich euch berichten, Unwiszende. Auf dasz ihr es ihnen erzaehlen moeget. Den anderen Unwiszenden, Unglaeubigen. Berichtet ihnen die Wahrheyt.\\

Ich sah die Warheyt.\\

Ich blickte in die tiefe Schwaerze, welche mein Herz mit Furcht und Schrecken erfuellte. Und Schmerzen straften meine Seele.
Aus der Tiefe aber stieg ein Mann. Ein alter, schwacher Greis von magerer Statur. Getuetzt auf einen Stab, welcher geformt gleich einem Drachen. Seine Augen waren von Gaenze weisz, so als sey er blind und habe das Licht der Welt verloren. Und er sprach zu mir mit schwacher Stimme.

Ich bin Selhenas, der Seher. Andere nennen mich den dunklen Augur. Ich diene dem dunklen Prinzen. Durch mich sieht er, weisz er, fuehlt er. Lasz Dich nicht taeuschen Fremder, von meiner Schwaeche. Lug und Trug, Schein und Sein sind meine Tugenden.
Einmal noch werde ich dem Dunkelprinz entfliehen in der Zeit. Doch dann unser Pact auf ewig sey geschloszen, wenn Wyrdrak kommt, den anderen, der in mir wohnet, zu vernichten. 

Und als er solcherart gesprochen, seine Stimme erklang grauenhafter als zuvor, so als spreche ein anderer nun zu mir. Und seine Augen waren von dunkler Flamm erleuchtet und aufrecht stand der einstge Greis und sprach abermals.

„Fortan soll ich, Morbus Aug’ geheiszen werden. Denn dies bin ich seit Anbeginn der Zeit. Weder Vergangenes, noch Gegenwaertiges, noch Zukuenftiges bleiben mir verborgen. So sehe ich die Furcht in Dir und allen Sterblichen. Wohl tut ihr daran, denn er wird euch vernichten. Ich bin und war und werde seyn. Selhenas, der Seher, dunkler Augur, Morbus ewges Auge. Gleich welchen Namen Ihr mir zutragt, dies ist einerley, denn Euer Verderbnis ist besiegelt. Dies ist das Wort des Einen.“

Und als er so gesprochen, ein Drache von schwarzen Flammen seinem Stab entsprang und sein giftger Odem streckte mich nieder. Dann ereilte mich das Dunkel.


Und erneut blickte ich in die tiefe Schwaerze und Furcht und Pein ergriffen abermals meine Sinne.
Da sprachen Tausende von Stimmen zu mir. Wehklagend, weinend, schreiend.
Und als sie an mein Ohr drangen, ich dem Wahnsinn bald anheim gefallen waer.

So blickte ich mich um voll Furcht. Da sah ich eine Statue von grauem schlichten Steyne, so wie man ihn findet an des Meeres steilen Klippen. Geformt war sie einem Menschen ebengleich in langem schlichten Gewande. Doch war der Steyn nicht starr und solcherart sprach er zu mir.
Ich bin Wyrdrak, der Prophet. Andere nennen mich den steinern’ boten. Ich diene dem dunklen Prinzen. Durch mich kuendet er vom Anbeginn seynes ewgen Reiches. Einstmals war ich starr und still. Geformt von toericht’ Hand. Doch er gab mir Leben am Anbeginn der Zeit. Er erwaehlte mich, sein Mund und seine Stimm zu seyn.“

Und als er solches mir zugetragen, verwirrt waren meine Sinne. Und ich sah. In seiner steinernen Hand hielt er von dunklem Blut getraenktes Tuch und darob Zeychen, welche berichten vom Anbeginn der verdorbenen Zeit. Und erneut ertoenten seyne tausend Stimmen im Strudel der Ewigkeit.

``Fortan sollt ihr mich ‚Morbus Mund’ nennen, Sterbliche. Folgt meinen Worten und schnell und unbekuemmert soll euer Ende seyn. So kuende ich Euch von der Zukunft, die gewisz ist. Schenket meinen Worten  Glauben. Alljene, welche Hoffnung in Euch wecken, nichts anderes sind als Heuchler, Wegner, Suender. Moeget Ihr ihren Worten auch Glauben schenken. Ich kenne die Warheyt und kuende Euch von Ihr. Ewger Stein und doch zugleich lebendig. Wyrdrak, der Prophet, steinerner Bote, Morbus Mund.
Gleich welchen Namen ihr mir zutragt, dies ist einerley. Denn Euer Verderbnis ist besiegelt. Dies ist das Wort des Einen.''

Und als er so gesprochen, ergriff er mich mit kalter Hand und fest umklammerte er meinen Hals und streckte mich nieder. Dann ereilte mich das Dunkel.

wieder sah ich in den tiefen dunklen Schlund. Und ebensolches, wie zuvor abermals geschah. Da sah ich einen Reiter auf dunklem Rosze. Finster war der Rappen, augenlos und schmutzig abgenutzt sein Fell. Der Reiter aber ward Acht der Ellen hoch, geharnischt wie die Unseren vor vielen Jahr und Tag. Auf seinen Schultern thront ein leerer Helm und nur das Blitzen zweier Augen erkannte ich darin. Und es schien als zittere der Grund, auf dem ich stand, als er zu sprechen sich anschickte.

``Ich bin Ildûr, der Feldherr. Andere nennen mich den Schlaechter. Mord und Pest sind meine Kinder. Ich diene dem dunklen Prinzen. Durch mich fuehrt er die Seynen zum Siege am Ende der Ewigkeyt. Lasz dich nicht taeuschen Fremder. Gnade wird er nicht obwalten laszen. In unsren Reihen niemand ist von Schwaeche oder Feigheit. Einstmals folgte ich dem Ruf der Ehre und des Ruhmes, doch dieser Pfad nur der des Narren ist.''

Und als er solch gesprochen, sah ich die Legionene, die ihm folgten. Scharen unbarmherzger, miszgebildet Kreaturen. Verstoszen vom Schicksal und den Himmlischen, hinab geschleudert in der Hoellen ewgen Schlund, so krochen sie heran.

``Fortan nennt mich ‚Morbus Hand’. Mein Schwert und meine Faust wird Euer Seyn vernichten und Eures und da Schicksal Eures Koenigs bald besiegeln. Ich trage seine schwarze Krone. Zeichen seiner Macht und Herrschaft. Ildûr, der Feldherr, Schlaechter, ‚Morbus Hand’.
Gleich welchen Namen Ihr mir zutragt, dies ist einerley. Denn Euer Verderbnis ist besiegelt. Dies ist das Wort des Einen.''

Und seyn gewaltger Schwerthieb traf mich und reiszender Schmerz ergriff mein Herz. Dann ereilte mich das Dunkel.

Schlieszlich blickte ich erneut in die Tiefen, da ich nunmehr auch den letzten Schrekcne sehen wollte. Der vierte seiner Lakaien aber blieb im Dunkeln. Nichts vermag ich zu berichten. Doch einen suesze Stimme meinen Namen mir zufluesterte und auf mein Fragen ich allein vernahm.
``Eschra, Eschra werde ich geheiszen. Mein Antlitz wirst du noch erblicken, bald schon im naechsten Zyclus der Zeit.`` Dann verstummte es. 


Mein Name ist Condradus Phelan Phileas. Ich diente treu meinem Koenig. Ich sah die Schatten. \\

Dies ist die Warheyt.\\

Nunmehr trage ich die Buerde meines Koenigs. Bei meinem Blute…\\
Betet fuer meine Seel.


\newpage

\subsubsection{ Von der Schlacht auf den grauen Feldern}

\yinipar{H}ier stehe ich nun, Tylon, Koenig von Limest. An meiner Seite hunderte getreue Brueder, bereit ihr Leben zu laszen fuer mein eignes auf diesem Felde. Grau ist die Erde und verdorrt. Nichts waechst hier.

Die Reihen der Lanzen und Schwerter, Schilde und Boegen erfuellen mich mit Stolz. Und doch dringt Trauer und Verzweiflung an mein Herz, denn das, was kommt, mir allzu sehr bereits gewisz. Dort drueben, wenige Manneszchritt steht der Feind mit seinen Horden. Welch seltsam anmutige Kreaturen. Und doch bereit uns nun dahin zu schlachten. Ach, weh uns!

Nun stuermen sie heran, manche auf allen Vieren. Die Zaehne fletschend, harrend nur darauf unser Fleisch zu reiszen und sich an unserm Blute wohl zu laben.

Eine erste Salve schickt man ihnen entgegen und schon kommt dieser Ansturm zu erliegen. Eine zweite Salve meiner treuen Boegen wird weiteren Ansturm gleich im Keim ersticken. Wie Flammen regnet es auf sie herab als weinten die Goetter bittere Traenen. Myrna hilf uns, steh uns bei, dies Unterfangen zu einem rechten End zu bringen.

Jetzt senden sie uns ihre Hunde. Pest und Tod im Nacken. Schwarz ist ihr Fell und blutig ihre Lefzen. Doch ahnen sie noch nicht, dasz hinter jenem Huegel dort zehn Dutzend unserer Getreuen harren, bereit sie gleich im Sturm der Piken hin zu strecken.

In den Augen der Unsrigen erblickte ich den Funken Hoffnung, welchen ich bereits seit vielen Tagen hab verlor’n. Noch glauben sie dem Feind widerstehn zu koennen. Oh Myrna, hilf ihnen und lasz sie nicht in Angst und Zweifeln sterben.

Am Horizont ziehen schwarze Wolken auf. Regen faellt in Stroemen auf das graue Feld, so als wolle er alles Blut hinfort wischen. Doch noch ist dies Gemetzel nicht vorueber.

Nun ist’s an Uns dem Feinde sich entgegen zu werfen. Der Blick der treuen Reiter, wartend nur auf mein Zeichen wird ungeduldger, je mehr des Stundenglases’ Sand verrint. Und voran reiten sie, stolz und erhaben. Solcherart haette sich jeder Feind in Furcht und Angst hinfort gemacht. Doch nicht dieser, da er solch’ Wort nicht kennt.
Nicht einmal die andere Waldesgrenz vermoegen die Getreuen zu erreichen. Schatten raffen sie dahin von ihren stolzen Roeszern.

Abermals der Feind am Zuge ist. Und seine Zahl ist uebermaechtig und wie viele wir auch zu Tode bringen, so werden wir dies Feld nicht siegreich mehr verlaszen. Auch wenn noch keiner der Unsrigen waer gefallen.

Neben mir steht mein guter elbisch’ Freund und Gefaehrte Liath Mith An Elathas. Das Schwert gezeuckt, bereit sich mit den Seinen in den Tod zu stuerzen. Doch mahn ich ihn noch abzuwarten.

Endlich erscheint er auf dem Felde. Lange hat er uns seine Miszgunst durch sein Fehlen spueren laszen. Doch nun bedarf es seiner selbst auf diesem Feld von Leichen. Acht Ellen an Groesze blickt sein leerer Helm ueber’s Feld. Er der dunkle Feldherr. Entsandt von seinem Herrn uns endgueltig zu vernichten. Auf seinem Haupte thront die schwarze Kron wie mir verheiszen. Ein Zeichen seiner schwarzen Klinge und es scheint, als werfe sich der finstre Wald uns selbst entgegen. Die Horden, wie eine wallend’ Flut, fallen sie ueber uns einher.

Das Schreien um mich erst verhallt, als ich ihm endlich gegenueber stehe. Gewaltig sind seine Hiebe, zerschmetternd seine Worte. Giftger Odem stoeszt aus der Erden Reich. Willig mich qualvoll zu ersticken. Doch noch widerstehe ich. Schmutz und dunkles Blut benetzt den einstmals blanken Harnisch. Da endlich dringt die dunkle Klinge in mich und zieht mein Leben mit sich. Erst spuert’ ich Waerme von dem heiszen Stahl, dann kalten Tod. Sein hoenisch Lachen dringt leise fernab an mein Ohr. Doch reisze ich die treue Kling – die Gabe meiner Ahnen, mein Erbe und Vermaechtnis – empor und mit schrecklich Hieb, wie ich es mir niemals ertraeumt zu fuehren, zerschmettert ist die schwarze Kron und seine Hand sinkt wankend in sich zusammen. Zurueck bleibt nur der leere Helm, des’ rote schreckenhafte Augen sind alsbald verglimmt. Die Macht des Schatten liegt in Splittern, doch ebenso auch meine Kling.

Langsam wird es Dunkel und der Tod kommt. Oh Myrn, sey meiner Seele gnaedig. Ich tat, was mir aufgetragen… Nun lieg ich hier, sterbend in den Armen meiner Mannen. Meine Klinge ist zerbrochen ebenso die schwarze Kron. Sein Feldherr ist verschwunden. Verweht im Nebel, der nun aufzieht, wie ein Schleier zu verhuellen alles Leid. 

Mein getreuer Branbart tritt an mich heran. Seine Augen sind voll Hoffnung und voll Freude. So nah ist er und doch schon fernab meines Lebens. Und seine Worte hallen leise an mein Ohr: Der Tag ist Euer, Herr! Der Tag ist Euer!
So nehm’ ich ihn mit letzter Kraft in meine Arme. Traenen und Entsetzen haben von seinem Blick Besitz ergriffen, als er gewahr, was wird geschehen. Die gerade erst geboren Hoffnung stirbt mit mir… Der Tag ist Euer, Herr! Der Tag ist Euer! Und dann kommt die ewge Nacht…
\newpage

\subsection{ Ueber den Bund}

\yinipar{D}amals gingn wir in das Land Belartah. Nach Tryx, wo das Oracel weilt. Weit und beschwerlich war unser Weg. Doch als wir an jenen Ort gelangten, waren bereits viele dort versammelt. Und so warteten wir viele Stunden und Tage, Naechte und Monde bis es uns erlaubt, das Oracel zu befragen. Wir brachten unsere Gaben dar und noch bevor Ebeneas unsre Frage stellte, antwortete das Oracel.

Es ist Nacht geworden in Eurem Land. Ihr fuerchtet die Dunkelheyt.
In laengst vergangenen ‚Tagen schmiedete Einer, der im Dunkel lebt und sich vom Tode naehrt, eine Krone in einem gewaltigen Brunnen voll irdenem Blut. Machtvoll und zerstoererisch ist diese. Zeichen seiner Herrschaft. Dann formte er aus dem Dunkel, welches ihn umgab, jene Kreaturen und Bestien, die Eure Lande plagen. Und mit der Hilf der Krone hauchte er ihnen Leben ein und befahl Ihnen, Euer Land zu ueberfallen, Eure Doerfer zu pluendern und die Tempel Eurer Goetter zu zerstoeren. Denn er haszt Eure Goetter, er glaubt sie hatten ihn einstmals verraten. Doch irrt er. Und ihr tut gut daran, die Goetter zu verehren.

Einer wird kommen, Euch alle zu einen. Ihm werden die Himmlischen eine Gabe schenken durch einen Boten. Und die Gabe soll fortan das Zeichen seiner Koeniglichen Herrschaft sein. Zeichen aller Koenige, die da kommen werden. Er wird das Dunkel zum ersten Mal verbannen.

Doch der Alte Feind, der im Dunkeln lebt wird wiederkehren und erneut ueber Eure Kinder hereinbrechen. Diesmal tosender und schrecklicher als zuvor. Wenn Eure Kinder mit ihm fechten, werden Berge Feuer speyen und das Land im weiten Meer versinken. Auch ich werde den gewaltigen Fluten anheim fallen.

Ein Mann aus deinem Geschlechte Ietmhen wird in jenen Tagen alles Land regieren. Er wird die alte Gabe, Zeichen seines koeniglichen Blutes, fuehren.\\

Was dort geschieht, liegt noch im Zwielicht.\\

Doch von da an das Dunkel mit dem Licht verlochten. Das Reich des Alten Feindes an das Blut des koeniglich’ Geschlecht gebunden fuer viele Mondenlaeufe. Und weder der eine, noch der andere vermag die Oberhand zu erringen. Dann wird Stille kommen fuer wenige Menschenleben. Fortan wird kein Koenig mehr das Land beherrschen und Zwist und Streit wird Jene schwaechen, die beszer daran taeten, sich auf die Rueckkehr des Alten Feindes zu besinnen. 

Denn dieser, geschwaecht und in Ketten, wird neue Diener sammeln, die fuer ihn streiten. Ihn zu befrei’n von seinen ‚menschgeformten’ Feszeln.
Die Kron aber, sie liegt in Splittern immer noch erfuellt von ihrer einstgen Macht. Sie endlich zu vernichten, nur vermag der Goetter guetge Hand. Geschmiedet werden musz sie nochmals. Denn nur was eins ist, kann endlich vernichtet werden. Doch sie zu fuegen, vermag nur er, der wahre Erbe des Koenigs. Und er wird herrschen ueber das Land… So wiszet: wer die Krone schmeidet, wird Koenig seyn, denn so haben die Goetter es bestimmt. 
Und als Zeichen seiner Herrschaft soll die Gabe, welche das Dunkel einst zerschmettert, ebenso gefuegt werden zu einem Stueck. Und jener wird sie fuehren.

Dann soll die Kron gebracht werden an jenen Ort, wo sie dereinst erschaffen und ein Stern wird seyn Opfer erbringen. Ich sehe Mauern dort und einen Stein, errichtet wo einstmals seine Schmiedestaette war. Und eines schwarzen Steines werde ich gewahr im Nebel. „Acht an der Zahl, Feszeln, Ketten, Frevel, Torheit, Menschen Sein“, fluestert mir der Wind. Ein zweifelhaftes Unterfangen. Denn, wenn die schwarze Kron gelangt in des alten Feindes Hand, wird Leid und Schmerz das geringste sein, was Euren Kindern widerfaehrt.

Doch auch die Gabe wird solcherart erneut geboren und nun vielleicht obsiegen. Dies aber noch verborgen in den Nebeln der Zeit. Geh nun Aurethea, da ich Dir Dein Schicksal offenbart.“

Aber wir verstanden nicht, da niemand solchen Namens unter uns war. Da sprach das Oracel nochmals zu uns.

„Tritt hervor Aswine, Eheweib des Ebeneas. Du wirst bald schon einen Jungen gebaeren von kraeftgem Wuchs. Dieser wird Euch einen wohlgesunden Enkel schenken. Dieser wiederum soll eine Tochter haben und jene wieder eine Tochter. Dieses Kind wird verstehen, wovon ich Euch gekuendet. Daher schreibt meine Worte auf, dasz sie ueberliefert werden bis zu jenem schicksalhaften Tag, da sie Bedeutung erlangen fuer die Welt.“

Und solcherart taten wir… Bald schon ward Ebeneas und Aswine ein Sohn geboren…
\newpage
\subsection{ Von Ishpathans Tempel}

Es war dereinst in den Tagen des Abnerus von Limest, dem Enkel des Abnons, des Einers der Staemme, als  Ishpathan dem groszen Tempel von Belthar als Hohepriester Myrns vorstand. In jenem Jahr zaehlte Ishpathan bereits an die Dutzend Sternenlaeufe.
Da erschien Ishpathan im Schlaf ein Bote der Ewgen und sprach zu ihm

„Hoere Ishpathan. Dies ist die Botschaft welche Dir die Ewgen und Einen senden. Richte dein Wort an die Glaubenden, welche zum Tempel kommen und kuende ihnen, was ich Dir sage.“

Und der Bote sprach mit schrecklichem goettlichem Munde zu Ishpathan, solcherart, dasz ein jeder andere Sterbliche vertilgt worden waere von Furcht und Wahnsinn.

Als Ishpathan am naechsten Tage erwachte, trat er hinaus aus seinem Gemacht und begab sich in den Tempel, wo die Glaubenden, Suchenden, Hoffenden, die Verzweifelten, Todgeweihten und Kranken Rat, Trost und Erloesung suchen und sprach zu ihnen wie ihm der Bote geheiszen.

„ Hoeret Ihr Maenner und Frauen aus Belthar, Ihr, die Ihr Rat im Tempel Myrns sucht. Dies Nacht erschien mir ein Bote der Ewgen. Er kuendete mir von Myrns Willen. Hoeret ihre Worte durch meinen Mund. Die ewge guetge Mutter ist voll der Freude ob Eurer Gebete und Lobpreysungen in diesem Tempel. Doch befahl sie mir das Folgende: Geh Ishpathan und nimm Dir zwei Dutzend der ehrbarsten und glaeubigsten Maenner und Frauen aus meinem Tempel. Geh fort von Belthar, nach Westen. Nicht weit vor der Stadt wirst Du eine Lichtung im Walde finden. Das naehrende Waszer flieszet dort. Du wirst einen Stein finden, unscheinbar und doch von ungeheurer kraft. Es ist meine Traene, welche ich in dunkler Nacht vergoszen, als mir Maewon mein Gemahl eroeffnet, was Euch, den Menschen, meinen Kindern bald bevorsteht. Sey wachsam und heute den Steyn wohl, wenn Du ihn gefunden. Doch wisze fuerderhin. Dort, wo du die Traene gefunden hast, errichte mir einen Tempel. Bringe den Stein hinein und versiegele ihn gut. An jenem Tempel aber 
sollst Du weder beten noch huldigen, noch Lobpreysen, noch ehren. Und so soll kein anderer tun auf lange Zeyt. Denn dies ist mein Wille. 

Und als der Bote geendet hatte, da rief ich ihn an und sprach:

„Sage mir ewge Mutter, wenn ich Dir einen Tempel errichten soll, an welchem weder gebetet, noch gehuldigt noch lobgepriesen, noch verehrt werden soll, wozu dienet dieser dann?“
Und der Bote sprach erneut.
„Frage nicht nach Grund noch Sinn. Zweifelst Du etwa an der Ewgen Willen. Nun tu wie Dir geheyszen.“

So stehe ich nun vor Euch und frage, wer geht mit mir in die Waelder, um der Ewgen Willen zu erfuellen?“
Und es dauerte nicht lang, da fanden sich zwey Dutzend Glaubende. Und sie taten wie Myrn ihnen durch Ishpathan geheyszen.

Am nachesten Tage erreichten sie jene Lichtung im Walde, unweit der Tore von Belthar. Doch konnte Ishpathan jenen Steyn nicht erblicken, von welchem Myrn ihm durch den Boten hatten Kunde bringen laszen. Den halben Sonnenlauf suchte er danach. Schlieszlich fiel er auf seine Knie und betete zu Myrn, sie moege ihm ein Zeychen senden. Viele Stunden verweylte er solcherart und die Nacht brach herein ueber das Land. Da kam einer der Glaubenden und sprach zu Ishpathan.

„Hoher Bruder. Verzeiht, doch die kalte Nacht bricht herein und wir haben nichts an Speysen, noch an Decken oder Unterschlupf. Lasz und zurueckkehren nach Belthar und morgen wieder kommen.“

Und als der Mann diese Worte gesprochen und Ishpathan sich gerade erheben wollte, da sah der Hohepriester ein Licht funkeln inmitten der Lichtung im trockenen Gras. Und als Ishpathan sich ihm naeherte, erkannte er den Steyn von welchem der Bote Kunde getan und hoerte die weisen Worte der ewgen Mutter.

„Hoere Ishpathan, uebet Gedult. Ich will Euch betten und behueten, Euch naehren und Euren Durst loeschen, doch zweifelt nicht an meinen Worten. Nun errichte meinen Tempel.“

Und als Ishpathan dies vernommen hatte, blickte er auf und wie durch ein Wunder erschien ein Zelt von feinstem Stoffe in den Farben des Nachthimmels auf er Lichtung. Darin brannte ein Feuer und viele Speysen und Weyne waren dort angerichtet. Und so zogen sich die Glaubenden zurueck in das Zelt und aszen und tranken und ruhten ueber die Stunden der Nacht, wohl behuetet in Myrns Schosz.
Am naechsten und den darauf folgenden Tagen begannen die Glaubenden mit der Errichtung des Tempels und Ishpathan wies sie an, wie sie zu handeln haetten. Ein langer Pfad fuehrte in das Herz des Tempels, in jene Kammer, dort wo der heylige Steyn hineingebracht werden sollte. Viele Sonnen und Monde brachten die Glaubenden Felsen heran, gruben und hauten, richteten Steyn auf Steyn auf. Als sie schlieszlich geendet hatten, liesz Ishpathan eine gewaltige Platte aus einem der Felsen schlagen, um den Eingang zu versiegeln. Dann aber sprach er zu seynen Bruedern und Schwestern.

„Hoeret meine Freunde. Ihr habt wohl getan. Doch ist es nun an der Zeyt fuer Euch nach Hause zurueckzukehren. Geht nun!“

Da fragten die Glaubenden ihn, ob er sie nicht begleiten wolle und Ishpathan antwortete ihnen.

„Nun wiszet. Damals als ich Euch bat mich zu begleiten, kunedete ich Euch nicht von all den Worten des Boten der ewgen Mutter. Denn diese waren noch nicht fuer Euer Ohr bestimmt. Dies ist mein letzter Herbst. Es ist Zeyt fuer mich heimzukehren in ihren Schosz. So erfuelle ich einen letzten Dienst. Eumnan, Du aber sollst meyn Nach folger seyn. So kuendet allen, dasz niemand zu diesem Tempel kommen soll. Weder um zu beten, noch um lobzupreysen, noch um zu huldigen oder zu ehren. Denn dies ist ein Tempel der Trauer und der Einsamkeyt. Ich vermag Euch nicht zu sagen, weshalb wir ihn errichtet haben, dies wiszen nur die Ewgen. Doch sind dies meine letzten Worte an Euch. Laszt sie Euch und alljenen, die es wagen wollen, diesen Tempel zu betreten, eine Warnung seyn!


\begin{huge}So spreche ich Ishpathan.\end{huge}


Dies ist der Tempel der ewgen und barmherzgen Traene. Tief in ihm liegt ein unermeszlich’ Schatz. Nicht Gold noch Silber, nicht Macht noch Wiszen ist es. Jene, die des Schatzes einstmals beduerfen, werden es erkennen. So wiszet, Ihr, die Ihr des Schatzes einstmals beduerft, dasz ob Eures Frevels der Pfad der Trauer beschritten werden musz. Wohltuende Laeuterung wird Euch widerfahren oder der ewige Hoellentod.



\begin{huge}So hoeret:\end{huge}
Porta Noctis in Maewis luce aperiit. Tam te Ishpathanem nuntium memento. Iter Tristitiae molestus est. Lapidosum et plenior miseria. Per purificatione ignem poenam da. Tandem credendus liberiterum mortem reperiit.

Sodann wisze.
Fuenf Auserwaehlte vermoegen den Pfad der Trauer zur gleichen Zeyt nur zu beschreiten. Den Sechsten und alle nach ihm Kommenden aber soll der Tod ereylen. So ist Myrns Wille. 

Je weiter Ihr schreitet, desto groeszer soll die Dunkelheyt um euch werden. Einsamkeyt und Furcht und Angst werden nach Euch greifen, doch zweifelt nicht. Die wahrhaft Glaubenden werden die Schwelle zum Licht erreychen. 

Und dies ist die Dunkelheyt:

Zuersten:
Ob Salz, ob Zucker, ob Weyn oder Gift, ist’s alles gleichermaszen.
Sodann:
Ob Schrei, ob Fluestern, ob Sang oder Sprach, ist’s alles gleichermaszen.
Sodann: 
Ob Licht, ob Schatten, ist’s alles gleichermaszen.
Zuletzten: 
Ob Steyn, ob Holz, ob Waszer oder Feuer, ist’s alles gleichermaszen.
Die Schwelle aber wird die Dunkelheyt vertreiben. 

Sodann wiszet, die ihr den Pfad beschreiten wollt.

Item die Erde ist wohltuend. Wir sind ihr entsprungen. Darunter aber liegt der Hoellen ewger Schlund. Tod und Pest koennen ihm entspringen wenn unachtsam Ihr seyd. So entsinnt Euch, dasz Euer Haus aus Holz und Stein errichtet, wie die Ewgen Euch es dereinst zugewiesen.

Item der goettliche Drachen haelt ewge Wacht. Seynem toedlich’ Bisz mueszt Ihr entrinnen. Die Zeyt und das Lied mag Euch behilflich seyn.

Item das Ungetier dem Dunklen dient. Es faengt die Fliege in seynem Netze. Giftig und todbringend ist seyn Stachel. Weckt Ihr es, sollt ihr fortan die Fliege in seinem Netze seyn. 

Item bald schon wird Euch Zweifel ueberkommen ob Ihr den richtgen Weg wohl eingeschlagen. So wiszet, der wahrhart Glaubende zweifelt nicht und laeszt sich nicht taeuschen durch Lug und Trug. Schreitet voran und Ihr werden Unueberwindiches ueberwinden.

Item nunmehr die Schwelle zum Lichte unmittelbar vor Euch. Fast vermoegt Ihr sie zu ergreifen. Doch erinnert Euch dieser Mahnung: alleyn der Demuetge mag das Reich der Ewgen wohl betreten! 
Dies sind die Worte Ishpathans, Diener Myrns, der Ewgen und Allumfaszenden.“

Und als Ishpathan geendigt hatte, betrat er den Pfad der Trauer und kehrte heim in Myrns Reich. Die Glaubenden aber versiegelten den Pfad mit der steynernen Platte, so wie es Ishpathan ihnen befohlen hatte. Dann kehrten sie zurueck nach Belthar und der Tempel geriet in Vergeszenheyt bis zu jenen Tagen in welchen Belthor der Dritte den Thorn in Belthar bestieg. In jenen Tagen aber waren die Felder, auf welchem der Tempel steht, die Telnischen geheyszen. 

\newpage

\subsection{ Vom verfluchten Fleische}

\yinipar{S}o berichten Wir Euch von jenen Tagen als das verfluchte Fleisch geboren ward. 

Einstmals die Brut des Stammesnamen Sulk Nakk dem Goetzen Morkoi huldigte.
Da erschien den niederen Bestien eine Gestalt von abscheulichem Wesen, deszen Antlitz unter einer guelden’ Fratze ward verborgen und es sprach in ihrer Zunge das folgend hoenisch, frevelhafte Wort.

„Ihr niederen Kinder vom Blute und Fleisch der Orcken, Mok Gakhan ist mein Name. Ashtar, der ewge grosze Rachevater sendet mich, seinen ergebenenen Diener zu Euch, um Euch ein groszes Geschenk zuteil werden zu laszen. So huldigt nicht laenger dem Goetzen Morkoi. Denn er hat euch verlaszen und schmaehet euer Blut und Fleisch. Betet nunmehr allein zum Rachevater und Macht und Staerke soll Euch widerfahren und Ihr sollt aller anderen Brut ueberlegen seyn.“

Dann verschwand Mok Gakhan, so wie er zuvor gekommen ward.

Als dies geschehen, entbrannte groszer Streit unter der Brut, denn gor Kar der grosze Geistvater der Sulk Nakk ward nicht bereit dem Goetzen Ashtar zu huldigen, so wie die jungen Kar Kramm und Tuul Schakk es verlangten und er widersetzte sich ihrer Begehr.

In der darauffolgenden Dunkelheyt aber, als die Brut aus ihren Hoehlen und Loechern kroch, vernahmen Kar Kramm und Tuul Schakk die gespalten und doch verlockend Stimm des Mok Gakhan erneut. Und der goetzenlaesterliche sprach zu ihnen.

„Hoert, Ihr vom kraftvollen Blute unter der Brut. Ihr habt die Groesze und die Gnade des Geschenks erkannt, das Ashtar Euch ueberlaszen will. Gorr Kar ist ein Narr, alt und gebrechlich und ebenso die anderen, welche zu Morkai beten. So hoert! Schart die Staerksten Eurer Brut zusammen und kommt im vollen Mondlicht auf den alten Huegel der Ahnengeister. Dort werdet Ihr ein Totem finden. Es ist jenes Eures neuen Herrn Ashtar: der Euch Kraft und Macht zuteil werden laszen wird.“

Dann verstummte die lockende Stimme. Kar Kramm und Tuul Schakk aber taten wie die Stimme geheiszen und scharten sieben weitere Brueder um sich. Im vollen Mondlicht erblickten sie aber auf dem Huegel der Ahnengeister ein grauenhaftes Totem von widerlichem, laesterlichen Bild und ihr Blut geriet hierob in Wallung. Und so huldigten sie dem Goetzenbild, welches dasjeniges des Ashtar war und frevelten Morkai ihrem einstigen Goetzen.

Da ward der Mond ploetzlich bluthrot und ihre Gesaenge verstummten. Und kalter Nebel kroch durch das Dickicht der Baeume, hierauf bis zur kahlen Spitze des Huegels. Fast vermochte die Brut ihre eigen’ Krall und Klau nicht mehr zu erkennen, da offenbarte sich Mok Gakhan in jener grauenhaften Fratze und sprach aus seinem gueldenen Munde mit falscher Zunge.

„Wohl habt Ihr daran getan hierher zu kommen und Ashtar zu huldigen, Tuul Schakk und Kar Kramm und auch Ihr anderen. Ihr sollt einstmals Blutvater und Geistvater eines groszen Stammes werden. Doch hoeret, welchen Pact ich Euch biete im namen meines Herrn Ashtar.
Dieses Totem bringt Euch Kraft und Staerke, wann immer Ihr zu ihm betet. Er naehrt Euch und lindert Euren Schmerz und Wunden.“
Und als er so gesprochen durchflosz die Brut ein ungeheuerlicher Stosz voll Kraft und Macht und man vernahm ihr Heulen weithin uebers Land. 

„Ashtar wuenscht, dasz Ihr das Totem sollt das Eure nennen, so werde ich ihn Euch ueberlaszen. Doch um den Bluthpact zu besiegeln, sollt Ihr ein Opfer darbringen, um so Ashtars Gnade zu erflehen. Geht und schlachtet die Eurigen dahin. Beginnt mit dem groszen Geistvater, zoegert nicht vor Eurem Bruder oder Sohn. Dann trinkt ihr Bluth und bringt ihr totes Fleisch hierher zum groszen Totem des Rachevaters.“

Da zoegerte die Brut, die dort im bluthroten Mond versammelt, ob dieses ungeheuerlichen Frevels, den sie begehen sollten. Doch da sprach Mok Gakhan erneut mit lockend zweifelhalfter Zung.

„Ich seh’ Euch zoegern, elende Brut. So wiszet, wenn es Frevel waere, den Ashtar von Euch verlangt, so wird Morkai Euer Handeln unterbinden und seine Brut behueten. So geht, wenn Ihr den Pact erfuellen wollt.“

Und als Mok Gakhan solcherart gesprochen und im dichten Nebel ward verschwunden, da gingen die neun Frevler hinab und schlachteten ihr eigen Fleisch und Blut dahin, tranken von dem Bluthe ihrer Ahnen, ihrer Vaeter und ihrer Soehne und brachten das Fleisch zum Huegel der Ahnengeister, wo sie Ashtar die Opfer darbrachten. Und Morkoi unterband den Frevel nicht, da er seine Brut vergeszen schon vor langer Zeit…

Da offenbarte sich ihnen Ashtar selbst im fahlen Mondenschein und sprach mit schrecklich grauenhaftem Wort.

„Brut, die Ihr hier versammelt, hoert das Wort Eures neuen Herrn. Ihr meine Kinder sollt grosze Kraft und Macht erlangen und alle andere Orkenbrut Euch unterwerfen, dasz sie Euch diene immerdar. Die Menschen und Elben aber sollt Ihr vertilgen, denn ihr Dasein beschaemt mich. Jagt sie und opfert mir ihr Bluth.
Der Pact sei hiermit besiegelt. Doch wiszet, dasz meine Rache fuerchterlich sein wird, wenn Ihr mein Totem einst verlieren sollt. So achtet gut darauf, Ihr zollt mir Pfand mit Eurer Seele…“

Und Ashtars Stimme verstummte in der Dunkelheyt, doch hallten seine Worte wider in den Koepfen seiner neuene Kinder.

Und diese begaben sich hinab in die Hoehlen, die mit dem Bluth ihrer Brueder bedeckt und von der Schande ihres Frevels erfuellt waren. 
In den darauffolgenden Naechten jedoch begaben sie sich auf die Jagd nach frischem Fleisch und Bluth und begaben sich auf den Ahnengeisthuegel, um das Totem anzubeten und Ashtar zu huldigen. Und der Rachevater heilte ihre Wunden und erfuellte sie mit unbaendiger, zuegelloser Kraft und ungemeiner Macht. Bald schon waren ihre Namen unter der Orkenbrut und den Menschen gefuerchtet.

Der Schamane Kar Kramm, den die Brut Geistfreszer nannte. Der Treiber Tuul Schakk, welchen sie Todbringer geheiszen. Der erste Waechter Sulk Pak, welches in der Orckenzunge Feuerklinge heiszt. Der Name des zweiten Waechters war d Hokk Hosch, der des Haeschers Gach Kar ward Rachegeist genannt. Der Unhold Homm Kramm ward allerorten nur als Menschenfreszer wohl bekannt. Der Krieger Gorr Nakk, deszen Name den groszen Krieg bezeichnet. Der Jaeger Homm Rakk auch Menschenjaeger wohl geheiszen. Und schlieszlich der Geisthueter mit Namen Kar Hirschakk, welcher dem groszen Geistfreszer Kar Kramm diente.
Doch es kam die zwoelfte Nacht nach dem groszen Frevel, da die Brut wiederum am Totem gehuldigt und geopfert. Da kroch erneut der kalte Nebel ueber den Rachehuegel und die Brut fiel in tiefen Schlaf. Und Mok Gakhan erschien im Nebel, doch sprach er nicht zu der Brut. Er nahm das Totem seines Meisters, deszen Auge in dieser Nacht nicht auf dem Huegel ruhte und verbarg es in den tiefen Hoellen, aus welchen er empor gestiegen, so wie er es seit Anbeginn geplant hatte.

Und als die Brut erwachte, sah sie was geschehen war und Furcht ueberkam ihren Geist ob der Rache Ashtars. Sie glaubten sich tief in ihren Hoehlen vor dem Zorn ihres Goetzen verbergen zu koennen, doch durchdrang sein Auge und sein Hasz den kalten Stein.

Und sein furchtbares Wort erklomm die Hallen und Hoehlen von Bluth besudelt, in welchen die Schreie der dahingeschlachteten Ahnen auf ewig erklingen werden. 

„Ihr elenden Tore, was habt ihr getan. Ihr habt mein Totem verloren, so wiszet, dasz ich mein Pfand nunmehr einloesen werde um meine Schuld zu tilgen. Sowerdet Ihr jagen in der finstren Nacht, am Tage aber sollen Eurer koerperlosen Seelen wandern ueber das Land auf der Suche nach meinen einstigen Geschenk an Euch. Und weder soll Euch Klinge, Gift und Feuer toeten noch verletzen. Doch die Pein sollen Eure Seelen wohl verspueren, wenn Eure Koerper nun verwesen bis in alle Ewigkeyt. Und Euer Hunger nach Fleisch und Euer Durst nach Blut soll nicht gestillt werden obgleich ihr beides im Uebermasz erbeutet. Und ihr sollt erst Ruhe finden, wenn ihr mein Titem gefunden habt. Dann will ich Euch entlaszen von dem Pact.
Doch will ich nicht mehr sprechen zu Euch toerichten Frevlern. Ihr werdet meinen Willen aus dem Munde meines Dieners Mok Gakhan erfahren.“

Und so geschah es, dasz das verluchte Fleisch geboren ward. Jene, deren koerperlose Seelen am Tage wandern ueber das Land auf der Suche nach dem Totem, welches sie niemals finden werden und deren auf ewig verwesende Koerper des nachts geschunden von der Pein unzaehlger Schlachten, jagen nach Fleisch und Bluth zu stillen Hunger und auch Durst und doch nie Labsal finden.

Und allerorten waren sie fortan mehr nicht die Kinder der Sulk Nakk geheiszen, sondern vielmehr die Lok Ashtar. Und der Klang des Namens des verfluchten Fleisches ward gefuerchtet ueberall…
Lok Ashtar

\newpage

\subsection{ Vom Zwielicht}

Itzo gibt es einen Ort, den wir da nennen das Zwielicht.
Er liegt ganz nah an dieser Welt, und gleicht ihr ueber alle Maszen, doch ist er nicht hier
Altvorderen nennen ihn An Duin Danôr, der Rest kennt ihn nur als das Zwielicht.

Raum wirst freilich du dort finden, wie du ihn kennst vom Dieszeits, doch die Zeit, sie verlaeuft dort ganz und gar in anderen Bahnen.

Licht und Dunkel sind dort nicht im gleichen Ebenmasz wie hier, so scheint es manchmal von dichten Nebeln durchzogen, und immer von Duesternis zerfreszen. Ewiglich Nacht ist es dort.

Und still ist es dort. 
Auch du wirst es hoeren. An jenem Ort herrscht mehr Stille aber Ruhe ist dort nicht. Denn gefaehrlich ist es jeden Augenblick.

Der Gelehrte weisz, dasz es ein Planum ist. Eine andere Ebene unserer Welt. Ein Betrug an unserer Welt zugleich, denn die Grenzen vom Hier zum Dort zu ueberschreiten bietet dir zahllose Moeglichkeiten, dem Dieszeits eine Niederlage abzuringen.

Du kannst dort tun, was das Hier und Jetzt veraendert, und doch, wenn du rueckkehrst wird niemand davon Kenntnis haben. Die Maechtigen der schwarzen Kunst nur versuchen sich jenes Reich zunutze zu machen.

Du also sei vorsichtig ueber alle Maszen. Die Weltordnung zu verletzen, indem hinueber du gehst, wird niemals ungesuehnt bleiben. So duldet es die Ordnung der Sphaeren nicht, wenn allzu lang man dort verweilt. Eile dich also, oder jenen Ort wirst nie mehr du verlaszen.

Wenn du dich entscheidest an jenen Ort zu gehen, so beachte auch dasz nur nach Sonnen untergang die Pforten zu oeffnen sind. Denn dringt Sonnenlicht durch das Tor in jene ewig dunkle Halbwelt, so wirst das Gefuege der Ebenen du ganz und gar durcheinander bringen. Zerbrechen koennen die darob, und ein Loch in das Gefuege reiszen, welches dich schluckt.

Dann wirst du im Totenreich, in der Astrale oder gar einer Hoelle landen. Niemand weisz es.
Wenn du also aber eine Pforte geoeffnet hast, welche dich nach An Duin Danôr bringt, so lasze sie nach deiner Rueckkehr einige Zeit ruhen. Ein Jahr ruhe sie, wenn du wieder willst, ein paar Stunden mindestens, wenn einen anderen du anstatt deiner gehen laeszt. Sonst zuernen dir die Waechter des Weltgefueges allzu sehr, und beim naechsten Gang werden sie dich oder ihn erwarten.

Falls Vertraute du, in deinem dunkeln Tun, an jenen Ort mitnehmen willst, so musz fuer jeden von ihnen ein eigenes Tor her. Niemals versuche einen zweiten durch das Tor zu nehmen, welches du durchschreitest, denn die Zeit dort kuemmert es nicht, dasz ihr nacheinander ginget. Am gleichen Ort zu gleicher Zeit werdet ihr dort sein, und so werdet im Zwielicht ihr eins sein und verwachsen und untrennbar entstellt fuer immer.

Und noch etwas beachte, wenn Dienerschaft du willst hinueber bringen. Durch das zweite und dritte und vierte Tor, wie viele du auch hast, musz einer von anderem Blute, Geschlecht und Herkunft schreiten. Seid durch des Blutes oder des Schicksals Faeden ihr zu verbunden, so wird deine Praesenz im Zwielicht zu gebuendelt erscheinen und es wird den Ebenenwaechtern ein zu leichtes sein deinen Frevel zu entdecken. Sie werden dich finden und deine Seele mit einer Strafe versehen, die weit schwerer wiegt als der Tod im Dieszeits.

\newpage

\subsection{ Lied von der miszstalteten Geburth}

Die Waelder der Schattenlande im Grenzbruecker Norden

Brutstaette von Geschichten und jahrhundertelangem Morden,

Und eine Geschichte, die so traurig beginnt,

schrieb das Bluth eines gebrochenen Herzens, dasz durch Hasz nicht gerinnt.

Viele Ohren lauschten in jener finsteren Nacht

Dem Wehklagen einer Mutter, die ein miszstaltetes Kind zur Welt gebracht.

Sie wickelte das Kind mit der traurigen Gestalt in ein Tuch

Und trug es hinaus in den Schattenwald.

Viele Augen folgten lautlos den gehetzten Schritten

Das schiere Entsetzen hatte eine Fratze in ihr Gesicht geschnitten.

Und sie bettete das Kind mit zittriger Hand

In die grausame Wiege des Schatten’ Land.

Es lagen fast zwanzig Winter Schnee ueber der Tath, 

als es hiesz, dasz in der Stadt eine schreckliche Gestalt gesehen ward.

Mein treuer Leser Ihr moegt es erahnen, wem diese Bezeichnung galt, 

dem kuemmerlichen Wesen, zurueckgelaszen im Schattenwald.

Von hundert Stoecken, Steinen und Fluechen gejagt,

begriff sie, toericht es war, dasz sie zu hoffen gewagt, 

dasz sie, eine miszstaltete Geburth aus dem Schattenwald,

unter den Menschen leben koennte, ungeachtet ihrer tragischen Gestalt.

Die Stoecke und Steine ihr Ziel trafen sehr wohl, 

doch das Geschrei bewirkte, dasz ihr Augapfel schwoll, 

die Lider ueberlappten, der Kiefer trat vor,

Die Lippen zerplatzten und die Iris verlor…

…traurige Graublau, es zerflosz im Weisz,

der geschwollenen Augen und das Bluth und der Schweisz

verklebten das ergraute, schlohweisze Haar

und ich kann kaum beschreiben was dann geschah.

Sie erbrach eine Klanggewalt mit einem vernichtendem Schrei,

als wuerde die verbrennen, waehrend sie bei hellem Verstande sei.

Und der Schrei hallte noch in jedem Haupte

Und bohrte sich in Abgruende, in die das Bewusztsein nicht folgte.

Und der Schrei zerrte Bilder mit verbittlicher Strenge 

Vor das innere Auge, der halb wahnsinnig gewordenen Menge. 

Nach einer Zeit, die niemand zu benennen gewuszt,

sackte mit seltsam verdrehten Augen ihr Kopf auf die Brust.

Dann floh das Kind, die mordende Brut,

in die Arme eines Blinden, deszen Schein war gut.

Er nahm sie mit und schmiedete finsteren Plan, 

welcher den Menschen bringen wuerde den wahren Wahn.

Und so brachte er das Kind mit jenem Spiegel zurueck,

ihr Vater aber erkannte sein Blut an ihrem Blick.

Und Hoffnung keimte in dem alten Herrn.

Seiner Tochter Schoenheit wollte erretten er gern.

Den Rath des Roten daher befolgte er wohl, 

doch jener, diem nur sein wahres Ich gestohl’n, 

hatte sinistren Plan ersonnen,

welchem der „ohne Frieden“ vermocht nicht entrinnen.

Er tat, was die Stimmen ihm aufgetragen,

sich am Leiden und Tod der anderen laben,

ein silbernes Antlitz hiermit entzuecken,

und seine Tochter hierdurch dreye Schrei entruecken.

Doch alles geschah nicht zur rechten Zeit, 

und anderer machte ein Ende dem Leid.

Der Vater erstochen, sein Kind noch immer entstellt,

Seine Seele dem Roten anheim gestellt.

So wartet er auf den Tag und die Stunde,

da die Nachricht macht von ihm die Runde,

er sey zurueck auf seiner Veste,

und vollende nun  die verbliebenen Reste.

Und eine Stimme mag dann auf der Burg erschallen,

die Toerrichten werden dann vielleicht fallen,

die Tochter aber befreit von schrecklichem Blick,

erringt der dunkle Prinz hierdurch den Sieg.

\newpage

\subsection{ Von Ytalas und der Thesys}

Und so ward es dereinst, dasz Maewon, Hueter des Schicksals, Beherrscher der Magie, zu seinem ehrfuerchtigen Diener Ytalas kam, jenem weisen Magier, welcher mit seiner Tochter einsam in den Waeldern Mendreth’s lebte. Dorthin war Ytalas gegangen seinen Geist zu reinigen und sich der Kunde des Verborgenen zu widmen. Und die Liebe zu seiner Tochter ward unermeszlich... 

Und da Maewon sein Haus betrat, verneigte sich Ytalas und Maewon sprach zu ihm: „Erhebe dein Haupt, mein geliebter Sohn und hoere, was ich dir offenbaren will, denn die Zeiten wandeln sich und etwas schier Unmoegliches sollst du vollbringen. Wisze, dasz niemandem zuteil werden darf, was hier geschehen wird, denn grosz ist die Gefahr...“.

Und Ytalas erhob sich und sprach: „Vater, sagt, was ist es, was ich fuer euch vollbringen soll?“.

Maewon erwiderte: „Es sind nunmehr viele Sternenlaeufe vergangen, seit der Kampf gegen das Dunkel, gegen Morbus, jene Pest, welche von diesem Land Besitz genommen, beendet. Euer Koenig, jener tapfere Sterbliche, welcher sich selbst geopfert fuer das Heil seines Volkes, starb um die Dunkelheit zu bezwingen. Doch wisze mein Sohn, dasz die Dunkelheit dereinst nicht bezwungen, nur gebannt in muerben Stein und irdenes Grab. Denn niemand vermag die Dunkelheit zu vernichten... Frevel ward begangen und wider beszeren Wiszens ward sein irden Grab getragen dorthin, wo er hernieder gefahren, wo seine Macht noch immer weilet und er sich neuer Kraft erfreuen kann.

Daher wird der Tag anbrechen, dasz der Frieden vorueber, dasz die Dunkelheit erwacht in ihren Feszeln, sich erneut empor zu strecken an das Tageslicht, um zu suchen, was Tylon, euer Koenig ihm entriszen in ebenbuertig Kampf. Sie wird hinwegfegen ueber die Erde. Nichts wird sein, wie es einmal war. Verdorren wird das Land und nichts zurueckbleiben als Leid und Elend wenn der Tag des Erwachens vollendet.

All dies soll geschehen in vielen Zyklen erst, wenn du schon laengst gegangen von hier. Dir will ich all dies offenbaren. Schreibe es nieder, auf das jene, in solchen Tagen da Morbus dunkle Hand erneut nach Grenzbrueck greift, einen Schimmer Hoffnung finden moegen, eine Flamme, welche lodern moege in ihren Herzen, sie aufrufend zu widerstreben dem leichten, dunklen Pfad der Finsternis. Jene Weisen baldger Tage moegen erfahren, was dir wird nun zuteil und schoepfen ihre Schluesze und Entscheidungen, zu bannen das Leid von ihrem Land und Volk. So schreibe nieder, die acht Zeichen – Signien – die kuenden von Morbus‘ Erwachen und vom Anbruch jener Zeit der Schrecken und des Todes fuer euch Sterbliche. Die Zeichen, sie gilt es zu verhindern, um Morbus Macht und Dunkel zu bezwingen. Ein jedes der Zeichen sei gewogen in der Waagschale des Schicksals, zu weszen Gunsten es auch sei. Denn wir, die Unendlichen entschieden, es in eure Hand zu legen, was geschehen soll an jenem Schicksalstage, da sich die Heere von Licht 
und Schatten gegenueber stehen und Morbus das vielleicht zurueckerlangt, was er verloren. Und wehe, wenn zuviel der Signien verstreichen ohne Obacht, dasz sie Morbus Sieg erleichtern...!“

Und Maewon verstummte, denn seines Dieners Tochter den Raum betreten, zurueck aus dem Walde, erstaunt ueber das Antlitz des Fremden, welcher in das Haus ihres Vaters gekommen ward. „Wer ist dies Vater, sag es mir?“, ward die zierlich Stimm des Maedchens zu vernehmen.

„Meine Tochter, du bist sicher muede, ist etwas Besonderes im Walde geschehen? Dies ist Maewon, unser Vater von dem ich dir berichtet, doch stoere uns nicht laenger, denn Maewon Wichtiges mir zuteil laszen will. Geh nun zu Bett und schlafe ruhig und sanft.“

Ytalas Tochter, aber von Neugier gepackt, nicht zu Bett gehen wollte. Erst der strenge Blick ihres Vaters, sie zu Bett zu gehen bewegen vermocht. Doch dunkle Traeume sie plagten...

Maewon aber sprach zu Ytalas: „Nun schreibe, was ich dir offenbart, ich will morgen und weitere acht Monde und acht Sonnen zu dir kommen, dir zu offenbaren, was vergangen, gegenwaertig und zukuenftig ist. Diese Schriften sollst du uebersetzen und in das Buch aufnehmen. Sie kuenden von dem alten und dem neuen Bund. Doch wisze, niemand anderes soll dies erfahren bis zu dem Tage, an dem das Buch von Nutzen ist. Nenne die Schrift Thesys, denn solcher Art das Schicksal von den Unendlichen genannt. Wenn es vollendet, bringe es nach Anatheyn, wo die Weisen zu Rate sitzen. Meine erwuerdigen Diener. Sie sollen es hueten und es zu Rate ziehen, wenn es von Noeten.“

Ytalas dankte seinem Herrn und begab sich an sein Pult. Er schrieb, was Maewon ihm offenbart und uebersetzte die Schriften, welche der Herr des Schicksals ihm ueberlaszen. Von den acht Signien und der Waagschale des Schicksals, von der Rueckkehr der Dunkelheit... Als er beendet hatte, was zu tun ihm befohlen, ging er zu Bett und schlief...

Auch am naechsten Tage kam Maewon in das Haus seines Dieners Ytalas und als er das Buch sah, sprach er: „So ist es recht, nun hoere, was ich dir heute offenbaren will...“. Und Maewon sprach von der Vergangenheit, welche in Vergeszenheit geraten. Von der ersten Fehde der reinen Voelker gegen Morbus, von der langen Zeit der Verbannung, von Wyrdrak und Eschra, und Selhenas und Ildur, den obersten Lakaien des dunklen Herrn, des Ashgun Adûr, wie ihn die Reinen nannten, des Morbus, wie sein Name heut noch immer ist. Als er fast geendigt, sprach Maewon: „Sage mir, Ytalas mein Sohn, wo ist Deine Tochter?“.

Und Ytalas antwortete: „Nun Herr, ich habe sie frueh zu Bett geschickt, da ich euch erwartet, sie schlaeft nun, denke ich...“. Und Maewon sprach, das es gut sei.

Doch irrte Ytalas, im Blick auf seine Tochter. In dieser die Flamme der Neugier entbrannt, lauschte sie an der Pforte zum Nebenzimmer, wo der eigenartige Fremde mit ihrem Vater sprach. Dann als sie muede ward, fiel sie in tiefen Schlaf und traeumte von einem Platz im Walde, welchen sie kannte, nicht weit entfernt vom Haus ihres Vaters. Ein altes steinern Tor, gekroent von gehoerntem Haupte, aus alter Zeit zurueckgeblieben. Ihr Vater ihr verboten in die Naehe der Pforte zu gehen. Viel Schreckliches durch dieses verursacht ward, vor vielen Jahren. Doch eine Stimme nunmehr ihren Namen rief und alsbald sie in ihrem Traume dem Tore naeher kam, die Stimme um so lauter und klarer nach ihr saeuselte und seltsam vertraut die Stimme war... Dann verblaszte der Traum im ewigen Nebel des Schlafes und am naechsten Morgen sich Ytalas Tochter nicht mehr vermochte zu erinnern.

Am Tage begab sich Ytalas Tochter in die Waelder zu finden Beeren und Wurzeln, so wie es ihre Mutter ihr vor deren Tode noch gelehrt. Viele Zyklen ward dies nun schon vergangen, doch niemals sie vergeszen die Stimme der geliebten Mutter. Und war dies Trug doch Wirklichkeit, etwas an ihr Ohr gedrungen, was ploetzlich sie erinnern liesz an jene Frau und ihren Traum. Man rief ihren Namen ohne Unterlasz. Erst weit, dann nah und wie so vor dem Tore stand, welches der Vater ihr verboten, da blickte sie hinauf in die Augen des steinern Abbild des Gehoernten. War es, dasz er zu ihr gesprochen? Und als sie wieder hinabblickte, erschrak zufoerderst sie, dann eine Traene ihre Wang benetzte, denn in dem Tore stand, welch Wunder, die laengst Gegangene, Vermiszte, die sehnlichst oft Herbeigewuenschte in mancher schweren Stund, in welcher sie solch einsam gewesen. Und sie sprach: „Meine Tochter, erschrecke nicht, ich bin es. Ich weisz, lange ich nicht bei dir gewesen, doch geht es mir gut hier...“

Und Ytalas Tochter sprach: „Mutter, seid ihr es wahrhaft, wo seid ihr gewesen und warum habt ihr Euch solange dort versteckt? Ich habe euch so sehr vermiszt. Kommt, kommt, wir gehen nach Hause...“. Doch Ytalas Frau, des Maedchens Mutter entgegnete: „Nein, meine Tochter ich kann nicht von hier fort. Es ist gut hier, wo ich weile. Ewiger Frieden umgibt uns im Flusz der Zeit, ach koenntest du nur bei mir sein mein Kind... Ich habe dich auch vermiszt, doch konnte ich nicht frueher zu dir sprechen. Du hast dich sehr veraendert meine Tochter, herangewachsen und erblueht bist du, seitdem ich euch verlaszen muszte!““

Und das Maedchen erwiderte: „Mutter, aber sagt, warum hat Vater mir verboten hierher zukommen, wenn Ihr hier seid...?“Und das Bild hinter der Pforte entgegnete: „Oh, meine Tochter, dies hatte ich befuerchtet. Als ich von euch ging, hat Gram und Schmerz das Herz deines Vaters befallen. Und eine dunkle Macht von ihm Besitz ergriffen, fuerchte ich. Doch sag mir, ist jemand zu euch gekommen in den letzten Monden?“ „Ja“, entgegnete ihre Tochter, „...ein Mann in strahlend weiszem Gewande. Von Licht umgeben, doch Vater verbot mir bei ihnen zu weilen und schickte mich zu Bett. Ich sollte nicht vernehmen, was dort gesprochen, doch lauschte ich an Vaters Pforte. Ich weisz, dies ward ungehorsam, aber...“. „Nein, nein“, unterbrach sie ihre Mutter, „dies war gut, worueber sprachen sie? Dieser Fremde, er ist es, welcher von deinem Vater Besitz ergriffen hat und seinen Geist verwirrt... Erzaehle mir, worueber sprachen sie?“


Und so erzaehlte ihr die Tochter alles, was sie vernommen und was verborgen bleiben sollte, im Glauben es sei das Antlitz ihrer Mutter. Doch in Wahrheit war es ein Trugbild Eschras, der Herrin von Luege und von Illusion, der falschen Gefuehle sich bemaechtigend und Intrige spinnend seit Anbeginn der dunklen Zeit. 

Sie, die Verfuehrerin vernahm des Kindes Worte und so auch Morbus fortan bekannt, was Maewon ersonnen, um die Menschen des Koenigreiches zu erretten.

Als das Maedchen alles ihr erzaehlt, sprach Eschra durch das Trugbild ihrer Mutter: „Gut, meine Tochter, ich will unserem geliebten Vater und dir helfen. Lausche weiterhin, wenn der Fremde wiederkehrt heut Nacht, kehre morgen hierher zurueck und berichte mir erneut. Und hier noch Ding, was dich erheitern wird... Doch sprich nicht mit Vater ueber all dies hier!“. Und durch die Pforte reckte Morbus’ Antlitz ihre Hand und reichte dem Maedchen einen Kamm. Das Maedchen aber kehrte freudig ueber ihre Entdeckung zurueck. Verbarg den Kamm in ihrem Zimmer und lauschte des Nachts als Maewon in das Haus ihres Vaters kam.

Maewon, der Ewge, offenbarte Ytalas in den folgenden Tagen viele der Dinge, welche in diesem Buch geschrieben. Viele der Worte verwirrten Ytalas Geist und Sinn und zusehends mehr er schrieb, desto schwaecher wurde sein Geist und die Schrecken, welche Maewon ihm offenbarte, lieszen ihn verzweifeln. Doch jedes Mal zuletzt bestaerkte Maewon ihn, das Vorangesagte niederzuschreiben, ihn an die Wichtigkeit seiner Pflicht erinnernd. Und Ytalas schrieb...

Seine Tochter aber lauschte jede Nacht und auch sie verstand die Schrecken nicht von denen dort gesprochen. Und bestaerkt durch die Worte ihrer truegerischen Mutter, glaubte sie, das rechte zu tun. So kehrte sie am naechsten Tage zurueck zu der Pforte und berichtete dem Antlitz ihrer Mutter, welche in Wahrheit aber Eschra war.

Diesmal gab Eschra dem Maedchen am Ende ihrer Unterredung eine Puppe, wunderschoen und weich und zart. Und das Maedchen eilte, besorgt um ihren Vater und einzig den Worten ihrer Mutter Glauben schenkend, zurueck zu ihres Vaters Haus.

Dieser ward erschoepft, denn viele Seiten war das Buch gewachsen, wie ein Gebirge tuermte es sich vor ihm auf und jede neue Seite schien, als muesze er vom Fusze des Gebirges hinaufsteigen, um einen weiteren kleinen Kiesel dorthin abzulegen, um dann erneut hinab zu steigen und dies Treiben wieder und wieder neu zu vollfuehren. Nur Maewons Worte gaben ihm die Kraft...

Seine Tochter aber lag des naechtens wach in ihrem Bette, ihren neuen Kamm und ihre Puppe bewundernd mit den Augen eines Kindes, welchem die Welt zurueckgegeben worden ward. Am naechsten Morgen lief sie eiligst in den Wald, um ihrer Mutter zu berichten, welch seltsame Veraenderung sich zugetragen mit ihrem Vater, mal um mal der Fremde in ihr Haus gekommen. Diese vernahm offenen Ohres die Worte des Kindes und gab ihm eine Karaffe mit einem Trunke und ein schmuckvoll Kaestchen. Diese sollte sie aufbewahren und spaeter dann, wenn Eschra es ihr auftrug, verwenden ihrem Vater Heil und Linderung zu bringen. Das Maedchen tat, wie ihr geheiszen und kehrte in das Haus des Vaters bald zurueck.

So ging dies fort, noch weitere vier Monde und sieben Sonnen, ohne aber dasz die Mutter dem Maedchen weiteres Geding zuteil werden lies. Indes schuerte sie die Angst in dem Kinde, der Fremde werde ihres Vaters Geist verwirren und sich Ytalas bemaechtigen. Doch in Wahrheit hatte Morbus verstanden, dasz das Buch vernichtet werden muszte, bevor es vollendet oder derjenige, welcher er schrieb. Und so entsann er einen truegerischen Plan in seinem irden Grab.

Er schickte das Maedchen, welches ihm durch das Trugbild hoerig geworden, in das Haus Ytalas und gab ihm auf Ytalas den Kamm und die Puppe zu offenbaren, von dem Tranke zu kosten und dann des Nachts, wenn ihr Vater schlief das Kaestchen zu oeffnen. Was darin enthalten, werde das uebrige vollfuehren und ihrem Vater Heil und Linderung bescheren. In Wahrheit aber sollt dies Tod und Verderben bringen, wie noch aufzuzeigen im Folgenden sein wird.
So kehrte das Maedchen in das Haus ihres Vaters zurueck. Des Abends, betrat ihr Vater ihre Stube. Seine Tochter sasz an der Staffelei, welche sie von Vater einst geschenkt und uebte sich der Zeichenkunst. Ytalas sprach: „Sag meine Tochter, was zeichnest du?“. Und diese erwiderte: „Seht Vater, dies ist der ewge Flusz der Zeit, in welchem ich Mutter gesehen. Sie ist besorgt um euch und gab mir diesen Kamm.“, und das Maedchen kaemmte sich durch das Haar. Ytalas aber erschrak ob dieser Worte und dem Anblick des Bildes und des Kamms. Denn letzterer aehnelte dem seines geliebten Weibes sehr. Und als er das Bild erkannte, verstand er. Er entrisz der Tochter den Kamm und schleuderte das Bild zu Boden. „Ich habe Dir doch verboten, dorthin zu gehen, meine Tochter. Du widersetzt dich also meinem Wort?“

Dann rannte er hinaus in den Wald und vergrub den Kamm, dort wo er einstmals das Hab und Gut seiner geliebten Frau verbrannt und damit die Erinnerung an sie getilgt. Nicht unweit des alten steinernen Portals. Sechzehn Manneszchritt schweift der Blick des steinernen Gehoernten bis zu der Stelle in gerader Richtung. Und wiederum ward Morbus dunkler Plan vollbracht...

Das Maedchen aber verstand nicht, weshalb ihr Vater so erzuernt, betrat die Kammer des Vaters und las in dem Buche, welches ihr Vater diese Nacht vollenden sollte. Doch verstand sie die seltsam Zeichen und Bilder nicht. Nur ein Gefuehl in Ihr geboren ward, von Angst und Wut und Hasz.

Als ihr Vater zurueckkehrte, fand er das Maedchen erneut vor der Staffelei geschaeftig in der Zeichenkunst vertieft. Dann, ihr vorwurfsvoller Blick ihn traf. Gerade wollte er die Worte zu ihr beginnen, als er der Puppe in ihrem Arm gewahr. Und wiederum stieg Zorn in ihm empor. Nicht auf das Maedchen, sondern auf die dunkle Macht, die solch seltsam Spiel und Trug mit seiner Tochter trieb. Er entrisz ihr die Puppe und fragte sie eindringlich mit ernstem Blicke: „Ist dort noch etwas, was du im Wald gefunden, sag es mir meine Tochter...?“. Doch sie antwortete, wie ihr das Trugbild in der Pforte geheiszen: „Nein, mein Vater dies ward alles...“. Er aber eilte hinaus und tat, wie er schon vormals es getan, doch waren es diesmal zwanzig Manneszchritt entgegen dem Blick des steinernen Gehoernten.

Unterdeszen nahm sie, wie ihr geheiszen, den Trank zu sich. oeffnete also die Karaffe und benetzte ihre Lippen mit dem sueszen, dunklen Gift. Und dieses Tat sein uebles, frevelhaftes Werk, vernichtete den Geist des Maedchens, verwirrte ihre Sinne vollends, ohne dasz sie sich deszen haette erwehren koennen. Und Morbus dunkler Plan ward nun begonnen...

Der Vater Ytalas aber verzweifelt, ob all dieser Geschehnisze kehrte in sein Haus zurueck und sprach: „Weil du solch ungehorsam und solch unerzogen, sollst weilen du fuer dreizehn Tag in deiner Kammer und ueberdenken, was du verrichtet...“. So schlosz er die Tuere zu der Kammer, des irren Blicks der Tochter nicht gewahr, wie sie so da sasz vor dem Bildnis, das Kaestchen verschloszen in ihrem Schosz.

Der Vater kehrte zurueck in seine Stube, schrieb einige Zeilen an den Anfang dieses Buches. Dann legte er sich zur Ruhe, denn bald schon wuerde Maewon zurueckkehren, um das Werk, welches vor acht Monden und acht Sonnen begonnen ward, zu vollenden.

Und als Ytalas in Schlaf geriszen, seine Tochter das Kaestchen oeffnet und den Dolch mit irrem Blick entnimmt. Sie schreitet zu der Pforte, die verschloszen. Doch wie von dunkler Hand gefuehrt, ward diese offen. Sie tritt heran an die Schlafstatt ihres Vaters, ein letzter Blick, vielleicht ein Abschied. So stoeszt sie den Dolch hinab in Ytalas, in ihres eignen Vaters Brust. Die schwarze Klinge dringt leicht und tief in deszen Herz, der nun erwacht. Die Augen seiner eignen Tochter fahl und leer, doch laechelt er, bevor er erst begreift, was grad geschehen. Kaelte ihn umgreift und nur die dunkle Stimme einer Macht vermag er zu vernehmen, die nun wahrhaft von seinem Geist Besitz ergreift...

„Ytalas, ach Ytalas, nun hat er euch doch noch verlaszen... Und Maewon, glaubtest du, die Sterblichen wahrlich erretten zu koennen... Wisze nun, dasz mein noch immer hier verweilt. Der Plan nun vollendet, da ich acht Siegel um das Buch geschmiedet, die eines Menschenlebens wohl beduerfen, um gebrochen zu werden. Ein Bann der Zeit moege diesen Raum umgeben, hinabgeschleudert in den Schlund der Ewigkeit. Auf dasz ein Sterblicher niemals erfahre, was hier geschrieben steht, ohne dasz sein Geist nicht mag verderben, wie der deines Dieners Ytalas. Der Kern voll Licht, soll es auch sein, dasz ich ihn nicht zu loeschen vermag, doch wenn niemand jemals ihn erfahren mag, ist es doch ebengleich...“.

Und so geschah es. Acht Siegel umgaben fortan die Thesys, die verbannt durch Morbus dunklen Fluch in den Schlund der Zeiten. Jedoch wenn das Vergangene wiederholt und der Risz dadurch geeint, dann mag ein Sterblicher das Buch erretten aus dem ewgen Flusz der Zeit. Doch wisze, dasz was geschehen kannst nicht veraendern, auch wenn dein Herz und Geist dies vielleicht wuenscht... an jenem glorreichen Tage, an welchem die Thesys zurueckkehrt in die Welt der Sterblichen... 

Im Anfang das Ende und im Ende der Anfang... Das Schicksal legen wir in Maewons Hand... 

\newpage

\subsection{ Von Morbus Lakaien}

\yinipar{D}Der Schattenprinz hat in Zahl und Gestalt nur all zu viele Gesichter, und so ist sein Antlitz, dasz wir, die Sterblichen manchmal zu erblicken glauben, stets nur eine Taeuschung. Man nennt sie Eschra.

Sie verachtet die Menschen nur all zu sehr, verhoehnt sie fuer ihre empfindsamen Seelen, vor allem aber ihre schwachen Geister. Sie spielt mit den Sterblichen und prueft sie, wie es die Unsterblichen zu tun pflegen. Einer Richterin gleich will sie das fleischliche Gezuecht fuer seine Unvollkommenheit abstrafen. Gib ihr nur nicht Gelegenheit dazu! Ihr Spiel ist gnadenlos und tueckisch, doch hat man es nicht von vorn herein verloren. Sie will deine Dummheit vor dem Loesbaren verspotten, nicht dich vor das Unloesbare stellen. Das waere ihr zu einfach.

Solltest du ihr erliegen, so wird ihr Lug und Trug dein Schicksal bald besiegeln. Du kannst sie fuerchten, vor ihrem Spiel fliehen wo du es witterst, doch wird diese Schwaeche ihr den Weg in deinen Geist eroeffnen. Das Einzige aber, was du tun sollst, ist sie durchschauen. Frisz nicht ihre Luegen, sondern speie ihr die Wahrheit entgegen. Verwundere nicht ob ihrer Erscheinung, sondern reisze das Trugbild dar nieder. Wenn dein Verstand scharf ist, und weder die tolle Kuehnheit des uebermuetigen, noch die erbaermliche Feigheit des Mutlosen dich beherrschen, dann allein wirst du gefeit sein vor ihrem Spiel.

Weichst du aber ab von diesem Pfad in die eine oder andere Richtung, so wird es ihr ein Leichtes sein dich und die deinen der Verdammnis preiszugeben. Denn tust du einen Fehltritt, so wird sie ihn ohne Zaudern ausnutzen. Wisze: Morbus Antlitz sieht alles! Beweise ihr, dasz der Verstand des Menschenvolkes scharf und durchdringend ist. Zeige ihr, dasz die Schoepfung der Goetter zwar unvollkommen doch stark genug der Finsternis zu trotzen – im Dunkel zu sehen. Alleine so kannst du sie besiegen, und dir fuer kurze Zeit Ruhe vor ihr erkaufen. Sei dir doch gewahr, dasz ihr Blick schon bald wieder auf dir ruhen mag, um zu vollenden, was dir bisher erspart blieb.

\newpage
\subsection{ Der innere Feind}

\yinipar{W}ir sind der innere Feind.
Unsere Zahl und Namen sind mannigfaltig.
Unser Herr ist Morbus, deszen Namen ebenso mannigfaltig sind.
Wir dienen ihm im Verborgenen, unter Euch.
Denn wir sind Euresgleichen, Bruder, Schwester, Vater, Mutter, Sohn oder Tochter.
Menschenkinder waren, sind wir. Doch den Ewgen gleich werden wir sein, denn Er hat uns erwaehlt.

Wir leben in euren Staedten und Doerfern, unter euren Daechern und in euren Mauern.
Wir sehen, hoeren und fuehlen fuer den Feind.

Unsere Zeichen sind euch wohlbekannt.
Jeden Tag, jede Nacht und jeden Augenblick liegen sie vor euch, dennoch seht ihr sie nicht, denn blind macht der Hochmut.

Langsam mag unser Tagwerk sein, beschwerlich und voll Muehen.
Dann aber kommt der Tag an dem es vollendet und ihr nicht mehr erwacht aus jenem Alptraum, welchen wir fuer euch ersonnen.

Sucht und findet uns.
Sucht tief in euch. Denn Wir haben euch schon gefunden.

Die Tage der Verderbnis sind nah, wenn er uns auferstehen laeszt und ihr vergeht im ewgen Strom der Qualen.

Sucht und findet uns.
Wir sind der innere Feind.

\newpage

\subsection{ Von der Art und dem Wesen der Eszenz}

\yinipar{S}o geschah es in jenen alten Tagen, als ich noch am Hofe des Protautian meinen Dienst als deszen Leibarzt versah, dasz dieser mich bat ueber die Art und das Wesen der Eszenzia zu forschen, da es geschehen, dasz seine Tochter, die liebreizende Anabel, von jenem uebel ergriffen ward. Ich mutmaszte, das ein reisender Haendler, welcher Weine aus dem Westen des Reiches an den Hofe gebracht hatte, einen Stein bei sich gefuehrt, kristallen und von roter Farbe, welchen er Anabel vermacht, fuer jenes Leiden verantwortlich ward.

Seit jenem Augenblicke, da des Koenigs Tochter sich mit diesem Stein umgeben, ward ihr ganzes Wesen veraendert, von abschauderlicher Pervesio et Absurdia. Ihr Geist schien entfleucht und ihr Blut in solche Wallung geraten, dasz nur die staerkste meiner Arzneien sie beruhigen konnte, wenngleich hierdurch sie in einen bedauerlichen Zustand versetzt.
Doch soll dies Schicksal der armen Anabel den Leser nicht taeuschen, noch all zu sehr in seinen Bann ziehen, als des wahren Inhalts Kern meiner Worte meine Erforschungen auf dem Gebiete der dunklen Eszenzia sind.

So ich mich nicht nur Medicus nannte, sondern auch der Alchemie ohne all zu bescheiden zu sein, in mancherlei Hinsicht meisterlich verbunden, nahm ich mit aller Vorsicht jenen Stein in meine Obhut und studierte ihn vielleicht zehn oder zwanzig meiner Jahre und schien es, dasz der Stein sein daimonisches Wirken nicht auf mich zu uebertragen mochte, welch raetselhafte Cautio dahinter auch mal liegen.

So mag der ehrenwerte Leser wiszen, dasz die Ars Alchemiae in aller Herren Laender gleichsam einig, mal mehr mal weniger der wiegenannten Elementas, als grundsaetzliche Constructio der Welt kennt. Oftmals ist es die Aera, das Aqua, die Terra, das Ignis und der Aether, manch einer mag anderes hinzufuegen wie es ihm beliebet. Und wie dem iszet, so gilt es, dasz jedem dieser Elemente gleichsam nachgesagt, dasz sich Wesenheiten nicht angulesque, noch daimonisch, ihnen widmen, wie Waechter oder Incorporati, sofern und so weit da nur genug Fors Elementari bestehet.
So ist es aber, dasz gleichsam dieses unseres Bildes Vorstellung pervertierend, auch der, den wir in unseren Tagen Schattenprinz heiszen, eine eigene widernatuerliche Natur, so paradox dies klingen mag, geschaffen, als er vor vielen Aeonen auf diese Erde stuerzte. Hiervon sollen meine Worte kuenden und moegen sie auch weder zur Freude taugen noch von Nutzen sein. Allein Erkenntnis liegt hier vor euch, ehrenwerte Leser.

Also ergruendete ich jene Natur des Dunklen, jenes verzerrte Spiegelbildnis unserer Welt, geschaffen, unvollkommen und hierdurch unserem Geist nicht eingaengig. So scheint es als bestuende die Eszenzia, welche ich als den Kern des Dunklen will bezeichnen aus drei Substanzen, wenn ich dieses Wort benutzen darf. Faelschlich waere es wohl von Emotiones zu sprechen, auch wenn dies nahe und verfuehrerisch ist, eben weil wir sein Wesen nicht begreifen. Und eben deshalb will ich von Elementen der Eszenz hier reden.
Und wie wir Waechter der Elemente kennen, so sind auch diese drei Ursubstanzien wohl Wesenheiten innewohnend, die manch Unwiszender vorwitzig schlicht als Daemonie bezeichnen wuerde. Doch waere hiermit nicht gewonnen.

Die drei Elemente, und hier liegt ein erstes Zerrbild wohl, will ich bezeichnen. Ad primum die Avaritia, die oftmals aehnlich der Aera oder dem Ignis sich formt und fluechtig ist. Ad secundum die Diszonancia, welche aehnlich dem Aqua sich verschiedentlicher Formen bemaechtigt und teils das Eine und teils das Andere ist und dennoch hier auf unsere Natur bezogen wohl voll Harmonia. Ad finitum die Lethargia, eine Substanzia welche von ungeheurer Schwere, wie wohl die Terra sein kann. Avaritia, Diszonancia et Lethargia ad conclusio.

Bemerkenswert ist in diesem Contextum, dasz sich eine aehnliche Teilung der widernatuerlichen Formen auch bereits Climorhas These ueber Verstuemmelung der westlichen Waelder wieder findet, so dasz ich mich hier bekraeftigt fuehlen will. Da wir nun also die widernatuerlichen Elemente naeher kennen, gilt es uns naeher zu beschaeftigen mit jenen Creatures, welche diese Elemente incorporieren und im Kleinen wie im Groszen in uns wirken, kommen wir mit der Eszenzia, dem Urstoff, all zu sehr zusammen.

Hier moegen wenige Worte nur geschrieben stehen und ich will zitieren aus jenen alten Riten des Ratmath An Burgia, Hohepriester Acrulons zu Zeiten Uspians von Limest. Dieser spricht also in seiner Prophezeiung von der Rueckkehr der dunklen Eszenz, wie folgt. Und da waren drei Wesen, welche sein Innerstes verkoerperten. Jedes von solch schauderlichem Anblick, dasz ich hiervon nicht will kuenden, sondern nur die Worte meines Herren wiedergeben, der sich dem ueber widersetzte.

Als das erste Wesen ihm entgegentrat, fluechtig wie die Luft, da sprach er. Dies ist das Eine, welches zum Licht strebt und doch niemals das Licht sein kann. Wisze, es ergreift Besitz von deinem Verstand. Verinnerlicht sich in dir. Wird dein Blut und dein Geist. Nimmt alles in Besitz von dir, auf das auch du zum Licht strebst, welches du niemals erreichen kannst. Denn das Licht ist Acrulon und den Ewgen vorbehalten.

Und so gleich erblickte er die zweite abscheuliche und absonderliche Wesenheit und er sprach. Dies ist das Andere, welches in sich vereint, was nicht zu vereinen. Es zerstoert deine Sinne, da du es nicht begreifst und fuehrt dich an die Abgruende irdischen Verstandes. Halt dich fern von seinem marternden Klang. Es ist Ende und Anfang in einem. Schwarz und weisz.

Da aber erschien die dritte Brut und mein Koenig erkannte. Dies ist das Letzte, das sterbende Leben und das lebende Sterben. Es steht am Ende aller Wege. Und wenn es auch still zu stehen scheint, so ist es ueberall und niemand vermag ihm zu entfliehen, der all zu nah an es gereicht. Wir werden es sein, und wenn auch nur fuer kurzen Augenblick, da die Goetter mit uns Erbarmen haben und uns erloesen.
Und mein Koenig erkannte deszen schwarzen Unholds ganzes Wesen. Dies sind die Elemente der dunklen Natur, welche sich vor uns verkoerpert. Wehe denen, die dies eines Tages erkennen mueszen. Er ist sie im Ganzen. Pars pro toto. So denke ich, genuegt dieses Zitat um dem ehrenwerten Leser die Incorporati der Elemente der Eszenz zu versinnbildlichen. Und wie es Lehre ist in der Ars Alchemiae will ich jedem ein Zeichen zubilligen. Ich will beginnen mit der Avaritia und seinem Waechter, welcher zum Licht strebt, doch niemals das Licht sein kann. Dies sei fort an sein Zeichen.

Von den anderen will ich spaeter noch schreiben. Doch brauch mein Geist, ob dieser Abschauerlichkeiten Ruhe und Labsal.

\newpage

\subsection{ Von den acht Signien}

\yinipar{U}nd ich sah acht Signien, ein jedes grauenhafter als das andere und die Dunkelheit ward erwacht, als das letzte der Signien sich offenbarte. Und Pest und Leid und Tod kamen erneut ueber die Welt und Verzweiflung ueber die Menschen. Und selbst der Himmel zitterte unter der Last, die auf die Kinder der Unendlichen gekommen ward.

Und ich sah zwei gewaltig Waagschalen in den ewigen Hallen der Unendlichen. Die eine ward aus purem Golde, mit edlen Steinen verzieret und so rein und klar geformt, wie nur die Unendlichen es vermoegen. Die anderen aber ward aus Pech mit Schwefel ueberzogen, so als sei sie eben aus dem Schlund der Hoellen selbst entsprungen.

Und Maewon der Eine und ewige Hueter des Schicksals, Beherrscher der Magie nahm die Signien und legte sie in die Schalen, je danach ob sie gutes oder schlechtes verheiszen, ob sie geschehen oder nicht.

Und Acrulon, der ewige Goetterfuerst, Sohn des einen Lichts, Herrscher ueber die himmlisch‘ Ebenen liesz seinen Blick schweifen auf die Schalen und die Erde. Und er sah, wie seine Kinder vergingen in der Flut der Finsternis, als die Pest erwacht. Dunkelheit umgab sein Werk und sein Zorn ward unermeszlich grosz und so sprach er zu Maewon.

Maewon, Hueter des Schicksals, Beherrscher der Magie, sende den Kindern Rat, ob dasz sie eingedenk werden der Gefahren, welche die Signien, die du erschaffen in unendlich weisem Entschlusze, in sich tragen. 

Dies will ich tun mein Koenig, antwortete Maewon Doch wisze, dasz schaendlicher und truegerischer Plan ersonnen von der Dunkelheit, um der Signien habhaft zu werden und ihr Werk endgueltig zu vollenden. Lang schon hat sie sich von uns abgewandt. Und unsere Kinder nicht vermoegen der wahren Bedeutung zu erkennen, denn allzu gering ihr Geist und ihr Gedanke ist gegenueber der Finsternis.

Da trat Myrna, liebende Mutter, Herrin der Gerechtigkeit, mildtaetige Helferin in der Not in den Kreis. Und Traenen sie vergoszen, als sie sah, was mit den Kindern mocht geschehen. Und sie sprach zu ihrem geliebten Maewon.

Maewon, mein Mann, so sende den Menschen drei Siegel zu jedem der Signien, auf dasz sie das Nahen derselben erkennen und danach handeln moegen, um ihr verderblich Schicksal abzuwenden.

Und so tat Maewon, wie ihm geheiszen, denn die Liebe zu seiner Frau und seinem Koenig waren unermeszlich, wie auch die Liebe zu seinen Kindern, welche von der Dunkelheit bedroht.
Und zu jedem der Signien er drei Siegel geschmiedet, welche kuenden und offenbaren des Schicksals nahen Lauf. So seien es achtmal drei der Siegel und je ein Signum bis die Finsternis dem irden Grab entsteigt... 

\newpage

\subsubsection{ Vom ersten Signum}

\yinipar{A}lso kam ich zu dem goldenen Kind, welches ruhte auf seinem einfachen Lager und da es nicht erwachte, legte auch ich mich zur Ruh, denn lang war der Weg gewesen. Ploetzlich aber, mitten in der Nacht ich erweckt von leiser, sanfter Kindesztimm‘ nah an meinem Ohre. Und da ich aufblickte, sah ich das Kind an meinem Bette stehen und es sprach zu mir:

„Koenig, wieder bist du zurueckgekehrt, um zu erfahren, was des Schicksals Faden sei. Doch wisze, dasz was du nun hoerest nicht fuer deine Sinne, sondern fuer die deiner Kindeskinder ist bestimmt. Denn wenn die Dunkelheit naht, wird Zweifel obsiegen und Wahn herrschen in ihrem Geist und Herz...

Das Buch des Schicksals ward mir heut Nacht eroeffnet, und das erste Siegel des ersten Signums ward gebrochen.

Ich sah einen Weisen, einen Alten, gut in seinem Herzen, doch von Wiszbegier geplagt. Daher er geschloszen finstren Pakt mit jener Pest, die auch dein End schlieszlich wird besiegeln. Lange Zeit er seines Lebens gefristet, bis zu jenem Tage, da der letzte Zug vollfuehrt. Und er zurueckkehrt an den Ort des Paktes zu ueberlaszen seinen Geist und seine sterblich Huell, jenem welchen sie ‘Morbus Mund‘ nennen werden, dem steinern‘ Prophet der ewgen Nacht. Sein Kampf vergebens, wird fortan er nicht mehr Augur geheiszen, sein Name ‘Morbus Aug‘ soll sein... dem nichts entgeht, was nah noch fern geschieht. Und Pesthauch wird ihn kuenden, seinen Namen, in alle Gegenden der Welt...‘Morbus Aug‘..

Sodann das zweite Siegel des ersten Signums ward zerbrochen und mir ward offenbar...
Ein schaendlich elender Verraeter, Vertrauen nutzen, Schwuere brechend, die Splitter aneinander schmiedet. Die Splitter jener Kron, die ihr zerschlagen werdet, mein Koenig, auf dem Felde der Ehre, wo ihr sterben mueszt. Lange Zeit euer Blut in Vergeszenheit geraet, bis wieder gefunden durch den weisen Schlusz derer, welche zu Rate sitzen in Anatheyn. ueber dies einer ebenfalls der Finsternis anheim gefallen, obgleich sein Sinn ihn deszen truegt, so ist vorausbestimmt ihm grosze Stunde, in welcher gar wird er entscheiden, ob ruhmreich Sieg, ob schaendlich Niederlage soll zuteil werden, den seinen, welche mit ihm sind und ziehen.

Schlieszlich das dritte Siegel des ersten Signums ward gebrochen und ich sah...
Die Thesys, gefangen bald im Schlund der ewgen Zeit vermag entrinnen ihrem duesteren Verlies, wenn gleich auch ungewisz, ob gut ob boese ihrer habhaft werden. Und Mensch mag ziehen seinen Rat und Schlusz aus diesem Buche, welches Ytalas, der noch geboren wird, ermordet durch Eschra - ‘Morbus Antlitz‘ sei ihr wahrer Name - feige Hand, geschrieben auf Maewon‘s weisen und unerschuetterbar Entschlusz.

Und dies sei das erste Signum des Erwachens...
Ein Stern wird vergehen am Firmament und stuerzen hinab, tief in die Waelder des verdorben Landes, wie einstmals die Finsternis gar selbst geborn am Anbeginn der Welt. Die dunkle Krone wird dann geschmiedet oder nicht, je ob Dunkel oder Licht obsieget. Sie zu vernichten einzig und allein das Masz der Dinge sei und schwer soll wiegen ein Verlust in der Waagschale des Schicksals. Und ‘Morbus Hand‘ wird wiederkehren, sie ist sein Feldherr, welcher reitet voran, die dunklen Heere in die Schlacht der Schlachten fuehrend. Und niemand vermag ihr zu widerstreben fuer lange Zeit... ihr, ‘Morbus Hand‘...“.

Und da ich diese Worte des Kindes vernommen, ging ich fort und liesz sie niederschreiben, um sie meiner Kindeskinder zu verkuenden, so wie es verheiszen worden war. 

\newpage

\subsubsection{ Vom zweiten Signum}

\yinipar{U}nd so schlief ich einige Zeit um Ruhe und Trost zu finden. Aber im Schlafe erschien mir das goldene Kind erneut und es sprach zu mir. Koenig, ich bin gekommen um dir vom Nebel zu berichten, welchen ihr Zukunft nennt. Ich schritt hindurch und folgte einem langen, schmalen Pfade bis zu einer Wiese. Dort sasz ein Mann, ein Greis und ich ging zu ihm. Er sagte zu mir: Hoere Kind, was mir die Himmlischen offenbarten, hoere wohl und gehe zu deinem Koenig und kuende ihm hiervon.

Und das erste Siegel des zweiten Signums ward gebrochen.

Ich sah einen Edlen, einen Reichen. Einstmals jedoch allein von einfacher Geburt. Er herrscht ueber das Land, was dazumal war Mendreth geheiszen. Doch toericht ist er und dumm. Ein Narr in seinen Mauern und in seinem Gewande. Denn in den Tagen des ersten Signums wird er in das Land der Gruenhaut ziehen, wird sie reizen, wird sie fordern bis aufs Blut. Mag seine Klinge auch noch so stolz und tugendhaft er fuehren. Nichts wird es ihm nutzen gegen den wilden, neu entfachten Hasz. Sein Stolz wird ihm sein Ende bald bereiten und Schande mag auf seinen Namen kommen, er, der so viel Leid nur seiner Habgier wegen hat herauf beschworen.

So dann das zweite Siegel des zweiten Signums ward zerbrochen und mir ward offenbar …

Ein alter Groszer, ebenfalls in Ketten, wie der Schattenprinz, wird sich seiner Feszeln entledigen. Sein einstges Ansinnen hat er aufgegeben. So gelangt er zu wahrhaft groszer Macht. Sein Zorn wird ueber jene kommen, die ihn so lang zum Narren hielten. Glaubten sie doch, dies ewiglich zu vermoegen. Er wandelt zwischen den Welten, keine Mauer und kein Schild sind ihm ein Hindernis. So wird er jagen, wie er selbst einst ward gejagt.

Schlieszlich das dritte Siegel des zweiten Signums ward gebrochen und ich sah …

Jener, der zu oberst in Anatheyn zu Rate sitzt, wird seinen Platz verlaszen, da der andere naht. Sein weiser Ratschlusz wird fortan nicht mehr gehoert in jenen Hallen. Doch finsteren Plan hat er ersonnen seine einstge Macht wiederzuerlangen und Rache zu nehmen. Wieder andere vormals wird er in das versunkene Land gehen und sich zu Nutzen machen, was einstmals der andere erschaffen hat in Liebe zu jener ungeliebten Schwester im hohen Bunde.

Und dies sei das zweite Signum des Erwachens …

Ein gruener Sturm wird kommen. Er bricht vom Osten her ueber das Land. Ungestuem, ungezuegelt und erfuellt von unbaendigem Hasz. Nichts wird dem Gruenblut vorerst Stand halten, wenn es aus den tiefen Mooren und Suempfen seiner Heimat den Doerfern und Staedten der Menschen entgegenstrebt. Und dem Schatten solcherart ein neuer Buendner ungewollt angetragen, da der Feind des Feindes ein Freund ist. Wehe den Menschen!

In diesen Tagen aber, voller Tod und Wut, wird das guelden Antlitz des Goetterdieners in eurer Kind Hand gelangen oder nicht, je ob Dunkel oder Licht obsieget. Die golden Fratze des zwiegespaltenen Betruegers zu laeutern einzig und allein das Masz der Dinge sei und schwer soll wiegen ein Verlust in der Waagschale des Schicksals. Und jenes verfluchte Fleisch wird jener Goetzendiener rufen. Jenes, das nicht vergehen mag im Sturm der Zeit, gefeszelt einstmals an Myrns einsam Traen. Und so jagt jene Brut erneut die Kinder des Goetzen. Jene, welche selbst der Tod fuerchtet und welche man die Lok Ashtar nennt.

Dann aber mein Koenig ging ich fort, um zu euch zu gelangen. Und ich verstand, dasz der Alte nicht blind war, denn vieles hatte er gesehen, was mir verborgen gewesen bislang. Und als ich erwachte war niemand um mich.

Ich aber liesz die Worte des Kindes niederschreiben, um sie meinen Kindeskindern zu verkuenden, so wie es verheiszen worden war.

\newpage

\subsubsection{ Vom dritten Signum}

\yinipar{U}nd solcherart stand ich auf den Zinnen meiner Burg und blickte beharrlich hinab in die Brandung des Ozeans, denn Trauer hatte mein Herz umgeben, ob all jener schicksalhaften Kunde. Doch wie das Rauschen des Meeres in meinem Herzen widerschallte, ward es, dasz Raum und Zeit um mich vergingen, gleichsam dem Falken, welcher ueber das Land fliegt. Und unten auf den Klippen erblickte ich das Kind, der tosend Gischt und tobend Brandung trotzend und seine Worte drangen an mein Ohr.

’Koenig, hoere was mir widerfahren in den Tagen, die vergangen seit ich zuletzt bei euch war. Als ich schlief in den Weiden meiner Heimat, drang ein Wind zu mir aus fernem, fremden Lande. Und als ich erwachte, vernahm ich tausend tosende Stimmen, gleichsam den Seelen der Verdammten, die in ewiger Kaelte gefangen sind. Und Schmerz durchdrang mich und ich fuehlte ihr Wehklagen. So sprachen sie… 

’Hoere Kind, was uns die Ewgen, die uns verdammt, offenbarten, hoere wohl und geh zu Deinem Koenig und kuende ihm hiervon. 

Das Buch des Schicksals ward heut uns offenbar, und das erste Siegel des dritten Signums ward gebrochen.

Die Sonne, die dem Schlueszel dient und der Baum, auf deszen Wurzeln einstmals die Bruecke ward errichtet, zueinander finden, doch noch im Verborgenen. Ungesehen und ungehoert von denen, die ihnen dienen und denen, denen sie ihren Dienst erweisen und ihre Pflicht erfuellen. Und das versunkene Land mag dann seinen Platz beanspruchen. 

Sodann das zweite Siegel des dritten Signums ward zerbrochen und uns ward offenbar...
Eine Grosze wird fallen aus fernem Reiche. Dort sitzt sie zu Rat auf jenem jung erbauten Throne mit ihresgleichen. Durch den Krieg wird sie geadelt. Eine Schneiderin war sie einst, bis ihr Volk sie ruft. Trotzen wird sie dem verfluchten Fleische, wie auch der Versuchung durch ungekannte Macht. Und grosz sind ihre Entbehrungen in unserem Lande. Ein Seelenband lang schon zertrennt und doch ungebrochen, so wird sie eingehen in der Ewgen Fluten Strom. Ihr Tod aber ungesuehnt, denn der Buendner aus fremden Landen den Pakt gebrochen und solch des Moerders’ Kling entkommen. Serenia wird einer ihrer Namen sein. 
Schlieszlich das dritte Siegel des dritten Signums ward gebrochen und wir sahen...

Fernab von jenen Feldern, wo die Schlachten toben, ward dem Schattenprinz ein unerwartet Geschenk dargetan. Dorthin in das Land des gueldnen Roszes, wo zuvor ein Bund von Menschenkindern ist geschloszen und jene von reinem guten Herzen einem falschen Freund getraut, kehren jene zurueck. Hinter gebrochenen Mauer finden sie den Pfad, geschaffen durch dem Wahn Anheimgefallenen, der sie mag leiten zu weiterem schicksalhaften Ort. Ein lichter Schein dem Verstaendigen den Pfad erleuchtet.

Und dies sei das dritte Signum des Erwachens...

Und die Stimme der Verzweiflung wird dreimal dann erschallen oder nicht, je ob Dunkel oder Licht obsieget. Ihren Klang zu hindern aber soll allein das Masz der Dinge sein und schwer soll wiegen ein Verlust in der Waagschale des Schicksals.
Und ‘Morbus Mund‘, der des Schatten Botschaft kuendet, wird niederkommen ueber das Land der Reiter, entledigt seiner steinern’ Feszel. Und seine Worte werden zu Tod und Verderbnis in den Seelen jener, die es wagen ihm zu trotzen. Und ihre Herzen, ergriffen von der Hoellen Botschaft, werden dunkel werden und seinem truegerischen Ansinn hoerig. Wehe denen, die solch erdulden mueszen…’

Dann verstummten die Stimmen und ich eilte zu euch. So wiszt ihr nun, was euren Kindeskindern mag widerfahren in fernen Tagen, jenseits der Schleier der Zeit.’ 

Dann erklomm die Flut die Klippe, auf welchem das Kind sich niedergelaszen und als die Waszer den Stein aus ihren Feszeln entlieszen, ward dort nichts, als der vom ewgen Meer geschliffene Fels.

Und da ich des Kindes Worte vernommen, entbrannte mein Herz erneut voll Trauer und voll Zorn. Doch liesz ich sie niederschreiben, um sie meinen Kindeskinder zu verkuenden, so wie es verheiszen worden war. Und darob unseres Geschlechts Vergaenglichkeit stets zu gedenken. 

\newpage

\subsubsection{ Vom vierten Signum}

\yinipar{U}nd es kamen die Tage, da ich zur Jagd ausritt, um die Trauer ob der kommenden Tage aus meinem Herzen abzustreifen. Da ritt ich also mit den meinen und mein Bogen ward gespannt. Und als wir dunklen Wald betraten, da sah ich einen Wolf, der scharrte mit den Pranken, bereit ein hilfloses Geschoepf gar niederzustrecken. Und doch vermochte mein Pfeil des Tieres Hals zu durchschlagen, auf dasz er schmerzverzerrt zu Boden sank. Da erst erkannte ich des Wolfes vermeintliche Beute. Das Kind aber trat vor mich und ich erkannte es wieder und wie ich es so anstarrte, da erkannte ich neben ihm den leblosen Koerper eines Alten.

Und das Kind sprach zu mir. Koenig, so hast du mir das Leben geschenkt und dies will ich deinen Kindeskindern vergelten. So wisze, was ich erfahren, als ich in das Auge des Toten geblickt. Denn die Toten kennen alles, das Vergangene, das Gegenwaertige und das Zukuenftige. Ihre Augen sind wie ein Spiegel. Sie offenbaren dir ihr Schicksal und das der anderen, wenn man ihren Blick zu lesen versteht. So hoere, was ich sah …

Das Buch des Schicksals ward dem Toten offenbart und das erste Siegel des vierten Signums ward gebrochen …

Des Schlueszels Waechter wird gerichtet in jenen Tagen. Seine zwei Antlitze nur wenigen bekannt. Doch faellt kein Sterblicher das Urteil. Mag sein, dasz andere durch seine Zeilen stuerzen werden in den Tagen, die dann kommen.

So dann das zweite Siegel des vierten Signums ward zerbrochen und dem Toten ward offenbart …

Der Schwan wird heimkehren. Reue und Busze zeugen seine Worte und Vergebung erbittet er von jener, die Illustris einst vereinen wird. Mit geteilten Rufen wird er willkommen geheiszen und der Krone einen Augenblick des Zweifelns abtrotzen. Und unter ihren Fluegeln vereint sie bereits Legionen, namhaft und auch niedrig.

Schlieszlich das dritte Siegel des vierten Signums ward gebrochen und ich sah in seinen Augen …

Ein Bollwerk wird errichtet werden wider den Schatten. Umstritten ist es unter jenen, welche in Anatheyn zu Rate sitzen. Wer Recht behaelt, wird das Schicksal offenbaren. Doch ist das einer, der mit edler Zunge spricht. Gleichwohl besudelt sind seine Haende. Und so wird er ersuchen, den Stein, welcher dem Lichte Acrulons geweiht, zu verderben. Deinen Kindeskindern mag gelingen, sein Tun zu unterbinden oder nicht.

Und dies sei das vierte Signum des Erwachens …

Wenn eines dunklen oder lichten ewge Macht den Sitz des Auges bald durchflutet, durch Menschentat entfeszelt, des dunklen Prinzen Blick, gen Nord sich wendet oder nicht, je ob Dunkel oder Licht obsieget. Sein Lid zu versiegeln aber soll allein das Masz der Dinge sein und schwer soll wiegen ein Verlust in der Waagschale des Schicksals.

Und Morbus Auge, der Augur, der das Vergangene, das Gegenwaertige und das Zukuenftige in sich vereint, hat ersonnen truegerischen Plan seit Anbeginn der Zeit. Sein Augenlicht offenbart dem dunklen Prinzen Diener Kindeskinder Plan und soll alles Tun und Streben ihm zu trotzen vergeblich sein. Und die, welche dir nachfolgen, werden erkennen, dasz die Zeit angebrochen, da seine Heere ueber ihr Land kommen. Dann, wenn die Hoffnung erlischt und alles entzweit wird. Wehe denen, die solch ertragen werden.

So wiszet ihr nun, was euren Kindeskindern mag widerfahren in fernen Tagen, jenseits der Schleier der Zeit. Dies sind die vier Signien der Lakaien. Die Dienerschaft ist dann erwacht.

Und wie das Kind gesprochen, da stieg ein tiefer Groll in mir empor, ob all dieses uebels. Das Kind aber schaute mich an und fluesterte: Lebe um zu geben, gebe um zu leben. In anderen Gewaendern werde ich dir fortan begegnen und dir kuenden von den letzten Signien. Lebe um zu geben, gebe um zu leben. Und wie das Kind gesprochen, da sah ich, wie es mit seiner Hand die Bestie beruehrte und diese sich erhob, als sei sie vom Tode noch einmal errettet. Und alsbald verschwand das Tier im dichten Gestrauch, da ich meinen Maennern Einhalt geboten. Das Kind laechelte und ich sah, dasz ich nach seinen Worten gehandelt, und es verschwand auf der Stelle. Und als ich den Toten dort ansah, da erkannte ich mein Ebenbild in ihm, in jenem alten Mann. Und ich erschrak und taumelte. Als ich erwachte aber, berichteten meine Getreuen, dasz ich gestuerzt, als ein Wolf mich angegriffen. Als ich aber von dem Toten ihnen kuendete, da sprachen sie, dasz da kein Leichnam gewesen sei, nie zu keiner Zeit.

Ich aber verstand, wer hier gelegen. Da lies ich die Worte des Kindes niederschreiben, um sie meinen Kindeskindern zu verkuenden, so wie es verheiszen worden war.

\chapter{Lexikon der Magiohermetischen Terminologien}

Abraxasgemme: aus der \textit{}Alchemie\
Achat: edler Stein, Sinnbild für Emsigkeit und Fleiß, aber auch für Güte. Er vermag verwirrte Sinne zu heilen, aber auch Melancholie.\
Adeptus: (A. minor, A. major) erster echter Titel eines Schülers, jemand, der die Lizenz erworben hat, seine Kunst frei zu lernen.\
Aeromantie: Wahrsagen aus den Erscheinungen der \textit{}Luft, wie z.B. aus der Form der Wolken.\
Aeternom: Sammelbegriff für Artefakte, die semipermanent oder permanent wirken.\
Affinitäten (1): aus der Verschiedenheit der \textit{Elemente} ergibt sich bei Wahl eines einzelnen stets auch die 
Entferntheit des Gegensätzlichen. Somit sind all die Beziehungen der Elemente untereinander, die Gegensätzlichkeiten und 
ihre Beiordnungen also ihre Affinitäten.\
Affinitäten (2): dies sind bestimmte Materialien, welche von Dämonen begehrt oder verabscheut und welche also zur 
Beschwörung oder Vertreibung eines solchen verwendet werden. Die A. gehören zu der Gruppe der \textit{}Paraphernalia.
Alchimie: Kunst und Wissenschaft zugleich, die sich mit der Veränderung der Materie vom Niederen zum Höheren befasst: 
unterteilt in die \textit{Niedere oder allgemeine Alchimie}, die auch des Öfteren mit der Kräuter- und Trankkunde gleichgesetzt wird, und zu deren Einsatz keine Zauberkunst nötig ist, und die \textit{}Hohe Alchimie, bei der dies sehr wohl der Fall ist.
Allegorie: Gleichnis
Alp: eine bösartige Geistererscheinung, die schwere Alpträume verursachen kann.
Amethyst: edler Stein, er vermag ein Herz in flammende Leidenschaft zu versetzen. Sein Schimmer schenkt seinem Träger Schönheit und weckt Begehrlichkeiten.
Analogien: Nach vielerlei Überlieferungen, vor allem alchimistischer Natur, soll es eine Reihe von Sympathien der Elemente mit Sternbildern, Planeten, Edelsteinen und dergleichen geben.
Anatomie: die Kunde vom menschlichen Körper und seinen Funktionen.
Antimagie: unter dem Begriffe der Antimagie oder Contra Magica versteht man all jene Zauberhandlungen, welchselbige zum 
Ziel haben, primo einen Zauber nicht zur Wirkung kommen zu lassen oder secundo eine bereits eingetretene Wirkung 
aufzuheben.
Aquamarin: edler Stein, er symbolisiert die Kraft des Wassers und behütet den Seemann auf Fahrt. Er ist der Stein der Harmonie und Freundschaft.
Archomagus, Archomaga: (auch Arcom.) Titel für Magier, der nur auf einem Konvent durch Beschluss der Gildenräte verliehen wird und die erfahrensten Magier ehren soll. Anrede: Euer Magnifizenz.
Arkanogenese: Erschaffung eines magischen \textit{}Artefakts.
Artefakt: hierbei handelt es sich um einen magischen Gegenstand, indem eine oder mehrere Zauber gespeichert sind und der auch von Nichtzauberern genutzt werden kann.
Astralreise: das Loslösen des Astralleibes vom physischen Körper. 
Astrolabium: feinmechanisches Rechenwerk. Zeigt für ein eingestelltes Datum den jeweiligen Stand der Sternbilder an.
Astrologie: Die A. beschäftigt sich im Gegensatz zur \textit{}Astronomie mit der Deutung von Sternenkonstellationen und 
deren Einfluss auf die Geschicke der Sterblichen.
Astronomie: die A. bezeichnet im Gegensatz zur \textit{}Astrologie die Sternkunde im Sinne der Erstellung und Erforschung 
von Sternkarten, insbesondere zur Navigation.
Bann: dies ist die korrekte Anlage von Beschwörungs-Heptagramm, Schutzkreis sowie der Zeichen des zu beschwörenden Daimon, 
auch der Schutz- und Bannmächte als Teil der \textit{}Paraphernalia.
Beherrschungsmagie: als Beherrschungsmagie oder Magica Contollaria kennen wir all jene Arten der Zauberei, welche in erster 
Linie auf den Geist des Opfers, will heißen, seinen Willen, sein Erinnern, gar auf seinen Instincto, nie jedoch auf seine 
Seele wirken.
Bergkristall: edler Stein, Sinnbild der Erde. Er vermag Wunden zu heilen und den Körper zu stärken damit er schädlichen 
Einflüssen gewappnet ist.
Bernstein: edler Stein, er ist das Symbol der Sonne und steht für den ersten Mond im Jahr. Sinnbild für Wahrhaftigkeit, 
Strenge und Gerechtigkeit.
Beschwörungszauberei: Wir betrachten die B. oder Magica Conjuratio als Kunde von den Jenseitigen, allweil ähnliche Riten 
für \textit{}Geister und \textit{}Daimonen Wirkung zeigen, so man sie in Persona invociren will. Item gehöret auch zur B. 
die Macht, Chimären und lebende Stand-Bilder zu machen, Pest und Siechtum zu conjuriren und letztendlich die finstere, 
lästerliche \textit{}Necromantia oder Untoten-Erhebung.
Blutachat: \textit{}Karneol
Blutmagie: „Mächtig aber ist das Blut, denn siehe! Es heilt die Kraft, so es von vornherein von Zaubertieren kommt, es gibt dir Kraft, so es nur von gemeinem Getier stammt. Am mächtigsten aber ist das Blut von Mensch und Elf. Um mit deinem Blute oder dem anderer zu zaubern, musst du einen eisernen Willen haben, denn es ist schiere Kraft am Werk, wo sonst Raffinesse und Kenntnis.
Canton: allg. Zauberformel oder Spruch.
Chimäre: mittels schwarzer Magie erzeugtes Mischwesen aus zwei oder mehr verschiedenen Tierarten, z.B. Harpyie, Greif oder Schlangenmensch.
Collega, Collegus: Anrede der Magier untereinander
Conjuratio: Herbeirufung transsphärischer Wesenheiten. \textit{}Beschwörungszauberei
Contramagie: \textit{}Antimagie
Convocatus: Titulatur für die Mitglieder der Gildenräte. Anrede: Euer Spektabilität.
Convocatus Primus: Sprecher der Gildenräte. Anrede: Euer Spektabilität.
Daimon: jenseitiges Wesen, Bewohner der anderen Sphären: kann im Diesseits nur existieren, wenn er von einem Zauberkundigen 
beschworen wird. Man unterscheidet zwischen \textit{}niederen Daimonen und \textit{}höheren Daimonen.
Daimonologie: (Daimonologica) \textit{}Beschwörungszauberei
Demonstratio: praktischer Teil der \textit{}Examinatio, bei dem der \textit{}Adeptus minor alle gelernten Sprüche, z.T. unter erschwerten Bedingungen demonstrieren muss.
Diamant: edelster der Steine, Sinnbild für die Vollkommenheit und die arkane Kraft. 
Diplomante: \textit{}Magus
Disliberatio: mittelschwere Form der Bestrafung für Magier. Verbot der Benutzung aller Bibliotheken.
Disputatio: theoretischer Teil der \textit{}Examinatio, bei dem die Kenntnisse bzgl. des weltlichen Wissens, der \textit{}Alchimie, der Rechtskunde und der klassischen Sprachen überprüft werden.
Disvocatio: leichteste Form der Bestrafung für Magier. Ausschluss von allen höheren Ämtern wie der Akademieleitung oder der Mitgliedschaft im Gildenrat.
Donaria: \textit{}Paraphernalia
Donarium: so nennt man die Opfergabe eines bestimmten, seltsamen, dem \textit{}Daimon gefallenden Gegenstandes oder 
Tieres. Gehört zu den eventuellen \textit{Paraphernalia} einer Beschwörung (\textit{Beschwörungs­zauberei}).\
Dottore: \textit{}Magister ordinarius
Drachen: allen Drachen gemein sind die schuppenbedeckte Haut, die geschlitzten Augen, die sie zur Nachtsicht befähigen, der meist sechsgliedrige Rumpf (von dem ein Beinpaar häufig zum Flügelpaar umgebildet ist) und der lange Schwanz. Es existieren verschiedenste Arten von Drachen ihnen allen eigen ist das man sie nur äußerst selten in den Mittellanden zu Gesicht bekommt. 
Drow: eine den an Erscheinung den elfen ähnliche Rasse, mit dunkler Haut und meist weißen Haaren. Sie leben unter der Erde und haben einen unbändigen Hass auf alle anderen Lebenden Wesenheiten.
Dschinn: \textit{}Elemtharii
Dunkelelf: \textit{}Drow
Ebene: \textit{}Sphäre
Einhorn: vielerlei Namen hat das stolze Ross mit dem Horne: Einhorn nennt es die gemeine Bauernperson, Unicornu der Gelehrte, die Zwerge Ligorn und die Elfen Valdra. Es ist ein Ross mit schneeweißem Fell, langer weißer Mähne und einem ebensolchen Schweif. Auf der Stirn besitzt es ein güldenes, zwiefach gedrehtes Horn von fast drei Spann Länge. Auch Tiere mit silbernem Horn oder gar von nachtschwarzer Farbe mit einem Horn in der Farbe des Blutes sollen schon gesichtet worden sein.
Elementarbeschwörung: die E. ist in der Tat eine \textit{}Invocatio und keine \textit{}Conjuratio, allweil die \textit{}Elementharii nicht jenseitig, sondern sehr wohl unserer Sphäre zugeordnet sind. 
Elementare Transition: dies bezeichnet die Ver­schiebung eines Zaubers von einer elementaren Zu­ordnung in eine andere.
Elementares Hexagramm: \textit{}Siegel der Elemente
Elementargeist: \textit{}Elementharii
Elementharii: eine beseelte Repräsentation elemen­tarer Kräfte.
Elementarismus: \textit{}Elementarbeschwörung
Elemente: \textit{}(Feuer, Wasser, Erde, Luft), die Elemente (auch: magische Elemente) sind die Bestandteile und Grundbausteine unserer Sphäre und alles Seins. Ohne sie gäbe es keine Existenz und keine Substanz.
Eleve: \textit{}Scolar
Erde: Erde und Natur. Hier findet sich alle belebte Materie wieder vom lebensspendenen Humus, über Käfer und Würmer, Kräuter und Bäume bis hin zu den großen Tieren und den kulturschaffenden Rassen, aber auch Holz und Leder, Knochen und welkes Laub. Kein Wunder also, dass die wichtigste Eigenschaft der Erde das Leben in all seinen Formen ist. (\textit{}Elemente)
Erzmagier, \textit{}Archomagus, Archomaga
Examinatio: Abschlussprüfung an einer magischen Akademie, bestehend aus \textit{}Disputatio und \textit{}Demon­stratio.
Expurgico: schärfste Form der Bestrafung für Magier. Streichung aus der Akademieliste, Entfernung des Siegels, dadurch kein Schutz mehr durch das Gildenrecht.
Fee: die wichtigsten der Arten sind die kleinen Blütenjungfern, auch Ladifari genannt, kaum spannen­groß in Blüten hausend, die Dryaden oder auch Nymphen, denen Pflanzen und Gewässer zu eigen sind, die greulichen Biestinger in aufrechter Tiergestalt, die Holden, von denen man spricht, sie seien die Vorfahren der Elfen, und die Wichtel und Kobolde, die wohl im Feenreiche als auf unsrer Seite der Tore zuhause sind.
Feuer: Das Feuer steht nicht nur für das Herd- oder Lagerfeuer, sondern ist die Essenz von Wärme und Licht, Umwälzung und Vernichtung. Im wohnt eine große Kraft inne. (\textit{}Elemente)
Gargyl: wenn dem Steine Leben eingehaucht wird, dass er ganz und gar einem Magus ergeben ist, so ist dies allemal wider die Natur und den Codex. Doch wisset von den Wasserspeiern, dass jener Magus, der sie ehedem erschuf, ihnen das Geheimnis des Lebens mit auf den Weg gab, ob zum Guten oder Bösen.
Geister: gemeinhin die Seele eines Verstorbenen, der noch nicht ins Totenreich hinüber getreten ist.
Geselle: Einteilung bei freien Lehrmeistern, gleiche Stufe wie ein \textit{}Adept an einer Akademie
Ghul: diese menschenähnlichen Wesen besitzen eine graugrüne Haut, einen vorspringenden Kiefer mit gefährlichen Reißzähnen und lange Klauenhände. Um ihren Leichenschmaus abzuhalten, verlassen sie nur des Nachts ihre Gräber, denn das Licht der Sonne tötet sie.
Gilde: Zusammenschluss mehrerer Akademien oder freier Magier
Golem: ein aus Erde oder Stein bestehendes Wesen das durch magische weise zum Leben erweckt wurde.
Greif: edles Wesen. Ein adlerköpfiger Löwe mit mächtigen Schwingen. 
Harpyie: nichts ist widernatürlicher als eine H. Halb Frau, halb Vogel und wirren Verstandes sind diese Ausgeburten der Schwarzen Magie. Harpyien kann man nie trauen, sie sind vollkommen unberechenbar.
Heilmagie: Magie die auf den Körper wirkt um Wunden und Krankheiten zu heilen.
Hermetica Destructiva: \textit{}Kampfmagie
Hexalogien: Wenn es die elementare Manifestation einer Zauberwirkung gibt, wieso soll nicht auch für jedes \textit{}Element ein solcher Zauber existieren? Unter H. verstehen wir also, dass sich eine Zauberwirkung durch verschiedene Elemente erreichen lässt.
Hippogriff: in entfernten Gebieten soll es Kreaturen geben, die aus der Brunft der Greifen entsprungen sind. Sie scheinen mit dem Pferd gekreuzt und bieten den Anblick einer Stute oder eines Hengstes mit dem Kopf, den Schwingen und den Fängen eines Adlers.
Hohe Alchimie: Wissenschaft der Verwandlung von Materie. Im Gegensatz zur \textit{}niederen Alchimie oft unter Einsatz von Magie und komplexen Strukturen. (\textit{}Alchimie)
Höhere Daimonen: eine Klasse von Wesen der anderen Sphären. Sie stehen in ihrer Macht deutlich über den \textit{}Niederen Dämonen.
Hochmagier: \textit{}Archomagus/a
Hofmagicus: Hofmagier (sowie meist Astrologe und Traum­deuter) eines Lehnsherren.
Hypervehemenz: gleichzeitige Auslösung aller gestapelten Formeln zu einem Gesamteffekt.
Irreversibilität: die Unwiderrufbarkeit einer Formel, also eines arkanen Musters, oder aber auch der Wirkung einer Formel, z.B. die I. der \textit{}Antimagie.
Ingredenzien: Zutaten eines alchimistischen Rezeptes. (\textit{}Alchimie)
Invocatio: Herbeirufung (normalerweise eines Elementes: auch \textit{}Dschinne etc.)
Invocatio Daimoniae: \textit{}Beschwörungszauberei
Invocatio Elementharii: \textit{}Elementarbeschwörung
Kadunom: Sammelbegriff für alle Artefakte, die auf Ladungen basieren.
Kampfmagie: Magie die zur Unterstützung im Kampfe oder zum direkten Schaden eines Feindes eingesetzt wird. Einteilung erfolgt meist nach Ziel des Zaubers und nicht nach der direkten Wirkung.
Karneol: edler Stein, er ist das Sinn­bild für Ruhe, Vergessen, innere Kraft, aber auch für das unwiderrufliche Ende. Er spendet den Trauernden Trost, den Rastlosen Ruhe, dem Zornigen Frieden.
Kender: kleine Geschöpfe mit meist kindlichem Gemüte, empfinden keine Angst und finden oft Dinge die ihnen nicht gehören.
Klabautermann: besondere \textit{}Kobolde sind die K., die ein wahrlich uraltes Gesicht auf einem Kindskörper zur Schau tragen. Man findet sie häufig auf alten Schiffen, denn sie sind der fleischgewordene Geist eines solchen Fahrzeugs oder des Holzes, aus dem es gebaut wurde.
Kobold: gar sonderlich anzuschauen sind die Kobolde, welche die Menschen allzu gerne durch ihre Schelmereien plagen. Sie besitzen die Gabe der Zauberei und sind nach menschlichen Ansätzen oft von sehr seltsamem Wesen. Man sagt, dass ein Mensch Macht über einen K. gewinnt, wenn er seinen Namen kennt und ruft.
Komponenten: Zutaten und Gegenstände die für eine Zauberhandlung notwendig sind.
Lapislazuli: edler Stein, in ihm findet sich die Kraft der Luft. Er ist der Schutzstein der Wanderer und Abenteurer. Er weist dem Reisenden den Weg und vermag durch seine Kraft, Unglücksfälle zu verhüten.
Lehrling: Einteilung bei freien Lehrmeistern, gleiche Stufe wie ein \textit{}Scolar an einer Akademie.
Licenzore: \textit{}Adeptus   
Luft: Wind und Sturm. Dies ist nicht nur die Luft, die wir atmen, es ist auch der Wind, vom lauen Lüftchen bis hin zum entfesselten Sturm. Demzufolge sind mit der Luft auch Eigenschaften wie Bewegung und Flüchtigkeit verbunden. (\textit{}Elemente)
Lykanthropie: \textit{}Werkreatur
Magia Anime: \textit{} Beherrschungsmagie
Magia Galad: \textit{}Kampfmagie
Magia Medicam: \textit{}Heilmagie
Magia Sanctum: \textit{}Schutzmagie
Magica Combativa: \textit{}Kampfmagie
Magica Conjuratio: \textit{}Beschwörungszauberei
Magica Contraria: \textit{}Antimagie
Magica Controllaria: \textit{}Beherrschungsmagie
Magica Transformatorica: \textit{}Verwandlungsmagie
Magiergilde: \textit{}Gilde
Magister, Magistra: Titel nach erfolgreicher Meisterprüfung.
Magister magnus: M. m. dürfen sich all jene nennen, die entweder als Lehrstuhlinhaber an Akademien oder als freie Meister das Recht haben, Eleven und \textit{}Scolaren anzunehmen und ein eigenes Siegel zu führen.
Magister spectabilitas: Akademieleiter.
Magnifizenz: Anrede für einen \textit{}Erzmagier.
Magus, Magica: Die Bezeichnung für alle Gelehrten, die ihre \textit{}Examinatio bestanden, und die Lizenz erworben haben, ihre Wissenschaft zu praktizieren. Anrede: gelehrte/r Dame/­Herr
Magister extraordinarius: freier Lehrer oder nicht ständiger Lehrer einer Akademie.
Magister ordinarius, Magia ordinaria: (Akademie-) Lehrer bzw. Lehrerin, bisweilen auch Titel eines an­erkan­nten Großmeisters.
maiestro de thesis...: Meister der Thesis.....
Matrix: Gitter oder Netz, welches als Medium für eine Spruch­wirkung dient. Matrices werden unterteilt in M. naturalis (nat., solche die natürlich in der Welt existieren und recht unstrukuriert sind), M. artificialis (art., von Zauberkundigen selbst erschaffene M.) und M. arcana (arc., sehr komplexe in hochmagischen Körpern vor­kommende M.)
Metamagie: zur M. gehören jene Formeln, welche ver­ändernd auf Matrices artificialis und arcana wirkt: \textit{}Matrix.
Metaphorische Transition: dies bezeichnet die Ver­schiebung eines Zaubers von einem Spezialgebiet in ein anderes.
monophil: Eigenschaft eines Artefakts, das nur von einem einzigen Träger üblicherweise dem Erschaffer - angewendet werden kann.
Nachtalp: \textit{}Alp
Nanduria: (auch: Nandus-Schrift) die 26 Zeichen des Nanduria werden bei Magiern, Alchimisten und einige anderen Gelehrten zur Aufzeichnung von Geheimnissen benutzt, aber auch bei Gravuren oder der Herstellung magischer \textit{}Artefakte finden sie Anwendung.
Necromantia: (Nekromantie) die N. ist die Lehre von den \textit{}Untoten. Sie befasst sich mit der gar ­lästerlichen Erweckung, aber auch mit der Beherrschung derselben. Die N. gilt allgemein als ver­abscheuungs­würdigste Form der Zauberei und ihre Aus­übung steht wohl vielerorts unter harter Strafe.
Necropathia: dies ist die Ver­ständigung mit den Verstorbenen, welche auf magische Weise geschieht.
Niedere Alchimie: (manchmal auch Trankkunde), die n. A. beschreibt im Gegensatz zur \textit{}Hohen Alchimie die Herstellung von einfachen Rezepten und Tränken die meist auf der direkten Wirkung ihrer Ingredenzien basieren. (\textit{}Alchimie)
Niedere Daimonen: unterste Klasse der \textit{}Daimonen. 
Novize: \textit{}Scolar
Obsekrator: Artefakt, dessen Auslösung eine \textit{}Beschwörung bewirkt.
Okkupation: Beseeltheit eines Artefakts: astrale/­elementare/­­transnekrotische/­daimonische Okkupation.
Onyx: edler Stein, er hat Macht über \textit{}Daimonen und \textit{}Geister und über die Gefilde, in denen die Toten wandeln. Er vermag den Kundigen sicher durch die verschlungenen Pfade der astralen Sphäre zu geleiten und bewahrt ihn davor, darin verloren zu gehen.
Opal: edler Stein, in ihm zeigen sich alle Farben, er ist das Sinnbild der Vielfalt. Er hat die Macht die Jugend zurückzugeben.
Paraphernalia: so werden die Randbedingungen ge­nannt, welche es bei Beschwörungen von \textit{}Daimonen zu beachten gilt. Dazu gehören einmal das \textit{}Temporalium, der Bann, die \textit{}Affinitäten sowie unter Um­ständen das \textit{}Donarium. (\textit{}Beschwörungszauberei)
Präservanz: Wirkungszeitraum der gespeicherten Sprüche: einmalig, aufladbar, semipermanent, permanent.
Parenthese: Nebeneffekt, der nicht durch die wirkenden Sprüche bewirkt wird.
Poltergeist: Erscheinungsform eines \textit{}Geistes, meist an einen bestimmten Ort gebunden.
Quarz: \textit{}Bergkristall
Rattenmenschen: \textit{}Skaven
Rehermetifikation: Wiederaufladung, Wideraufladbarkeit eines Artefaktes.
Rettore, Rettrice: \textit{}Magister spectabilitas
Rubin: edler Stein, in ihm findet sich die Kraft des Feuers. Mit ihm kann man die Flammen beherrschen, er kann aber auch den lodernden Zorn in der Brust eines Kriegers wecken.
Runenschrift: meist alte Schrift die verschiedene Zeichen nutzt. Es existieren viele verschiedene Formen von Runenschriften.
Saphir: edler Stein, das Sinnbild der Rein­heit und Beständigkeit, der Stein der Liebenden, die zueinander gefunden haben. In ihm liegt die Kraft des Seelenfriedens und der Freundschaft.
Satyr: Wesenheit mit menschlichem Körper aber den Beinen einer Ziege.
Scolar: (auch: Eleve, Novize, Studiosi, Lehrling) Bezeichnung für einen Schüler einer Magierakademie oder eines freien Lehr­meisters, je nach Wissensstand. 
Siegel der Elemente: nach dem „\textit{}Groszen Elemen­tharium stellt dieses die allgemein gültigen Gesetze und Gegensätze der Elemente dar. (\textit{}Elementarbeschwörung)
Skaven: kleine einer Ratte ähnliche Geschöpfe die tief in der Erde hausen. Sie nutzen oft die Kraft magischer Steine, allen voran den seltsamen \textit{}Wechselstein. 
Smaragd: edler Stein, er steht für Mut, Hoffnung und Treue. Mit ihm beschwört man das Schlachtenglück. Unter seinem Schimmer wird der Feigling zum Helden, der Held aber erfährt ungeahnten Mut.
Sphäre: alle Ebenen jenseits der unsrigen.
Spektabilität (1): Anrede für einen \textit{}Convocatus.
Spektabilität (2): \textit{}Magister spectabilitas
Sternkunde: \textit{}Astrologie,\textit{} Astronomie
Spezialgebiet: allgemein gültige, auf dem „Groszen Arcanum beruhende Einteilung der verschiedenen Bereiche der Zauberei nach ihrem Ziel und Zweck, nicht aber nach den Modalitäten des Zaubervorgangs. Ein S. der Magie, der ein Spruch zuzurechnen ist, z.B. Kampf.
Schutzmagie: Magie die zum Schutze der eigenen oder anderer Personen genutzt wird
Studiosi: \textit{}Scolar
Technik (eines Zaubers): dies sind die Zauber­hand­lungen, die vom Magiekundigen auszuführen sind, sei dies das laute Rezitieren einer Formel oder eine Hand­bewegung.
Temporalium: dies sind die Umstände der Be­schwörung, welche seien die korrekte Tageszeit, der kor­rekte Ort und besondere Sternkonstellationen. Diese müs­sen natürlich mit den restlichen \textit{}Para­phernalia der Be­schwörung übereinstimmen. (\textit{}Beschwörungs­zauberei)
Thesis: dies ist die Summe von komplexen geo­metrischen Mustern, den Kraftfäden der \textit{}Matrix, welche die Eigenschaften eines Zaubers und die Zauberformel beinhalten.
Thaumaturgie: Artefaktzauberei, eigentlich alle Aus­formungen der Hohen Magie, im Gegensatz zur Profan­magie der Scharlatane und Naturzauberkundigen.
Thaumatursom: Träger der Verzauberungssprüche: das Artefakt, bisweilen auch Sammelbezeichnung für \textit{}Aeternome und \textit{}Kadunome.
Trankkunde: \textit{}niedere Alchimie
Transmutatio: eine größere Freveltat noch als die Be­lebung von Toten zu Untoten ist das Erschaffen völlig künstlichen Lebens Transmutatio welches allen gött­lichen und natürlichen Gesetzen widerspricht und daher nur mittels \textit{}Beschwörungszauberei zu voll­bringen ist.
transsphärische Wesenheiten: Bewohner der äußeren \textit{}Sphären.
Türkis: edler Stein, er ist zugleich Glücks­bringer und Sinnbild für Unbeständigkeit. Er lässt Herzenswünsche wahr werden. Er ist der Freund des Spielers, verleiht den Händen magische Flinkheit.
ultima occasio: günstigste Vorgaben zur Artefakt­erschaffung.
Unicornu: \textit{}Einhorn
Untote: von den wandernden Toten: Alle Nicht-Toten sind gar schauerlich anzusehen und von üblem Geruch umgeben. Wer sie berührt, den rafft die Pest dahin. Nicht-Tote sind unheilig, denn ihr Dasein ist ein Frevel.
Vampir: Vampire haben spitze Zähne und beißen ihre Opfer in den Hals um ihr Blut zu trinken. Viele können sich in Fledermaus und Wolf verwandeln, sowie selbige herbeirufen. Sie hassen Knoblauch, heilige Symbole und geweihtes Wasser. Sie fürchten sich vor dem hellen Lichte der Sonne. Werden sie bezwungen, fliehen sie an einen dunklen Ort, oft in ihren Sarg. Getötet werden sie mit einem Holzpflock von Weißdorn oder Eiche, welchen man ihnen direkt ins Herz treiben muss.
Verwandlungsmagie: Magie die einen direkten Einfluss auf Form, Substanz oder andere stoffliche Eigenschaften eines Objektes hat.
Waldmännchen: \textit{}Kobold
Wasser: Feuchtigkeit und Flüssigkeit. Im Wasser finden nicht nur Tiefe und Unergründlichkeit ihre Re­präsen­tation, sondern vor allem die Veränderlichkeit und die Kraft der langsamen, aber stetigen Veränderung. Unergründlicher Geist und auch Melancholie und Wahn­sinn werden mit dem Wasser assoziiert, ebenso wie Wandelbarkeit, Beweglichkeit und Eleganz. (\textit{}Elemente)
Wasserspeier: \textit{}Gargyl
Wechselstein: seltsamer Stein, mit unbekannter magischer Kraft. Wird oft von \textit{}Skaven für allerlei schreckliche Dinge genutzt.
Werkreatur: Fluch und Krankheit gleichermaßen ist die Lykanthropie, die Menschen in wilde Tiere ver­wandelt: Die Werwesen. Wenn der Mond voll ist, bricht das Wertier hervor. Manche Opfer werden gänzlich zum Tier, andere verwandeln sich in ein Mischwesen. Nicht Gewalt noch Vernunft können sie halten, ihre Blutgier ist größer als die jeder Kreatur, und nichts kann sie erlösen als eine Klinge von purem Silber.
Widergänger: \textit{}Untote
Xenophon: unbekannter wirkender Spruch.
Zhayad: eine besonders bei Beschwörern (\textit{}Be­schwörungs­zauberei) beliebte Geheimsprache und Schrift, von der behauptet wird nur in ihr könne man die korrekten Formeln sprechen. 
Zombie: \textit{}Untote
Thalian Sanderfels, überarbeitete Version, 12. Februare 1004

\appendix



\chapter{ "uber den Autor}

\yinipar{F}lorian Phelleas Ph"onixflug wurde im Jahre 232 nach Jeldrik auf Burg Hanekamp in Engonien gebohren. Durch seinen Vater Aldus, Haushofmagier des Herzogs von Hanekamp und seine Mutter Anastasia seit fr"uhester Kindheit gef"ordert schlug er schon in jungen Jahren eine hervorragende akademische Karriere ein.
Mit zw"olf Jahren wurde er von Magus Galad et Sanctum Cederic Greifenhorst, als Lehrling angenommen und trat nach Gr"undung des Bundes zu Ayd Owl durch seinen Meister, dem Bund als einer der ersten Sch"uler bei. Im Alter von 16 Jahren wurde er Lehrling des Bundes zu Ayd Owl und im Alter von 18 Jahren zum Scolarius.
Als Scolarius absolvierte er Gaststudien an der Akademie zu Amonlonde und der Cantus Harmonae zu Condra. Ferner belegte er bereits w"ahrend dieser Lehrzeit den zweiten Platz im offenen Ritualkontest der Akademie zu Montralur.
Im Alter von 23 Jahren trat er in die Dienste des Markgrafen Jerevan zu Arkenwald aus Grenzbrueck und wurde von der Akademia zu Ayd Owl in den Adeptenstatus erhoben. In den Diensten des Markgrafen erhielt er einen Lehrauftrag f"ur die Ars Bellorum von Arkenwald.
Mit 25 Jahren trat er der Akadmia Clavis Mundi Grenzbrueckensis im Range eines Adepten bei und begann weiterf"uhrende Forschungen auf dem Gebiet der Ars Magica Temporalis.
Mit 26 Jahren promovierte er mit seiner Arbeit "uber Deterministisches Schicksal und Chaos an der Akademia Clavis Mundi als Magus Minor.
Seit dem lehrt und forscht er als A"sistent von Senator, Magus Galad et Contra, Praeceptor Primus Galadmagica Erszbet Lauschenberg im Bereich der Galadmagica und der Ars Magica Temporalis an der Akademia Clavis Mundi Grenzbrueckensis.





\end{document}

